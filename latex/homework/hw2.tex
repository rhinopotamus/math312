\documentclass[12pt]{article}
% controlling the geometry of the page:
\usepackage[margin=1in, paperwidth=8.5in, paperheight=11in]{geometry} 
\usepackage{amsmath, amssymb} % useful math symbols and environments

\usepackage{amsthm} % Theorem-like environments
\theoremstyle{definition} % Without this line, theorem statements (and therefore problem statements etc.) show up in italic text.
\newtheorem{conjecture}{Conjecture}
\newtheorem{problem}{Problem}

% pretty colors!
\usepackage[dvipsnames]{xcolor}
\colorlet{darkgrey}{black!70}
\colorlet{darkgreen}{green!50!black}

\usepackage{graphicx} % for inserting figures with \includegraphics
\usepackage{setspace} % for controlling space between lines, paragraphs, etc.

\usepackage{tikz} % for drawing diagrams
\usetikzlibrary{arrows,automata,positioning} 
\usetikzlibrary{decorations.markings}
\usetikzlibrary{decorations.pathreplacing}
\usetikzlibrary{patterns}
\usetikzlibrary{shapes.geometric}

\usepackage{fancyhdr} % for controlling headers and footers
\usepackage{newtx} % changes the default font family
\usepackage{enumitem} % controllable labels for ordered and unordered lists

\usepackage{hyperref} % controls hyperlinks, both internal and external
\hypersetup{
    colorlinks=true,
    urlcolor=blue,
}

\setlength{\headheight}{14.5pt}
\newcommand{\Q}{\mathbb{Q}}
\newcommand{\R}{\mathbb{R}}
\newcommand{\Z}{\mathbb{Z}}
\newcommand\inv{^{-1}} % I am very tired of typing ^{-1}
\DeclareMathOperator\Rect{\mathbf{Rect}}
\DeclareMathOperator\Tri{\mathbf{Tri}}
\DeclareMathOperator\Sq{\mathbf{Sq}}
\DeclareMathOperator\Light{\mathbf{Light}}

\newenvironment{red}{\color{red}}{\ignorespacesafterend}

% I don't like how LaTeX renders section headings by default
\renewcommand{\section}[1]{\begin{center} \textbf{#1} \\\end{center}}
%
\setlength{\parindent}{0in}
%\oddsidemargin=-.25in
\allowdisplaybreaks
\pagestyle{fancy}
\renewcommand{\headrulewidth}{0pt}
\lhead{MATH 312}
\rhead{Spring 2025}
%\lfoot{\copyright\ CLEAR Calculus 2010}
\cfoot{}
\renewcommand{\thefootnote}{*} 
\hyphenpenalty=10000 % LaTeX by default really likes hyphenating things

% all the stuff above this line is called the preamble...
%##################################################################
\begin{document} % this is always the first line of what's actually produced
\section{Homework \#2} % notice that if you want the character # to appear, you have to "escape" it with a backslash

Here's your homework for this week, due Sunday 2/2 by pdf upload to Canvas. I'm also posting the .tex source on the \href{https://github.com/rhinopotamus/math312}{MATH 312 github repo}.

I'm breaking this document into a couple of chunks. Don't feel like you need to work the chunks in order, nor like you must follow the order within each chunk.

\subsection*{Conjectures from Wednesday's hw}

You've already started thinking about these; now it's time to write up formal arguments. I've provided a proof of Conjecture \ref{inv-unique} so that you can see (1) what the writeups should look like, (2) what kind of tricks we have up our sleeves in an algebra class (see the italicized notes), and (3) how to use the proof environment in LaTeX.

\begin{conjecture}[Andrew]
If $g^2 = e$ for every element $g\in G$, then $G$ is abelian.
\end{conjecture}
Hint: Problem \ref{sns} below may be helpful.

\begin{conjecture}
    If $x$ is idempotent (that is, $x^2 = x$), then in fact $x = e$.
\end{conjecture}

\begin{conjecture}
    \label{inv-unique}
    Inverses are unique. (More precisely: Every element has a unique inverse.)
\end{conjecture}
\begin{proof}
    \textit{(To show that something is unique, it is often nice to say, suppose there were two of them, oh wait, they're actually equal.)}

    Let $g\in G$. By the inverse axiom, there exists $g\inv \in G$ such that $g\inv g = g g\inv = e$.

    Suppose that there was some element $h\in G$ such that $h g = e = gh$. 

    \textit{(We are working in a group, so the good operation we have is to multiply stuff together:)}

    Consider $g\inv g h$. \textit{(I'm going to resolve this in two ways and then use the associative property to assert that they're the same.)} 
    
    On the one hand: 
    \[g\inv g h = (g\inv g) h = eh = h.\]
    On the other hand, 
    \[g\inv g h = g\inv(gh) = g\inv e = g\inv.\] 
    
    By the associative property, these two things must be equal, so $h = g\inv$. Therefore, the inverse of $g$ is unique.
\end{proof}

\begin{conjecture}
    The identity element is unique. (More precisely: Every group has a unique identity element.)
\end{conjecture}

\begin{conjecture}
    The sudoku property of group tables is true. (More precisely: \underline{\hspace{1in}})
\end{conjecture}

\begin{conjecture}
    Even in a non-abelian group, every element commutes with some other elements.
\end{conjecture}

\subsection*{Some basic facts that seem obvious}

\begin{problem}[Shoes `n' socks theorem]
    \label{sns}
    Express $(ab)\inv$ in terms of $a\inv$ and $b\inv$. Prove you're right.
\end{problem}

\begin{problem}
    What is $(a\inv)\inv$? Prove that you're right.
\end{problem}

\begin{problem}
    Prove that every cyclic group is abelian.
\end{problem}

\subsection*{Dihedral groups}

\begin{problem}
    In a generic dihedral group $D_n$, prove that $r\inv = r^{n-1}$.
\end{problem}

\begin{problem}
    Consider $D_4$, aka $\Sq$. Lately we've been generating it with one rotation $r$ and one reflection $f$, but there were actually four different reflections. To give them names, let's say that the horizontal, vertical, up-diagonal, and down-diagonal reflections are called $h$, $v$, $u$, and $d$, respectively. Find two reflections that generate all of $D_4$, and draw the Cayley diagram that results.
\end{problem}

\subsection*{Matrix groups}
Groups whose elements are matrices and whose operation is matrix multiplication are a very useful source of examples. Let's explore a few such groups. It may be helpful to go back to what you learned in linear algebra to remember how matrix multiplication and inverses work.
\begin{problem}
    Consider the following subset of the big ol' universe of all $3\times 3$ matrices:
    \[ H = \{
    \begin{pmatrix} % "parentheses matrix" -- use bmatrix if you like brackets
        1 & a & b \\
        0 & 1 & c \\
        0 & 0 & 1
    \end{pmatrix} \mid a, b, c\in \R\}.\]
    \begin{itemize}
        \item Show that $H$ is \textit{closed under multiplication}: if $X, Y \in H$, then $XY \in H$ also.
        \item Show that the identity matrix is an element of $H$.
        \item Show that if $X\in H$, then $X\inv \in H$.
        \item In general, matrices don't necessarily commute. Do matrices in $H$ commute? Either prove that they always do, or give an explicit example of $X, Y \in H$ such that $XY \neq YX$.
    \end{itemize}
    ($H$ is called the \textit{Heisenberg group} and it has something to do with quantum mechanics.)
\end{problem}

\begin{problem}
    Consider the group generated by $A$ and $B$, where
    \[A = 
    \begin{pmatrix}
        0 & 1 & 0 \\
        1 & 0 & 0 \\
        0 & 0 & 1
    \end{pmatrix} \quad \text{and} \quad
    B = 
    \begin{pmatrix} %This is bad practice but you should know it's possible:
        1 & 0 & 0 \\ 0 & 0 & 1 \\ 0 & 1 & 0 
    \end{pmatrix}.
    \]
    Draw a Cayley diagram, give a presentation, and write a multiplication table. Who is this group isomorphic to? (It's one of the ones we've seen before!)
\end{problem}

\subsection*{The quaternion group $Q_8$}

The quaternion group $Q_8 = \{1, -1, i, -i, j, -j, k, -k\}$ is some kind of generalization of how complex numbers work. It is generated by three different ``imaginary numbers'' $i$, $j$, and $k$, with $i^2 = j^2 = k^2 = -1$. They multiply in the following non-commutative way:
\begin{align*}
    ij &= k & ji &= -k \\
    jk &= i & kj &= -i \\
    ki &= j & ik &= -j
\end{align*}

\begin{problem}
    Draw a Cayley diagram and a multiplication table for $Q_8$.
\end{problem}

\begin{problem}
    Find the orbit of every element in $Q_8$ and draw a cycle diagram.
\end{problem}
\end{document}

