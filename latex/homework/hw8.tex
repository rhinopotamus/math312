\documentclass[12pt]{article}
% controlling the geometry of the page:
\usepackage[margin=1in, paperwidth=8.5in, paperheight=11in]{geometry} 
\usepackage{amsmath, amssymb} % useful math symbols and environments

\usepackage{multicol} % multiple columns side-by-side

\usepackage{amsthm} % Theorem-like environments
\theoremstyle{definition} % Without this line, theorem statements (and therefore problem statements etc.) show up in italic text.
\newtheorem{conjecture}{Conjecture}
\newtheorem{problem}{Problem}
\newtheorem*{remark}{Remark}
\newtheorem*{definition}{Definition}

% pretty colors!
\usepackage[dvipsnames]{xcolor}
\colorlet{darkgrey}{black!70}
\colorlet{darkgreen}{green!50!black}


\usepackage{tikz} % for drawing diagrams
\usetikzlibrary{arrows,automata,positioning} 
\usetikzlibrary{decorations.markings}
\usetikzlibrary{decorations.pathreplacing}
\usetikzlibrary{patterns}
\usetikzlibrary{shapes.geometric}

\usepackage{visualalgebra}
\usepackage{graphicx} % for inserting figures with \includegraphics
\usepackage{setspace} % for controlling space between lines, paragraphs, etc.

\usepackage{fancyhdr} % for controlling headers and footers
\usepackage{newtx} % changes the default font family
\usepackage[shortlabels]{enumitem} % controllable labels for ordered and unordered lists

\usepackage{hyperref} % controls hyperlinks, both internal and external
\hypersetup{
    colorlinks=true,
    urlcolor=blue,
}

\setlength{\headheight}{14.5pt}
\newcommand\inv{^{-1}} % I am very tired of typing ^{-1}

% I don't like how LaTeX renders section headings by default
\renewcommand{\section}[1]{\begin{center} \textbf{#1} \\\end{center}}
%
\setlength{\parindent}{0in}
%\oddsidemargin=-.25in
\allowdisplaybreaks
\pagestyle{fancy}
\renewcommand{\headrulewidth}{0pt}
\lhead{MATH 312}
\rhead{Spring 2025}
%\lfoot{\copyright\ CLEAR Calculus 2010}
\cfoot{}
\renewcommand{\thefootnote}{*} 
\hyphenpenalty=10000 % LaTeX by default really likes hyphenating things

% all the stuff above this line is called the preamble...
%##################################################################
\begin{document} % this is always the first line of what's actually produced
\section{Homework \#8 (due Mar 16 or 23, whichever you prefer)} % notice that if you want the character # to appear, you have to "escape" it with a backslash

\begin{definition}
    Let $G$ and $H$ be two groups. A \Balert{homomorphism} $\phi:G\to H$ is a map from $G$ to $H$ that sends products to products:
    \[\phi(g_1 g_2) = \phi(g_1) \phi(g_2).\]
    (Note: An output of $\phi$ is an element of $H$. So, $\phi(g_1)$ and $\phi(g_2)$ and $\phi(g_1 g_2)$ are all elements of $H$.)
\end{definition}

%%======================================================================
\subsection*{Properties of homomorphisms}

The following are straightforward to verify, but writing out the symbols is good practice for ``sending products to products.'' In each part, suppose that $\phi:G\to H$ is a homomorphism.

\begin{problem}\label{properties}
    Figure out how to write each of these in homomorphism language, and prove them.
    \begin{enumerate}[(a)]
        \item $\phi$ sends the identity to the identity. ($\phi(1_G) = 1_H$)
        \item $\phi$ sends inverses to inverses.
        \item $\phi$ sends powers to powers.
        \item $\phi$ sends orbits to orbits.
        \item $\phi$ sends conjugates to conjugates.
    \end{enumerate}
\end{problem}

\begin{problem}\label{orders}
    Prove that if $|g|$ is finite, $|\phi(g)|$ divides $|g|$. Hint: Use \ref{properties}(c) and (a). 
\end{problem}

\begin{problem}
    Prove that $\Image(\phi) \leq H$ (``the image of $\phi$ is a subgroup of $H$'').
\end{problem}

\subsection*{Examples and non-examples}

\begin{problem}
    Show that there is no \Alert{embedding} $\phi\colon\Z_n\into\Z$, for $n\geq 2$. 
    
    Hint: $\phi$ is determined by what it does to generators. $\Z_n = \<1\>$; what can possibly be $\phi(1)$?
\end{problem}

\begin{problem}\label{examples}
    Decide whether or not each of these is a homomorphism.
    \begin{enumerate}[(a)]
        \item The ``projection map'' $\pi_A: A\times B \to A$ defined by $\pi_A:(a, b) \mapsto a$. (Similar for $\pi_B$, btw.)
        \item The ol' switcheroo: define $\gamma:A\times B \to B\times A$ by $\gamma:(a, b) \mapsto (b,a).$
        \item Conjugation by a fixed element: choose $x\in G$ and define $\phi_x:G\to G$ by $\phi_x(g) = xgx\inv$.
        \item The tripling map: $\theta:\Z_6\to \Z_6$ defined by $\theta(m) = 3m$. (Careful: the operation is $+$.)
        \item More generally, $\theta:\Z_n \to \Z_n$ defined by $\theta(m) = km$ for some fixed integer $k$.
        \item Define the ``squaring map'' $s: D_4 \to D_4$ by $s(x) = x^2$. 
        \item For an abelian group $G$, define the ``squaring map'' $s:G \to G$ by $s(x) = x^2$.
        \item Under what circumstances is $\phi:\Z_n \to \Z_m$ defined by $\phi(1) = 1$ a homomorphism? \\(Hint: Does this work for $\Z_3 \to \Z_4$? How about for $\Z_3 \to \Z_6$?)
    \end{enumerate}
\end{problem}

\begin{problem}
    For each of (a), (b), (c), (d), and (e) in Problem \ref{examples}, decide whether the map is an embedding (injective), a quotient map (surjective), an isomorphism (bijective).
    
    For number (e), your answer will be ``it depends:'' under what circumstances is this map injective? surjective? bijective?
\end{problem}

\begin{problem}
    Prove that each of these are automorphisms (isomorphisms from a group to itself).
    \begin{enumerate}[(a)]
        \item The ``identity map'' $\iota: G\to G$ defined by $\iota(g) = g$.
        \item It has probably been a minute since you thought about the complex numbers $\C$, so here's a bit of review. A complex number $z\in \C$ looks like $z=a+bi$, where $i = \sqrt{-1}$, and we like to think of this number living at the coordinate $(a, b)$ on the complex plane; the real axis is horizontal and the imaginary axis is vertical. You can also use polar form $z=re^{i\theta}$ to specify this location, because $e^{i\theta} = \cos\theta + i \sin\theta$.
        
        The ``complex conjugate'' $\overline{z}$ is the reflection of $z$ across the real axis: so,
        define $\overline{\phantom{z}}:\C \to \C$  by $\overline{a+bi} = a - bi$. Or, in polar form $re^{i\theta}$, then $\overline{re^{i\theta}} = re^{i(-\theta)}$.
        \item (You have already proved that conjugation by a fixed element $\phi_x(g) = xgx\inv$ is an automorphism.)
        \item Under what circumstances is the ``inversion map'' $\phi(g) = g\inv$ an automorphism?
    \end{enumerate}
\end{problem}

\begin{problem} Automorphisms of cyclic groups:
    \begin{enumerate}[(a)]
        \item Find every possible automorphism of $\Z_5$. (There are four; where can the generator go?)
        \item Find every possible automorphism of $\Z_6$. (There are two; why aren't there five?)
        \item (Bonus!) Conjecture how many automorphisms there are of $\Z_n$ for a general $n$.
    \end{enumerate}
\end{problem}

\subsection*{Kernels and preimages}

\begin{problem} As promised in class:
    \begin{enumerate}[(a)]
        \item Prove that $\Ker\phi$ is a subgroup of $G$.
        \item Prove that $\Ker\phi$ is normal in $G$. (Hint: for $k\in \Ker\phi$, calculate $\phi(xkx\inv)$.)
        \item Prove that $\Ker\phi$ is trivial \Alert{iff} $\phi$ is injective.
    \end{enumerate}
\end{problem}

\begin{problem}
    Prove that for any $h\in H$, $\phi\inv(h)$ is a coset of $\Ker\phi$.
\end{problem}

\begin{problem}
    Find the kernel of each of these homomorphisms (hint: none of them are trivial):
    \begin{enumerate}[(a)]
        \item $\pi_A:A\times B\to A$ given by $\pi_A(a,b) = a$ (and how about $\pi_B$?)
        \item $\phi:\Z \to \Z_5$ where $\phi(n)$ is the remainder of $n$ mod 5. (For instance, $\phi(17) = 2$.)
        \item The very rude ``squishing map'' $\phi:G\to H$ defined by $\phi(g) = 1_H$.
        \item $\phi:(\R^2, +)\to (\R, +)$ defined by $\phi(x, y) = x+y$. (The domain is the $xy$-plane, basically; what does the kernel look like in the $xy$-plane?)
    \end{enumerate}
\end{problem}

\pagebreak
\subsection*{Lastly, a fun problem that previews something very cool}

In class we saw these two homomorphisms: 
\[\alpha:Q_8 \to V_4 = \<a, b\> \text{ defined by }\alpha(i) = a, \quad \alpha(j) = b\] 
\[\phi: Q_8 \to A_4 \text{ defined by }\phi(i) = (12)(34),\quad  \phi(j) = (13)(24)\] They are related in an important way.

\begin{problem}
    Compute $\alpha$ for all the rest of the elements of $Q_8$.
\end{problem}
\begin{problem}
    Compute $\phi$ for all the rest of the elements of $Q_8$. (This was the warmup on Wednesday.)
\end{problem}
\begin{problem}
    Here are the Cayley diagram pictures of these homomorphisms. Note that $Q_8$ is the domain of both, so both the $\alpha$ arrow to $V_4$ and the $\phi$ arrow to $A_4$ are leaving the same place $Q_8$.
    \[
    \begin{tikzpicture}[scale=1.3]
        %%
        \tikzstyle{every node}=[font=\small]
        \tikzstyle{to} = [draw, dashed, ultra thick, -stealth]
        %%
        \begin{scope}[shift={(4,4)}]
            \node (e) at (135:1) [v] {$e$};
            \node (a) at (45:1) [v] {$a$};
            \node (b) at (-135:1) [v] {$b$};
            \node (ab) at (-45:1) [v] {$ab$};
            \node at (0,0) {$V_4$};
            \path[bb] (e) to (a);
            \path[bb] (b) to (ab);
            \path[rr] (e) to (b);
            \path[rr] (a) to (ab);
        \end{scope}
        %%
        \begin{scope}[shift={(0,0)}]
            \tikzstyle{every node}=[font=\small]
            \node (1) at (135:2) [v] {$1$};
            \node (i) at (45:2) [v] {$i$};
            \node (k) at (-45:2) [v] {$k$};
            \node (j) at (-135:2) [v] {$j$};
            \node (-1) at (135:1) [v] {$-1$};
            \node (-i) at (45:1) [v] {$-i$};
            \node (-k) at (-45:1) [v] {$-k$};
            \node (-j) at (-135:1) [v] {$-j$};
            %%
            \path[b] (1) to (i);
            \path[b] (i) to (-1);
            \path[b] (-1) to (-i);
            \path[b] (-i) to (1);
            %%
            \path[b] (-j) to (-k);
            \path[b] (-k) to (j);
            \path[b] (j) to (k);
            \path[b] (k) to (-j);
            %%
            \path[r] (-k) to (-i);
            \path[r] (-i) to (k);
            \path[r] (k) to (i);
            \path[r] (i) to (-k);
            %% 
            \path[r] (1) to (j);
            \path[r] (j) to (-1);
            \path[r] (-1) to (-j);
            \path[r] (-j) to (1);
            %%
            \path[gg] (1) to (-1);
            \path[gg] (j) to (-j);
            \path[gg] (i) to (-i);
            \path[gg] (k) to (-k);
            %%
            \node at (0,0) {\normalsize $Q_8$};
        \end{scope}
        %%
        \begin{scope}[shift={(8,0)},scale=.75,bend angle=55]
            \tikzstyle{every node}=[font=\small]
            \tikzstyle{v-light} = [circle, draw, gray, fill=lightgray,inner sep=0pt, 
            minimum size=5.5mm]
            %%
            \begin{scope}[shift={(0:2)}]
                \node at (-2, 0) {\normalsize $A_4$};
                \node (e) at (-.75,.75) [v] {$e$};
                \node (x) at (.75,.75) [v] {$x$};
                \node (z) at (-.75,-.75) [v] {$z$};
                \node (y) at (.75,-.75) [v] {$y$};
            \end{scope}
            %%
            \begin{scope}[shift={(120:2)}]
                \node[v-light] (a) at (-.75,.75) [v] {$a$};
                \node[v-light] (c) at (.75,.75) [v] {$c$};
                \node[v-light] (d) at (-.75,-.75) [v] {$d$};
                \node[v-light] (b) at (.75,-.75) [v] {$b$};
                \draw [bbFaded] (a) to (c);
                \draw [bbFaded] (d) to (b);
                \draw [rrFaded] (a) to (d);
                \draw [rrFaded] (c) to (b);
            \end{scope}
            %%
            \begin{scope}[shift={(240:2)}]
                \node[v-light] (dd) at (.75,-.75) [v] {$d^2$};
                \node[v-light] (bb) at (.75,.75) [v] {$b^2$};
                \node[v-light] (aa) at (-.75,.75) [v] {$a^2$};
                \node[v-light] (cc) at (-.75,-.75) [v] {$c^2$}; 
                \draw [bbFaded] (aa) to (bb);
                \draw [bbFaded] (cc) to (dd);
                \draw [rrFaded] (aa) to (cc);
                \draw [rrFaded] (bb) to (dd);
            \end{scope}
            %%
            \draw [gFaded] (x) to [bend right=20] (b);
            \draw [gFaded] (e) to [bend right=20] (a);
            \draw [gFaded] (z) to [bend right=7] (c);
            \draw [gFaded] (y) to [bend right=5] (d);
            \draw [gFaded] (a) to [bend right=25] (aa);
            \draw [gFaded] (b) to [bend right=8] (cc);
            \draw [gFaded] (c) to [bend right=22] (dd);
            \draw [gFaded] (d) to [bend right=10] (bb);
            \draw [gFaded] (aa) to [bend right=10] (e);
            \draw [gFaded] (bb) to [bend right=25] (y);
            \draw [gFaded] (cc) to [bend right=25] (x);
            \draw [gFaded] (dd) to [bend right=10] (z);
            \draw [bb] (e) to (x);
            \draw [bb] (z) to (y);
            \draw [rr] (e) to (z);
            \draw [rr] (x) to (y);
            \node at (0:2) {\normalsize $\Image(\phi)$};
        \end{scope}
        \draw[->] (0, 2) -- (3,4) node[midway,above left]{\normalsize $\alpha$};
        \draw[->] (1.75,0) -- (6,0) node[midway,above]{\normalsize $\phi$};
        \draw[->, dashed, xPurple] (5,4) to [bend left=40] node[midway, above]{\normalsize\Palert{$\theta$}} (9.5, 1);

    \end{tikzpicture}
    \]
    It really seems like there should just be a homomorphism $\theta$ from $V_4$ to $\Image(\phi)$, as I've indicated with a dashed purple arrow. 
    \begin{itemize}
        \item What properties would you like this map to have? Injective, surjective, both?
        \item Create an explicit homomorphism $\theta:V_4\to A_4$. \\(That is, tell me what element of $A_4$ you want $\theta(a)$ and $\theta(b)$ to be, kinda like $\phi$ above.)
        \item Show that your map has the properties you want.
    \end{itemize}
\end{problem}

\end{document}

