\documentclass[12pt]{article}
% controlling the geometry of the page:
\usepackage[margin=1in, paperwidth=8.5in, paperheight=11in]{geometry} 
\usepackage{amsmath, amssymb} % useful math symbols and environments

\usepackage{lscape} %making one page landscape mode
\usepackage{tabularx}

\usepackage{multicol} % multiple columns side-by-side

\usepackage{amsthm} % Theorem-like environments
\theoremstyle{definition} % Without this line, theorem statements (and therefore problem statements etc.) show up in italic text.
\newtheorem{conjecture}{Conjecture}
\newtheorem{problem}{Problem}
\newtheorem*{remark}{Remark}
\newtheorem*{definition}{Definition}
\newtheorem*{theorem}{Theorem}

% pretty colors!
\usepackage[dvipsnames]{xcolor}
\colorlet{darkgrey}{black!70}
\colorlet{darkgreen}{green!50!black}


\usepackage{tikz} % for drawing diagrams
\usetikzlibrary{arrows,automata,positioning} 
\usetikzlibrary{decorations.markings}
\usetikzlibrary{decorations.pathreplacing}
\usetikzlibrary{patterns}
\usetikzlibrary{shapes.geometric}

\usepackage{visualalgebra}
\usepackage{graphicx} % for inserting figures with \includegraphics
\usepackage{setspace} % for controlling space between lines, paragraphs, etc.

\usepackage{fancyhdr} % for controlling headers and footers
\usepackage{newtx} % changes the default font family
\usepackage[shortlabels]{enumitem} % controllable labels for ordered and unordered lists

\usepackage{hyperref} % controls hyperlinks, both internal and external
\hypersetup{
    colorlinks=true,
    urlcolor=blue,
}

\setlength{\headheight}{14.5pt}
\newcommand\inv{^{-1}} % I am very tired of typing ^{-1}

% I don't like how LaTeX renders section headings by default
\renewcommand{\section}[1]{\begin{center} \textbf{#1} \\\end{center}}
%
\setlength{\parindent}{0in}
%\oddsidemargin=-.25in
\allowdisplaybreaks
\pagestyle{fancy}
\renewcommand{\headrulewidth}{0pt}
\lhead{MATH 312}
\rhead{Spring 2025}
%\lfoot{\copyright\ CLEAR Calculus 2010}
\cfoot{}
\renewcommand{\thefootnote}{*} 
\hyphenpenalty=10000 % LaTeX by default really likes hyphenating things

%##################################################################
\begin{document}

Here is my super-secret and extremely confidential collection of problems that I think would be good for oral exams. It would be so bad if this super-secret and extremely confidential list were to accidentally leak to students, because then students may be well-prepared to answer one or more of these basic yet revealing problems about group theory during their comprehensive conversations! Also, it would be \textit{extra} bad if a github link were to leak, because then students would even be able to see if I added more stuff to the list!!

\subsection*{Define and prove things}

\begin{problem}
    Suppose $G$ is a group, and let $H$ be a sub\textbf{set} of $G$. 
    \begin{enumerate}[(a)]
        \item Write a careful definition of what it means for $H$ to be a sub\textbf{group} of $G$.
        \item Prove the ``one-step subgroup test:'' If $xy\inv \in H$ for all $x, y\in H$, then $H$ is a subgroup of $G$.
    \end{enumerate}
\end{problem}

\begin{problem}
    Let $H$ and $K$ be subgroups of a group $G$, with $K \leq H \leq G$.
    \begin{enumerate}[(a)]
        \item Define $[G:H]$, the ``index of $H$ in $G$.''
        \item State and prove Lagrange's theorem.
        \item Use Lagrange's theorem to prove the ``tower law:'' $[G:K] = [G:H] \cdot [H:K].$
    \end{enumerate}
\end{problem}

\begin{problem}
    Let $G$ be a group, $N$ a normal subgroup, and $G/N = \{gN \mid g\in G\}$ be the set of left cosets of $N$.
    \begin{enumerate}[(a)]
        \item Write careful definitions for ``coset'' and ``normal subgroup.''
        \item Explain why $G/N$ is equivalent to $N \backslash G = \{Ng \mid g\in G\}$, the set of right cosets of $N$.
        \item Prove that the binary operation on $G/N$ defined by \[aN\cdot bN := (ab)N\] is ``well-defined;'' that is, it does not depend on choice of coset representative.
        \item Prove that $G/N$ with this binary operation is a group.
    \end{enumerate}
\end{problem}

\begin{problem}
    Suppose that $\phi:G\to H$ is a homomorphism.
    \begin{enumerate}[(a)]
        \item Write down a careful definition of a homomorphism.
        \item Prove that $\phi(1_G) = 1_H$.
        \item Use this result to prove that $\phi(g\inv) = \phi(g)\inv$.
        \item Now you can show that $\Image(\phi) := \{\phi(g) \mid g\in G\}$ is a subgroup of \underline{\quad},
        \item and also that $\Ker(\phi) := \{g \in G \mid \phi(g) = 1\}$ is a subgroup of \underline{\quad}.
    \end{enumerate}
\end{problem}

\subsection*{Calculate things}

\begin{problem}
    \href{https://www.math.clemson.edu/~macaule/classes/s24_math4120/exams/s24_math4120_midterm1.pdf}{Macauley S24 midterm 1, problem 3}, but also, draw the subgroup lattice
\end{problem}

\begin{problem}
    \href{https://www.math.clemson.edu/~macaule/classes/s24_math4120/exams/s24_math4120_midterm1.pdf}{Macauley S24 midterm 1, problem 4}
\end{problem}

\begin{problem}
    \href{https://www.math.clemson.edu/~macaule/classes/f22_math4120/exams/f22_math4120_finalexam.pdf}{Macauley F22 final, problem 7}
\end{problem}

\end{document}

