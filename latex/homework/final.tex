\documentclass[12pt]{article}
% controlling the geometry of the page:
\usepackage[margin=1in, paperwidth=8.5in, paperheight=11in]{geometry} 
\usepackage{amsmath, amssymb} % useful math symbols and environments

\usepackage{lscape} %making one page landscape mode
\usepackage{tabularx}

\usepackage{multicol} % multiple columns side-by-side
\usepackage[normalem]{ulem}   

\usepackage{amsthm} % Theorem-like environments
\theoremstyle{definition} % Without this line, theorem statements (and therefore problem statements etc.) show up in italic text.
\newtheorem{conjecture}{Conjecture}
\newtheorem{problem}{Problem}
\newtheorem*{remark}{Remark}
\newtheorem*{definition}{Definition}
\newtheorem*{theorem}{Theorem}

% pretty colors!
\usepackage[dvipsnames]{xcolor}
\colorlet{darkgrey}{black!70}
\colorlet{darkgreen}{green!50!black}


\usepackage{tikz} % for drawing diagrams
\usetikzlibrary{arrows,automata,positioning} 
\usetikzlibrary{decorations.markings}
\usetikzlibrary{decorations.pathreplacing}
\usetikzlibrary{patterns}
\usetikzlibrary{shapes.geometric}

\usepackage{visualalgebra}
\usepackage{graphicx} % for inserting figures with \includegraphics
\usepackage{setspace} % for controlling space between lines, paragraphs, etc.

\usepackage{fancyhdr} % for controlling headers and footers
\usepackage{newtx} % changes the default font family
\usepackage[shortlabels]{enumitem} % controllable labels for ordered and unordered lists

\usepackage{hyperref} % controls hyperlinks, both internal and external
\hypersetup{
    colorlinks=true,
    urlcolor=blue,
}

\setlength{\headheight}{14.5pt}
\newcommand\inv{^{-1}} % I am very tired of typing ^{-1}

% I don't like how LaTeX renders section headings by default
\renewcommand{\section}[1]{\begin{center} \textbf{#1} \\\end{center}}
%
\setlength{\parindent}{0in}
%\oddsidemargin=-.25in
\allowdisplaybreaks
\pagestyle{fancy}
\renewcommand{\headrulewidth}{0pt}
\lhead{MATH 312}
\rhead{Spring 2025}
%\lfoot{\copyright\ CLEAR Calculus 2010}
\cfoot{}
\renewcommand{\thefootnote}{*} 
\hyphenpenalty=10000 % LaTeX by default really likes hyphenating things

%##################################################################
\begin{document}

Here is my super-secret and extremely confidential collection of problems that I think would be good for oral exams. It would be so bad if this super-secret and extremely confidential list were to accidentally leak to students, because then students may be well-prepared to answer one or more of these basic yet revealing problems about group theory during their comprehensive conversations! Also, it would be \textit{extra} bad if a github link were to leak, because then students would even be able to see if I added more stuff to the list!!

\subsection*{Define and prove things}

\begin{problem}
    Suppose $G$ is a group, and let $H$ be a sub\textbf{set} of $G$. 
    \begin{enumerate}[(a)]
        \item Write a careful definition of what it means for $H$ to be a sub\textbf{group} of $G$.
        \item Prove the ``one-step subgroup test:'' If $xy\inv \in H$ for all $x, y\in H$, then $H$ is a subgroup of $G$.
    \end{enumerate}
\end{problem}

\begin{problem}
    Let $H$ and $K$ be subgroups of a group $G$, with $K \leq H \leq G$.
    \begin{enumerate}[(a)]
        \item Define $[G:H]$, the ``index of $H$ in $G$.''
        \item State and prove Lagrange's theorem.
        \item Use Lagrange's theorem to prove the ``tower law:'' $[G:K] = [G:H] \cdot [H:K].$
    \end{enumerate}
\end{problem}

\begin{problem}
    Let $G$ be a group, $N$ a normal subgroup, and $G/N = \{gN \mid g\in G\}$ be the set of left cosets of $N$.
    \begin{enumerate}[(a)]
        \item Write careful definitions for ``coset'' and ``normal subgroup.''
        \item Explain why $G/N$ is equivalent to $N \backslash G = \{Ng \mid g\in G\}$, the set of right cosets of $N$.
        \item Prove that the binary operation on $G/N$ defined by \[aN\cdot bN := (ab)N\] is ``well-defined;'' that is, it does not depend on choice of coset representative.
        \item Prove that $G/N$ with this binary operation is a group.
    \end{enumerate}
\end{problem}

\begin{problem}
    Suppose that $\phi:G\to H$ is a homomorphism.
    \begin{enumerate}[(a)]
        \item Write down a careful definition of a homomorphism.
        \item Prove that $\phi(1_G) = 1_H$.
        \item Use this result to prove that $\phi(g\inv) = \phi(g)\inv$.
        \item Now you can show that $\Image(\phi) := \{\phi(g) \mid g\in G\}$ is a subgroup of \underline{\quad},
        \item and also that $\Ker(\phi) := \{g \in G \mid \phi(g) = 1\}$ is a subgroup of \underline{\quad}.
        \item Indeed, you can prove that $\Ker(\phi)$ is a \textit{normal} subgroup.
    \end{enumerate}
\end{problem}

\subsection*{Calculate things}

\begin{problem}
    A Cayley graph of a mystery group $G$ of order $12$ is shown below; let $1$ (not $e$) denote the identity element.
  \[
  \begin{tikzpicture}[scale=1,auto]
    \begin{scope}[shift={(0,0)},xscale=.8]
      \tikzstyle{every node}=[font=\small]
      \node (1) at (.2,2.8) [v] {$1$};
      \node (r2) at (.2,1.25) [v] {$d$};
      \node (r4) at (.2,-.3) [v] {$h$};
      \node (f) at (0+2.53,2.8) [v] {$a$};
      \node (r2f) at (0+2.53,1.25) [v] {$e$};
      \node (r4f) at (0+2.53,-.3) [v] {$i$};
      \node (r3f) at (4.87,2.8) [v] {$b$};
      \node (r5f) at (4.87,1.25) [v] {$f$};
      \node (rf) at (4.87,-.3) [v] {$j$};
      \node (r3) at (7.2,2.8) [v] {$c$};
      \node (r5) at (7.2,1.25) [v] {$g$};
      \node (r) at (7.2,-.3) [v] {$k$};
      \draw [bb] (1) to (f); \draw [bb] (r3f) to (r3);
      \draw [gg] (1) to [bend left=15] (r3); \draw [gg] (f) to (r3f);
      \draw [bb] (r2) to (r2f); \draw [bb] (r5f) to (r5);
      \draw [gg] (r2) to [bend left=15] (r5); \draw [gg] (r2f) to (r5f);
      \draw [bb] (r4) to (r4f); \draw [bb] (rf) to (r);
      \draw [gg] (r4) to [bend right=15] (r); \draw [gg] (r4f) to (rf);
      \draw [r] (1) to (r2);
      \draw [r] (r2) to (r4);
      \draw [r] (r4) to [bend left=30] (1);
      \draw [r] (r4f) to (r2f);
      \draw [r] (r2f) to (f);
      \draw [r] (f) to [bend right=30] (r4f);
      \draw [r] (r3f) to (r5f);
      \draw [r] (r5f) to (rf);
      \draw [r] (rf) to [bend right=30] (r3f);
      \draw [r] (r) to (r5);
      \draw [r] (r5) to (r3);
      \draw [r] (r3) to [bend left=30] (r);
    \end{scope}
    %%
    \begin{scope}[shift={(8,0)},xscale=.8]
      \node (1) at (.2,2.8) [v] {};
      \node (r2) at (.2,1.25) [v] {};
      \node (r4) at (.2,-.3) [v] {};
      \node (f) at (0+2.53,2.8) [v] {};
      \node (r2f) at (0+2.53,1.25) [v] {};
      \node (r4f) at (0+2.53,-.3) [v] {};
      \node (r3f) at (4.87,2.8) [v] {};
      \node (r5f) at (4.87,1.25) [v] {};
      \node (rf) at (4.87,-.3) [v] {};
      \node (r3) at (7.2,2.8) [v] {};
      \node (r5) at (7.2,1.25) [v] {};
      \node (r) at (7.2,-.3) [v] {};
      \draw [bb] (1) to (f); \draw [bb] (r3f) to (r3);
      \draw [gg] (1) to [bend left=15] (r3); \draw [gg] (f) to (r3f);
      \draw [bb] (r2) to (r2f); \draw [bb] (r5f) to (r5);
      \draw [gg] (r2) to [bend left=15] (r5); \draw [gg] (r2f) to (r5f);
      \draw [bb] (r4) to (r4f); \draw [bb] (rf) to (r);
      \draw [gg] (r4) to [bend right=15] (r); \draw [gg] (r4f) to (rf);
      \draw [r] (1) to (r2);
      \draw [r] (r2) to (r4);
      \draw [r] (r4) to [bend left=30] (1);
      \draw [r] (r4f) to (r2f);
      \draw [r] (r2f) to (f);
      \draw [r] (f) to [bend right=30] (r4f);
      \draw [r] (r3f) to (r5f);
      \draw [r] (r5f) to (rf);
      \draw [r] (rf) to [bend right=30] (r3f);
      \draw [r] (r) to (r5);
      \draw [r] (r5) to (r3);
      \draw [r] (r3) to [bend left=30] (r);
    \end{scope}
  \end{tikzpicture}
  \]
  
  \begin{enumerate}[(a)]
    \item Write the order of each element in the nodes of the blank
      Cayley graph on the right.

    \item The subgroup $H\!=\!\<a,b\>\!\cong$ \uline{\hfill}, and
      $K\!=\!\<d\>\!\cong$ \uline{\hfill}.

    \item The subgroup $\<a,d\>$ has order \uline{\hfill}, and is
      isomorphic to \uline{\hfill}.

    \item The subgroup $\<b,d\>$ has order \uline{\hfill}, and is
      isomorphic to \uline{\hfill}.
      
    \item The center of this group is $Z(G)=$ \uline{\hfill}. [Write
      it in terms of generator(s).]
    
  \item Find all left cosets of $H=\<a,b\>$. Then find all right
    cosets. Write your answers as subsets of $G$, or describe them
    in words (e.g., ``the rows'' or ``the columns''). 
  \item Find all left cosets of $K=\<d\>$. Then find all right
    cosets. Write your answers as subsets of $G$, or describe them
    in words. 
  \item The normalizers are $N_G(H)=\big\<$\hfill$\big\>$ and
    $N_G(K)=\big\<$\hfill$\big\>$. {\hfill} \medskip
    
  \item Find all conjugate subgroups to $H$ and to $K$. Write each
    subgroup only once.
    \item Is $H$ normal? Is $K$ normal? 
    \item Draw the subgroup lattice of $G$.
\end{enumerate}
\end{problem}

\newpage
\begin{problem}
    \colorlet{midgrey}{black!50}
    Answer questions about the following group, whose subgroup
  lattice is shown below.
  \[
  \begin{tikzpicture}[shorten >= -3pt, shorten <= -3pt, scale=1.3,yscale=1.4]
    \begin{scope}[shift={(0,0)}]
    \tikzstyle{every node}=[font=\small]
      \node[anchor=east] at (-4.5,4) {\normalsize \textbf{Order} $=24$};
      %%
      \node[anchor=east] at (5,4) {\normalsize \textbf{Index}};
      %%
      \node (G) at (0,4) {$G$};
      \node (D6) at (-2.2,3.2) {$D_6$};
      \node (C12) at (0,3.2) {$C_{12}$};
      \node (Dic6) at (2.7,3.2) {$\Dic_6$};
      \node (C4xC2) at (2.2,2.7) {${}_{\color{midgrey} 3}C_4\!\times\!C_2$};
      \node (D3-1) at (-3.2,2.2) {$D_3$};
      \node (D3-2) at (-2.2,2.2) {$D_3$};
      \node (C6) at (-1.2,2.2) {$C_6$};
      \node (V4) at (1.2,1.8) {${}_{\color{midgrey} 3}V_4$};
      \node (C4-1) at (2.2,1.8) {$C_4$};
      \node (C4-2) at (3.2,1.8) {${}_{\color{midgrey} 3}C_4$};
      \node (C3) at (-2.2,1.3) {$C_3$};
      \node (C2-1) at (-2.7,.8) {${}_{\color{midgrey} 3}C_2$};
      \node (C2-3) at (0,.8) {${}_{\color{midgrey} 3}C_2$};
      \node (C2-2) at (2.2,.8) {$C_2$};
      \node (1) at (0,0) {$C_1$};
      %%
      %%
      \draw[faded] (G) to (D6); \draw[faded] (G) to (C12); 
      \draw[faded] (G) to (Dic6); \draw[faded] (G) to (C4xC2); 
      %%
      \draw[faded] (D6) to (D3-1); \draw[faded] (D6) to (D3-2); 
      \draw[faded] (D6) to (C6); \draw[faded] (D6) to (V4);
      %%
      \draw[faded] (C12) to (C6); \draw[faded] (C12) to (C4-1);
      %%
      \draw[faded] (Dic6) to (C6); \draw[faded] (Dic6) to (C4-2);
      %%
      \draw[faded] (C4xC2) to (V4); \draw[faded] (C4xC2) to (C4-1);
      \draw[faded] (C4xC2) to (C4-2);
      %%
      \draw[faded] (D3-1) to (C2-1); \draw[faded] (D3-1) to (C3);
      \draw[faded] (D3-2) to (C2-3); \draw[faded] (D3-2) to (C3);
      %%
      \draw[faded] (C6) to (C3); \draw[faded] (C6) to (C2-2); 
      %%
      \draw[faded] (C4-1) to (C2-2); \draw[faded] (C4-2) to (C2-2);
      %%
      \draw[faded] (V4) to (C2-1); \draw[faded] (V4) to (C2-2); \draw[faded] (V4) to (C2-3);
      %%
      \draw[faded] (C2-1) to (1); \draw[faded] (C2-2) to (1); 
      \draw[faded] (C2-3) to (1); \draw[faded] (C3) to (1);
      %%
    \end{scope}
    \end{tikzpicture}
  \]
  \begin{enumerate}[(a)]
    \item Determine the order and the index of each row of subgroups in the lattice.
  \item $G$ has \uline{\hfill} subgroup, which fall into \uline{\hfill}
    conjugacy classes. \smallskip
  \item $G$ has exactly \uline{\hfill} normal subgroups. \smallskip

  \item $G$ has \uline{\hfill} subgroup(s) of order $2$ and
    \uline{\hfill} element(s) of order $2$. \smallskip
    
  \item $G$ has \uline{\hfill} subgroup(s) of order $3$ and
    \uline{\hfill} element(s) of order $3$. \smallskip

  \item $G$ has \uline{\hfill} subgroup(s) of order $4$, of which
    \uline{\hfill} are cyclic. \smallskip
    
  \item Find a normal subgroup $N\normaleq G$ such that $G/N \cong V_4$.
    
  \item Each non-normal order-$2$ subgroup has a normalizer isomorphic
    to \uline{\hfill}.
    
  \item Each $D_3$ subgroup has a normalizer isomorphic to
    \uline{\hfill}.
    
  \item This group has a quotient $G/C_4$ isomorphic to \uline{\hspace{10mm}}.
    [\emph{Hint}: Determine the order, then count the index-$2$
      subgroups.]

  \item This group has a quotient $G/C_2$ isomorphic to \uline{\hspace{10mm}}.
    [\emph{Hint}: Same as above.]  
    
  \item The quotient $G/C_3$ is isomorphic to \uline{\hspace{10mm}}.
    [\emph{Hint}: Determine the order. Which lattice do you see?]

  \item There are $n_2=\uline{\hspace{10mm}}$ Sylow $2$-subgroups,
    which are isomorphic to $\uline{\hspace{15mm}}$.


  \item There are $n_3=\uline{\hspace{10mm}}$ Sylow $3$-subgroups,
    which are isomorphic to $\uline{\hspace{15mm}}$.

  \item The largest order of an element in $G$ is
    \uline{\hspace{10mm}}, and there are \uline{\hspace{10mm}}
    element(s) of that order.
  \end{enumerate}
\end{problem}

\newpage
\begin{problem}
    The subgroup diagram of a group $G$ is shown below.
  %%
  %% Subgroup lattice of A4 x C3
  \[
  \begin{tikzpicture}[shorten >= -2pt, shorten <= -2pt,yscale=.65, scale=1.75]
    %%
    \newcommand\Hg{5} % height of 36
    \newcommand\Hf{3.5} % height of 12 
    \newcommand\He{3} % height of 9
    \newcommand\Hd{2.5} % height of 6 
    \newcommand\Hc{2} % height of 4 
    \newcommand\Hb{1.5} % height of 3
    \newcommand\Ha{1} % height of 2
    \tikzstyle{every node}=[font=\small]
    %%
    \begin{scope}[shift={(-2.75,0)}]
      \tikzstyle{every node}=[font=\small]
      \node[anchor=east] at (0,\Hg) {\textbf{Index} $=1$};
      \node[anchor=east] at (0,\Hf) {$3$};
    \end{scope}
    %%
    \begin{scope}[shift={(3,0)}]
      \tikzstyle{every node}=[font=\small]
      \node[anchor=east] at (0,\Hg) {\textbf{Order} $=36$};
      \node[anchor=east] at (0,\Hf) {$12$};
   \end{scope}
   %%
    \begin{scope}[shift={(0,0)}]
      \node (G) at (0,\Hg) {$G$};
      \node (A4-1) at (-.5,\Hf) {$A_4$};
      \node (A4-2) at (0,\Hf) {$A_4$};
      \node (A4-3) at (.5,\Hf) {$A_4$};
      \node (C6xC2) at (1.2,\Hf) {$C_6\!\times\!C_2$};
      \node (C3xC3) at (-1.6,\He)  {${}_{\color{midgray}4}C_3^2$};
      \node (C6) at (1.8,\Hd) {${}_{\color{midgray}3}C_6$};
      \node (V4) at (0,\Hc) {$V_4$}; 
      \node (C3-1) at (-2.1,\Hb) {${}_{\color{midgray}4}C_3$};
      \node (C3-2) at (-1.6,\Hb) {${}_{\color{midgray}4}C_3$};
      \node (C3-3) at (-1.1,\Hb) {${}_{\color{midgray}4}C_3$};
      \node (C3-4) at (-.5,\Hb) {$C_3$};
      \node (C2) at (.6,\Ha) {${}_{\color{midgray}3}C_2$};
      \node (1) at (0,0) {$C_1$};      
      %%
      \draw[f] (G) to (A4-1); \draw[f] (G) to (A4-2); \draw[f] (G) to (A4-3);
      \draw[f] (G) to (C3xC3); \draw[f] (G) to (C6xC2);
      \draw[f] (C6xC2) to (C6); 
      \draw[f] (C6xC2) to (V4);
      \draw[f] (A4-1) to (V4); \draw[f] (A4-2) to (V4); \draw[f] (A4-3) to (V4);
      \draw[f] (A4-1) to (C3-1); \draw[f] (A4-2) to (C3-2); \draw[f] (A4-3) to (C3-3);
      \draw[f] (C3xC3) to (C3-1); \draw[f] (C3xC3) to (C3-2); \draw[f] (C3xC3) to (C3-3);
      \draw[f] (C3xC3) to (C3-4);
      \draw[f] (C6) to (C3-4); \draw[f] (C6) to (C2); \draw[f] (V4) to (C2);
      \draw[f] (C3-1) to (1); \draw[f] (C3-2) to (1); \draw[f] (C3-3) to (1);
      \draw[f] (C3-4) to (1); \draw[f] (C2) to (1);
    \end{scope}
    %%%
    \end{tikzpicture}
  \]
  \begin{enumerate}[(a)]
    \item Determine the order and the index of each row of subgroups in the lattice.
    \item The group $G$ has \uline{\hfill} subgroups, which fall into \uline{\hfill} conjugacy classes.

    \item The quotient of $G$ by its unique normal subgroup $N$ of order $3$ has order \uline{\hfill}, and $G/N$ is isomorphic to the familiar group \uline{\hfill}.


   \item There are $n_2=$\uline{\hfill} Sylow $2$-subgroup(s), isomorphic to \uline{\hfill}, 
   
   and $n_3=$\uline{\hfill} Sylow $3$-subgroup(s), isomorphic to \uline{\hfill}.
   
    \item Find all distinct ways that $G$ can be written as a direct or semidirect product of two of its proper subgroups. 
    
    \item Find the center $Z(G)$, and justify your answer. Though this cannot always be done by inspection, it can in this case, using a result from the previous part.

   \item Let $G$ act on its subgroups by conjugation. This action has \uline{\hfill} orbit(s) 
   
   and \uline{\hfill} fixed point(s). \hfill\hfill \,

   \item Let $G$ act on itself by multiplication. This action has \uline{\hfill} orbit(s) 
   
   and \uline{\hfill} fixed point(s). \hfill\hfill \,   

   \item Let $H$ be a subgroup of order $9$, and let $G$ act on the right cosets of $H$ 
   by right multiplication. This action has \uline{\hfill} orbit(s) and \uline{\hfill} fixed point(s).

   \item Still assuming that $|H|=9$, let $G$ act on the \emph{left} cosets of $H$ 
   by \emph{left} multiplication. This action has \uline{\hfill} orbit(s) and \uline{\hfill} fixed point(s).

  \item Let $g\in G$ be an element of order $2$. Then $g$ commutes with exactly 
  \uline{\hfill} element(s), and the centralizer of $\<g\>$ is isomorphic to \uline{\hfill}. The centralizer of $g$ is 
  
  (bigger than)(smaller than)(equal to) [$\longleftarrow$ \emph{circle one}] its normalizer.
 \end{enumerate}
\end{problem}

\newpage
\begin{problem}
    \colorlet{lorange}{orange!50!white}
    \definecolor{lpurple}{rgb}{0.7 0.45 0.9}
    Consider the following set of ``binary rectangles'':
  \[
    \begin{tikzpicture}[scale=.9,yscale=1.618]
    \node at (-1,.5) {$S=\Bigg\{$};
    \node at (13.5,.5) {$\Bigg\}$};
    \node at (1.5,.25) {,};
    \node at (3.5,.25) {,};
    \node at (5.5,.25) {,};
    \node at (7.5,.25) {,};
    \node at (9.5,.25) {,};
    \node at (11.5,.25) {,};
    %%
    \begin{scope}[shift={(0,0)}]
      \path[fill=lorange] (.5,.5)--(0,1)--(1,1);
      \path[fill=lorange] (.5,.5)--(1,1)--(1,0);
      \path[fill=lorange] (.5,.5)--(0,0)--(1,0);
      \path[fill=lorange] (.5,.5)--(0,0)--(0,1);
      \draw (.5,.78) node{$0$}; 
      \draw (.18,.5) node{$0$}; \draw (.82,.5) node{$0$}; 
      \draw (.5,.22) node{$0$};
      \draw (0,0) rectangle (1,1);
    \end{scope}
    %%
    \begin{scope}[shift={(2,0)}]
      \path[fill=lpurple] (.5,.5)--(1,1)--(1,0);
      \path[fill=lorange] (.5,.5)--(0,1)--(1,1);
      \path[fill=lpurple] (.5,.5)--(0,0)--(0,1);
      \path[fill=lorange] (.5,.5)--(0,0)--(1,0);
      \draw (.5,.78) node{$0$}; 
      \draw (.18,.5) node{$1$}; \draw (.82,.5) node{$1$}; 
      \draw (.5,.22) node{$0$};
      \draw (0,0) rectangle (1,1);
    \end{scope}
    %%
    \begin{scope}[shift={(4,0)}]
      \path[fill=lorange] (.5,.5)--(1,1)--(1,0);
      \path[fill=lpurple] (.5,.5)--(0,1)--(1,1);
      \path[fill=lorange] (.5,.5)--(0,0)--(0,1);
      \path[fill=lpurple] (.5,.5)--(0,0)--(1,0);
      \draw (.5,.78) node{$1$}; 
      \draw (.18,.5) node{$0$}; \draw (.82,.5) node{$0$}; 
      \draw (.5,.22) node{$1$};
      \draw (0,0) rectangle (1,1);
    \end{scope}
    %%
    \begin{scope}[shift={(6,0)}]
      \path[fill=lpurple] (.5,.5)--(1,1)--(1,0);
      \path[fill=lpurple] (.5,.5)--(0,1)--(1,1);
      \path[fill=lorange] (.5,.5)--(0,0)--(0,1);
      \path[fill=lorange] (.5,.5)--(0,0)--(1,0);
      \draw (.5,.78) node{$1$}; 
      \draw (.18,.5) node{$0$}; \draw (.82,.5) node{$1$}; 
      \draw (.5,.22) node{$0$};
      \draw (0,0) rectangle (1,1);
    \end{scope}
    %%
    \begin{scope}[shift={(8,0)}]
      \path[fill=lorange] (.5,.5)--(1,1)--(1,0);
      \path[fill=lpurple] (.5,.5)--(0,1)--(1,1);
      \path[fill=lpurple] (.5,.5)--(0,0)--(0,1);
      \path[fill=lorange] (.5,.5)--(0,0)--(1,0);
      \draw (.5,.78) node{$1$}; 
      \draw (.18,.5) node{$1$}; \draw (.82,.5) node{$0$}; 
      \draw (.5,.22) node{$0$};  
      \draw (0,0) rectangle (1,1);
    \end{scope}
    %%
    \begin{scope}[shift={(10,0)}]
      \path[fill=lorange] (.5,.5)--(1,1)--(1,0);
      \path[fill=lorange] (.5,.5)--(0,1)--(1,1);
      \path[fill=lpurple] (.5,.5)--(0,0)--(0,1);
      \path[fill=lpurple] (.5,.5)--(0,0)--(1,0);
      \draw (.5,.78) node{$0$}; 
      \draw (.18,.5) node{$1$}; \draw (.82,.5) node{$0$}; 
      \draw (.5,.22) node{$1$}; 
      \draw (0,0) rectangle (1,1);
    \end{scope}
    %%
    \begin{scope}[shift={(12,0)}]
      \path[fill=lpurple] (.5,.5)--(1,1)--(1,0);
      \path[fill=lorange] (.5,.5)--(0,1)--(1,1);
      \path[fill=lorange] (.5,.5)--(0,0)--(0,1);
      \path[fill=lpurple] (.5,.5)--(0,0)--(1,0);
      \draw (.5,.78) node{$0$}; 
      \draw (.18,.5) node{$0$}; \draw (.82,.5) node{$1$}; 
      \draw (.5,.22) node{$1$};
      \draw (0,0) rectangle (1,1);
    \end{scope}
  \end{tikzpicture}
    \]
    The Klein-$4$ group $V=\{1,v,h,vh\}$ acts on $S$
    via $\phi\colon V\to\Perm(S)$, where \vspace{-1mm}
  \begin{align*}
    \phi(v)&=\text{flips each tile vertically} \vspace{3mm} \\
    \phi(h) &= \text{flips each tile horizontally}\vspace{3mm} \\
    \phi(vh)&=\text{rotates each tile by $180^\circ$}
  \end{align*}

  \begin{enumerate}[(a)]
  \item Pick a minimal generating set and then draw the \emph{action
    graph}. (Feel free to label the rectangles above A,B,C,D,E,F,G to
    save time.)
    
    \vfill
    
    
  \item Find the following:
    \begin{multicols}{2}
    \begin{itemize}
    \item $\stab\bigg(
        \begin{tikzpicture}[yscale=1.618,scale=.6,baseline=2ex]
          \tikzstyle{every node}=[font=\scriptsize]
          \path[fill=lpurple] (.5,.5)--(1,1)--(1,0);
          \path[fill=lorange] (.5,.5)--(0,1)--(1,1);
          \path[fill=lpurple] (.5,.5)--(0,0)--(0,1);
          \path[fill=lorange] (.5,.5)--(0,0)--(1,0);
          \draw (.5,.78) node{$0$}; 
          \draw (.18,.5) node{$1$}; \draw (.82,.5) node{$1$}; 
          \draw (.5,.22) node{$0$};
          \draw (0,0) rectangle (1,1);
        \end{tikzpicture}\bigg)=$ \\ \\
        
      \item $\stab\bigg(
        \begin{tikzpicture}[yscale=1.618,scale=.6,baseline=2ex]
          \tikzstyle{every node}=[font=\scriptsize]
          \path[fill=lorange] (.5,.5)--(0,1)--(1,1);
          \path[fill=lorange] (.5,.5)--(1,1)--(1,0);
          \path[fill=lorange] (.5,.5)--(0,0)--(1,0);
          \path[fill=lorange] (.5,.5)--(0,0)--(0,1);
          \draw (.5,.78) node{$0$}; 
          \draw (.18,.5) node{$0$}; \draw (.82,.5) node{$0$}; 
          \draw (.5,.22) node{$0$};
          \draw (0,0) rectangle (1,1);
      \end{tikzpicture}\bigg)=$ \\ \\

      \item $\stab\bigg(
        \begin{tikzpicture}[yscale=1.618,scale=.6,baseline=2ex]
          \tikzstyle{every node}=[font=\scriptsize]
                \path[fill=lpurple] (.5,.5)--(1,1)--(1,0);
      \path[fill=lpurple] (.5,.5)--(0,1)--(1,1);
      \path[fill=lorange] (.5,.5)--(0,0)--(0,1);
      \path[fill=lorange] (.5,.5)--(0,0)--(1,0);
      \draw (.5,.78) node{$1$}; 
      \draw (.18,.5) node{$0$}; \draw (.82,.5) node{$1$}; 
      \draw (.5,.22) node{$0$};
      \draw (0,0) rectangle (1,1);
        \end{tikzpicture}\bigg)=$ \\ \\
        %%
  \item $\stab\bigg(
    \begin{tikzpicture}[yscale=1.618,scale=.6,baseline=2ex]
      \tikzstyle{every node}=[font=\scriptsize]        
      \path[fill=lorange] (.5,.5)--(1,1)--(1,0);
      \path[fill=lpurple] (.5,.5)--(0,1)--(1,1);
      \path[fill=lpurple] (.5,.5)--(0,0)--(0,1);
      \path[fill=lorange] (.5,.5)--(0,0)--(1,0);
      \draw (.5,.78) node{$1$}; 
      \draw (.18,.5) node{$1$}; \draw (.82,.5) node{$0$}; 
      \draw (.5,.22) node{$0$};  
      \draw (0,0) rectangle (1,1);
    \end{tikzpicture}\bigg)=$ \\ \\
    %%
    \end{itemize}
    \end{multicols}
    
    %\vspace{0mm}
    
    \begin{multicols}{2}
      \begin{itemize}
      \item $\fix(1)=$ \\ \\
        
      \item $\fix(v)=$ \\ \\
        
      \item $\fix(h)=$ \\ \\

      \item $\fix(vh)=$ \\ \\
      \end{itemize}
    \end{multicols}

    \begin{itemize}
        
      \item This action has \uline{\hspace{.7in}} orbits, which by the
        orbit-counting theorem, is also \medskip

        equal to the average
        \uline{\hfill}.  \\ \smallskip
        
      %\item Average size of $\fix(g)$, where $g\in V_4=$ \\ \smallskip
    \end{itemize}


    \begin{multicols}{2}
      \begin{itemize}
        
      \item $\Fix(\phi)=$ 

      \item $\Ker(\phi)=$ \\

      \end{itemize}
    \end{multicols}
  \end{enumerate}
\end{problem}

\end{document}

