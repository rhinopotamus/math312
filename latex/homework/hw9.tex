\documentclass[12pt]{article}
% controlling the geometry of the page:
\usepackage[margin=1in, paperwidth=8.5in, paperheight=11in]{geometry} 
\usepackage{amsmath, amssymb} % useful math symbols and environments

\usepackage{multicol} % multiple columns side-by-side

\usepackage{amsthm} % Theorem-like environments
\theoremstyle{definition} % Without this line, theorem statements (and therefore problem statements etc.) show up in italic text.
\newtheorem{conjecture}{Conjecture}
\newtheorem{problem}{Problem}
\newtheorem*{remark}{Remark}
\newtheorem*{definition}{Definition}

% pretty colors!
\usepackage[dvipsnames]{xcolor}
\colorlet{darkgrey}{black!70}
\colorlet{darkgreen}{green!50!black}


\usepackage{tikz} % for drawing diagrams
\usetikzlibrary{arrows,automata,positioning} 
\usetikzlibrary{decorations.markings}
\usetikzlibrary{decorations.pathreplacing}
\usetikzlibrary{patterns}
\usetikzlibrary{shapes.geometric}

\usepackage{visualalgebra}
\usepackage{graphicx} % for inserting figures with \includegraphics
\usepackage{setspace} % for controlling space between lines, paragraphs, etc.

\usepackage{fancyhdr} % for controlling headers and footers
\usepackage{newtx} % changes the default font family
\usepackage[shortlabels]{enumitem} % controllable labels for ordered and unordered lists

\usepackage{hyperref} % controls hyperlinks, both internal and external
\hypersetup{
    colorlinks=true,
    urlcolor=blue,
}

\setlength{\headheight}{14.5pt}
\newcommand\inv{^{-1}} % I am very tired of typing ^{-1}

% I don't like how LaTeX renders section headings by default
\renewcommand{\section}[1]{\begin{center} \textbf{#1} \\\end{center}}
%
\setlength{\parindent}{0in}
%\oddsidemargin=-.25in
\allowdisplaybreaks
\pagestyle{fancy}
\renewcommand{\headrulewidth}{0pt}
\lhead{MATH 312}
\rhead{Spring 2025}
%\lfoot{\copyright\ CLEAR Calculus 2010}
\cfoot{}
\renewcommand{\thefootnote}{*} 
\hyphenpenalty=10000 % LaTeX by default really likes hyphenating things

% all the stuff above this line is called the preamble...
%##################################################################
\begin{document} % this is always the first line of what's actually produced
\section{Homework \#9 (due Mar 30)} % notice that if you want the character # to appear, you have to "escape" it with a backslash

\subsection*{Finish HW\#8}

I don't know if it was spring break brain or what, but only one person has so far turned in HW\#8. So, do that. I've changed everyone's due date for HW\#8 to be Mar 30.

%%======================================================================
\subsection*{Automorphism groups}

In class we figured out \(\Aut(\Z_n, +)\) for $n = 2 \ldots 12$. (They end up being isomorphic to $(U_n, \cdot)$, the group of integers relatively prime to $n$ with the operation multiplication mod $n$.) Let's explore $\Aut(G)$ for some non-cyclic groups $G$.

\begin{problem}
    Find $\Aut(V_4)$. Hints:
    \begin{itemize}
        \item Say $V_4 = \<a, b\>$. Any automorphism of $V_4$ is determined by what it does to those two generators. How many choices do you have for where $a$ goes? How many choices do you then have for where $b$ goes? So how many possible automorphisms do you have in total?
        \item Do your automorphisms commute, or does the order of composition matter?
    \end{itemize}
\end{problem}

\begin{problem}
    Find $\Aut(D_3)$. Hints:
    \begin{itemize}
        \item Again, automorphisms are determined by what they do to the generators.
        \item Also, automorphisms preserve order. Where can $r$ go? Where can $f$ go?
        \item Do your automorphisms commute?
    \end{itemize}
\end{problem}
%%======================================================================
\subsection*{Semidirect products}

\begin{problem}
    The semidirect product \(\Alert{\Z_3} \rtimes \Balert{\Z_2}\) is a group of order 6 ($=2\times 3$). Draw it:
    \begin{itemize}
        \item Inflate $\Balert{\Z_2}$ and insert a copy of $\Alert{\Z_3}$ in each inflated node.
        \item Rewire the copy of $\Alert{\Z_3}$ in the $\Balert{1}$-node. (There is only one interesting way to do this.)
        \item Pop the nodes of $\Balert{\Z_2}$ and reconnect the $\Balert{\Z_2}$ arrows ``straight up'' -- $\Alert{0}$ to $\Alert{0}$, $\Alert{1}$ to $\Alert{1}$, and $\Alert{2}$ to $\Alert{2}$.
    \end{itemize}
\end{problem}

\begin{problem}
    I will tell you for free that \(\Alert{\Z_3} \rtimes \Balert{\Z_2}\) is isomorphic to one of the groups of order 6 that we already know. Construct an \textbf{explicit} isomorphism $\phi:\Alert{\Z_3} \rtimes \Balert{\Z_2} \to G$ for the correct group $G$:
    \begin{enumerate}[(a)]
        \item write a recipe for sending elements of \(\Alert{\Z_3} \rtimes \Balert{\Z_2}\) to $G$,
        \item show that your recipe is a homomorphism,
        \item show that your homomorphism is injective and surjective.
    \end{enumerate}
    (Hint: This will be easier if you figure out a minimal set of generators for both \(\Alert{\Z_3} \rtimes \Balert{\Z_2}\) and $G$, and then figure out how to map the generators over correctly.)
\end{problem}

\pagebreak

\begin{problem}
    In class we constructed three different versions of $\Alert{C_5}\rtimes \Balert{C_4}$; two are drawn below. $\Alert{C_5}$ and $\Balert{C_4}$ were ``compatible'' because there were automorphisms of $\Alert{C_5}$ that had order dividing $|\Balert{C_4}|$. Choose two other cyclic groups that are ``compatible'' and construct their semidirect product.
\end{problem}

(Look at the code here and conclude that you should \textbf{not} try to draw these in tikz, lol.)
\[
\begin{tikzpicture}[scale=1]
    \tikzstyle{v} = [circle, draw, fill=lightgray,inner sep=0pt, 
      minimum size=4.25mm]
    \tikzstyle{v-y} = [circle, draw, fill=vYellow,inner sep=0pt, 
      minimum size=4.25mm]
    \tikzstyle{B2} = [draw, eBlue, -stealth']       % Blue --->
      %%%%%%%%%%%%
      \begin{scope}[shift={(0,0)}]
        \tikzstyle{every node}=[font=\footnotesize]
        \node at (5,3.5) {The \Palert{purple} labels are the automorphism at that corner:};
        \node at (5, 3) {replace \Alert{1 red arrow} in original $\Alert{C_5}$ by \Palert{this many arrows},};
        \node at (5, 2.5) {calculated \Alert{mod 5}};
        \node at (5, 2) {Note I've collapsed $(a^n, b^m)$ into $a^n b^m$ for space reasons};
        %%
        \begin{scope}[shift={(-.25,6.5)}]
          \node (1) at (0:1) [v-y] {$1$};
          \node (r) at (72:1) [v] {$a$};
          \node (r2) at (144:1) [v] {$a^2$};
          \node (r3) at (216:1) [v] {$a^3$};
          \node (r4) at (288:1) [v] {$a^4$};
          \node at (180: 2) {\normalsize\Palert{$2^0 = 1$}};
        \end{scope}
        %%
        \begin{scope}[shift={(-.25,0)}]
          \node (s) at (0:1) [v] {$b$};
          \node (rs) at (72:1) [v] {$ab$};
          \node (r2s) at (144:1) [v] {$a^2\!b$};
          \node (r3s) at (216:1) [v] {$a^3\!b$};
          \node (r4s) at (288:1) [v] {$a^4\!b$};        
          \node at (180:2) {\normalsize\Palert{$2^1 = 2$}};
        \end{scope}
        %%
        \begin{scope}[shift={(10.25,0)}]
          \node (s2) at (0:1) [v] {$b^2$};
          \node (rs2) at (72:1) [v] {$ab^2$};
          \node (r2s2) at (144:1) [v] {$a^2\!b^{\!2}$};
          \node (r3s2) at (216:1) [v] {$a^3\!b^{\!2}$};
          \node (r4s2) at (288:1) [v] {$a^4\!b^{\!2}$};
          \node at (0:2) {\normalsize\Palert{$2^2 = 4$}};
        \end{scope}
        %%
        \begin{scope}[shift={(10.25,6.5)}]
          \node (s3) at (0:1) [v] {$b^3$};
          \node (rs3) at (72:1) [v] {$ab^{\!3}$};
          \node (r2s3) at (144:1) [v] {$a^2\!b^{\!3}$};
          \node (r3s3) at (216:1) [v] {$a^3\!b^{\!3}$};
          \node (r4s3) at (288:1) [v] {$a^4\!b^{\!3}$};
          \node at (0:2) {\normalsize\Palert{$2^3 \equiv 3$}};
        \end{scope}
        %%
        \draw [B2] (1) to (s);
        \draw [B2] (r) to[bend left=13] (rs);
        \draw [B2] (r2) to[bend right=13] (r2s); 
        \draw [B2] (r3) to[bend left=15] (r3s);
        \draw [B2] (r4) to[bend right=13] (r4s);
        %%
        \draw [B2] (s) to (s2);
        \draw [B2] (rs) to (rs2);
        \draw [B2] (r2s) to (r2s2);
        \draw [B2] (r3s) to (r3s2);
        \draw [B2] (r4s) to (r4s2);
        %%
        \draw [B2] (s2) to (s3);
        \draw [B2] (rs2) to[bend right=13] (rs3);
        \draw [B2] (r2s2) to[bend left=13] (r2s3);
        \draw [B2] (r3s2) to[bend right=15] (r3s3);
        \draw [B2] (r4s2) to[bend left=13] (r4s3);
        %%
        \draw [B2] (s3) to (1); 
        \draw [B2] (rs3) to (r);
        \draw [B2] (r2s3) to (r2);
        \draw [B2] (r3s3) to (r3);
        \draw [B2] (r4s3) to (r4);
        %%
        \draw [r] (1) to (r); \draw [r] (r) to (r2); \draw [r] (r2) to (r3);
        \draw [r] (r3) to (r4); \draw [r] (r4) to (1);
        %%
        \draw [r] (s) to (r2s); \draw [r] (r2s) to (r4s); \draw [r] (r4s) to (rs);
        \draw [r] (rs) to (r3s); \draw [r] (r3s) to (s);
        %%
        \draw [r] (s2) to (r4s2); \draw [r] (r4s2) to (r3s2);
        \draw [r] (r3s2) to (r2s2); \draw [r] (r2s2) to (rs2);
        \draw [r] (rs2) to (s2);
        %%
        \draw [r] (s3) to (r3s3); \draw [r] (r3s3) to (rs3);
        \draw [r] (rs3) to (r4s3); \draw [r] (r4s3) to (r2s3);
        \draw [r] (r2s3) to (s3);
      \end{scope}
    \end{tikzpicture}
\]

\[
\begin{tikzpicture}[scale=1]
    \tikzstyle{v} = [circle, draw, fill=lightgray,inner sep=0pt, 
      minimum size=4.25mm]
    \tikzstyle{v-y} = [circle, draw, fill=vYellow,inner sep=0pt, 
      minimum size=4.25mm]
    \tikzstyle{B2} = [draw, eBlue, -stealth']       % Blue --->
      %%%%%%%%%%%%
      \begin{scope}[shift={(0,0)}]
        \tikzstyle{every node}=[font=\footnotesize]
        \node at (5,3.5) {The \Palert{purple} labels are the automorphism at that corner:};
        \node at (5, 3) {replace \Alert{1 red arrow} in original $\Alert{C_5}$ by \Palert{this many arrows},};
        \node at (5, 2.5) {calculated \Alert{mod 5}};
        \node at (5, 2) {Note I've collapsed $(a^n, b^m)$ into $a^n b^m$ for space reasons};
        %%
        \begin{scope}[shift={(-.25,6.5)}]
          \node (1) at (0:1) [v-y] {$1$};
          \node (r) at (72:1) [v] {$a$};
          \node (r2) at (144:1) [v] {$a^2$};
          \node (r3) at (216:1) [v] {$a^3$};
          \node (r4) at (288:1) [v] {$a^4$};
          \node at (180: 2) {\normalsize\Palert{$4^0 = 1$}};
        \end{scope}
        %%
        \begin{scope}[shift={(-.25,0)}]
          \node (s) at (0:1) [v] {$b$};
          \node (rs) at (72:1) [v] {$ab$};
          \node (r2s) at (144:1) [v] {$a^2\!b$};
          \node (r3s) at (216:1) [v] {$a^3\!b$};
          \node (r4s) at (288:1) [v] {$a^4\!b$};        
          \node at (180:2) {\normalsize\Palert{$4^1 = 4$}};
        \end{scope}
        %%
        \begin{scope}[shift={(10.25,0)}]
          \node (s2) at (0:1) [v] {$b^2$};
          \node (rs2) at (72:1) [v] {$ab^2$};
          \node (r2s2) at (144:1) [v] {$a^2\!b^{\!2}$};
          \node (r3s2) at (216:1) [v] {$a^3\!b^{\!2}$};
          \node (r4s2) at (288:1) [v] {$a^4\!b^{\!2}$};
          \node at (0:2) {\normalsize\Palert{$4^2 \equiv 1$}};
        \end{scope}
        %%
        \begin{scope}[shift={(10.25,6.5)}]
          \node (s3) at (0:1) [v] {$b^3$};
          \node (rs3) at (72:1) [v] {$ab^{\!3}$};
          \node (r2s3) at (144:1) [v] {$a^2\!b^{\!3}$};
          \node (r3s3) at (216:1) [v] {$a^3\!b^{\!3}$};
          \node (r4s3) at (288:1) [v] {$a^4\!b^{\!3}$};
          \node at (0:2) {\normalsize\Palert{$4^3 \equiv 4$}};
        \end{scope}
        %%
        \draw [B2] (1) to (s);
        \draw [B2] (r) to[bend left=13] (rs);
        \draw [B2] (r2) to[bend right=13] (r2s); 
        \draw [B2] (r3) to[bend left=15] (r3s);
        \draw [B2] (r4) to[bend right=13] (r4s);
        %%
        \draw [B2] (s) to (s2);
        \draw [B2] (rs) to (rs2);
        \draw [B2] (r2s) to (r2s2);
        \draw [B2] (r3s) to (r3s2);
        \draw [B2] (r4s) to (r4s2);
        %%
        \draw [B2] (s2) to (s3);
        \draw [B2] (rs2) to[bend right=13] (rs3);
        \draw [B2] (r2s2) to[bend left=13] (r2s3);
        \draw [B2] (r3s2) to[bend right=15] (r3s3);
        \draw [B2] (r4s2) to[bend left=13] (r4s3);
        %%
        \draw [B2] (s3) to (1); 
        \draw [B2] (rs3) to (r);
        \draw [B2] (r2s3) to (r2);
        \draw [B2] (r3s3) to (r3);
        \draw [B2] (r4s3) to (r4);
        %%
        \draw [r] (1) to (r); \draw [r] (r) to (r2); \draw [r] (r2) to (r3);
        \draw [r] (r3) to (r4); \draw [r] (r4) to (1);
        %%
        \draw [r] (s) to (r4s); \draw [r] (r4s) to (r3s); \draw [r] (r3s) to (r2s);
        \draw [r] (r2s) to (rs); \draw [r] (rs) to (s);
        %%
        \draw [r] (s2) to (rs2); \draw [r] (rs2) to (r2s2);
        \draw [r] (r2s2) to (r3s2); \draw [r] (r3s2) to (r4s2);
        \draw [r] (r4s2) to (s2);
        %%
        \draw [r] (s3) to (r4s3); \draw [r] (r4s3) to (r3s3);
        \draw [r] (r3s3) to (r2s3); \draw [r] (r2s3) to (rs3);
        \draw [r] (rs3) to (s3);
      \end{scope}
    \end{tikzpicture}
\]

\begin{problem}
    (Bonus!) Try constructing $\Alert{V_4} \rtimes \Balert{C_3}$, and also $\Alert{C_3} \rtimes \Balert{V_4}$. 
    \\(I promise both of them are ``compatible'' in the sense above.)
\end{problem}



\end{document}

