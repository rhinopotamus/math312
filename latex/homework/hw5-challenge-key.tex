\documentclass[12pt]{article}
% controlling the geometry of the page:
\usepackage[margin=1in, paperwidth=8.5in, paperheight=11in]{geometry} 
\usepackage{amsmath, amssymb} % useful math symbols and environments

\usepackage{multicol} % multiple columns side-by-side

\usepackage{amsthm} % Theorem-like environments
\theoremstyle{definition} % Without this line, theorem statements (and therefore problem statements etc.) show up in italic text.
\newtheorem{conjecture}{Conjecture}
\newtheorem{problem}{Problem}
\newtheorem*{remark}{Remark}

% pretty colors!
\usepackage[dvipsnames]{xcolor}
\colorlet{darkgrey}{black!70}
\colorlet{darkgreen}{green!50!black}


\usepackage{tikz} % for drawing diagrams
\usetikzlibrary{arrows,automata,positioning} 
\usetikzlibrary{decorations.markings}
\usetikzlibrary{decorations.pathreplacing}
\usetikzlibrary{patterns}
\usetikzlibrary{shapes.geometric}

%%---------------------------------------------------------------------------
%% included from visualalgebra.sty (see the beamer folder)
%% TEXT COLORS
%%
\definecolor{xRed}{rgb}{.9,0,0}       
\definecolor{xBlue}{rgb}{0,0,.9}      
\definecolor{xGreen}{HTML}{009000}   %% "Islamic green"
\definecolor{xPurple}{HTML}{D14FFF}  
\definecolor{xOrange}{HTML}{F56600}  %% "Clemson orange"

\newcommand{\Alert}[1]{\textcolor{xRed}{#1}}
\newcommand{\Balert}[1]{\textcolor{xBlue}{#1}}
\newcommand{\Galert}[1]{\textcolor{xGreen}{#1}}
\newcommand{\Palert}[1]{\textcolor{xPurple}{#1}}
\newcommand{\Oalert}[1]{\textcolor{xOrange}{#1}}
\newcommand{\Walert}[1]{\textcolor{white}{#1}}

%% vertices in cayley graphs
\tikzset{v/.style={circle, draw, fill=lightgray,inner sep=0pt, 
  minimum size=6mm}}

%% Edge colors
%%
\definecolor{eRed}{rgb}{1,0,0}      % Cayley diagram edges
\definecolor{eBlue}{rgb}{0,0,1}     % Cayley diagram edges
\definecolor{eGreen}{HTML}{7EC636}  % Goodnotes green (a little darker)
\definecolor{eGreen}{HTML}{3CAC13}  % I like this a litte better
\definecolor{ePurple}{HTML}{D287FF} % Close to goodnotes 
\colorlet{eOrange}{orange}

% Edge styles 
%%
\tikzset{r/.style={draw, very thick, eRed, -stealth}}  % Red -->
\tikzset{rr/.style={draw, very thick, eRed}}           % Red ---
\tikzset{r2/.style={draw, very thick, eRed,stealth'-stealth'}} % Red <-->
\tikzset{b/.style={draw, very thick, eBlue, -stealth}} % Blue -->
\tikzset{bb/.style={draw, very thick, eBlue}}          % Blue ---
\tikzset{b2/.style={draw, very thick, eBlue,stealth'-stealth'}} % Blue <-->
\tikzset{g/.style={draw, very thick, eGreen, -stealth}} % Green -->
\tikzset{gg/.style={draw, very thick, eGreen}}          % Green ---
\tikzset{g2/.style={draw, very thick, eGreen,stealth'-stealth'}} % Green <-->
\tikzset{p/.style={draw, very thick, ePurple, -stealth}} % Purple -->
\tikzset{pp/.style={draw, very thick, ePurple}}           % Purple ---
\tikzset{p2/.style={draw, very thick, ePurple,stealth'-stealth'}} % Purple <-->
\tikzset{o/.style={draw, very thick, eOrange, -stealth}} % Orange -->
\tikzset{oo/.style={draw, very thick, eOrange}}           % Orange ---
\tikzset{o2/.style={draw, very thick, eOrange,stealth'-stealth'}} % Orange <-->
%%
\tikzstyle{cy2} = [draw,very thick]           %% cycle graph edges
\definecolor{faded}{rgb}{.75,.75,.75}
\tikzstyle{f} = [faded]                         % This is used all the time
%%---------------------------------------------------------------------------
%%
%% COLORS FOR COSET BUBBBLES 
%%
\colorlet{cosetGray}{gray!15!white}
\colorlet{cosetBlue}{blue!15!white}
\colorlet{cosetRed}{red!15!white}
\definecolor{cosetPurple}{rgb}{.9 .84 .965}
\definecolor{cosetGreen}{rgb}{.863 .92 .855}
%%---------------------------------------------------------------------------

\usepackage{graphicx} % for inserting figures with \includegraphics
\usepackage{setspace} % for controlling space between lines, paragraphs, etc.

\usepackage{fancyhdr} % for controlling headers and footers
\usepackage{newtx} % changes the default font family
\usepackage[shortlabels]{enumitem} % controllable labels for ordered and unordered lists

\usepackage{hyperref} % controls hyperlinks, both internal and external
\hypersetup{
    colorlinks=true,
    urlcolor=blue,
}

\setlength{\headheight}{14.5pt}
\newcommand{\Q}{\mathbb{Q}}
\newcommand{\R}{\mathbb{R}}
\newcommand{\Z}{\mathbb{Z}}
\newcommand\inv{^{-1}} % I am very tired of typing ^{-1}
\def\<{\langle}
\def\>{\rangle}
\DeclareMathOperator\Rect{\mathbf{Rect}}
\DeclareMathOperator\Tri{\mathbf{Tri}}
\DeclareMathOperator\Sq{\mathbf{Sq}}
\DeclareMathOperator\Light{\mathbf{Light}}
\DeclareMathOperator{\lcm}{lcm}
%% Abstract algebra commands
\def\normal{\lhd}
\def\normaleq{\unlhd}
\def\nnormal{\ntriangleleft}
\def\nnormaleq{\ntrianglelefteq}

\newenvironment{red}{\color{red}}{\ignorespacesafterend}

% I don't like how LaTeX renders section headings by default
\renewcommand{\section}[1]{\begin{center} \textbf{#1} \\\end{center}}
%
\setlength{\parindent}{0in}
%\oddsidemargin=-.25in
\allowdisplaybreaks
\pagestyle{fancy}
\renewcommand{\headrulewidth}{0pt}
\lhead{MATH 312}
\rhead{Spring 2025}
%\lfoot{\copyright\ CLEAR Calculus 2010}
\cfoot{}
\renewcommand{\thefootnote}{*} 
\hyphenpenalty=10000 % LaTeX by default really likes hyphenating things

% all the stuff above this line is called the preamble...
%##################################################################
\begin{document} % this is always the first line of what's actually produced
\section{Homework \#5 - \Alert{Challenges key}} % notice that if you want the character # to appear, you have to "escape" it with a backslash

Here are proofs for each of the parts of the challenge problems that are proof-based. I have written them to purposefully be a bit annoying; your job is to use the data-claim-warrant structure to validate the proofs, and also to say what is annoying about them.

At least one of these proofs has a subtle error in it (that I originally made by honest accident but then decided to leave in for pedagogical purposes). Can you find it?

\begin{problem} Here we shall track down the details from our discussion of the mystery group of order 16 from class on Wednesday.
    \begin{enumerate}[(a)]
        \item Let $g\in G$ and suppose that $\<g\>$ is a normal subgroup of order 2. Prove that $g\in Z(G)$.
        \begin{proof}
            Let $x\in G$. $x\<g\>x\inv = \<g\>$, so either $xgx\inv = e$, in which case $xg = x$, so $g = 1$, which certainly isn't true, or else $xgx\inv = g$. Then $xg = gx$, so $g \in Z(G)$.
        \end{proof}
        \item Suppose that $G$ is generated by two generators, say $G = \<g, h \mid \ldots \>$. Prove that if $g\in Z(G)$, then $h\in Z(G)$.
        \begin{proof}
            A generic element of $G$ looks like $s_1^{p_1} s_2^{p_2} \ldots s_k^{p_k}$, where $p_i \in \Z$ and each $s_i$ is either $g$ or $h$. Therefore, it's enough to show that $h\cdot g^p = g^p \cdot h$. But since $g\in Z(G)$, $hg = gh$, so $hg^p = g^ph$. Therefore, $h\in Z(G)$.
        \end{proof}
        \item Let $G$ be a finitely generated group, say $G = \<g_1, \ldots, g_n \mid \ldots\>$. (Note that $G$ doesn't have to be finite -- the integers, for example, are finitely generated.) Prove that if all the generators $g_i \in Z(G)$, then $G$ is abelian.
        \begin{proof}
            A generic element of $G$ looks like $s_1^{p_1} s_2^{p_2} \ldots s_k^{p_k}$, where $p_i \in \Z$ and each $s_i$ is one of the generators $g_i$. Therefore, it's enough to show for generic generators $g_i$ and $g_j$ that $g_i \cdot g_j^p = g_j^p \cdot g_i$. Since $g_i \in Z(G)$, $g_i g_j = g_j g_i$. Therefore, $g_i g_j^p = g_j^p g_i$, so $G$ is abelian.
        \end{proof}
        \item Now, getting more specific: in the mystery group, we knew that $s^2 = r^8 = 1$. How did we know those two things?

        \begin{proof}
            (Well, not really a proof, more of just an observation.) \\
            By looking at the lattice, $|\<s\>| = 2$ and $|\<r\>| = 8$.
        \end{proof}
        \item Suppose that $\<s\>$ and $\<r^4s\>$ \textit{aren't} normal; therefore they must be conjugate. Prove that $srs = r^5$. (Hint: conjugate by $r$.)

        \begin{proof}
            First, note that $r\not\in \<r^4s\>$. Therefore, $r\<r^4s\>r\inv = \<s\>$. Either $r(r^4s)r\inv = s$, in which case $r^5s = sr$ so $r^5 = srs$, or $r(r^4s)r\inv = 1$, in which case $r^5s = r$, so $s = r^4$. But $s$ certainly doesn't equal $r^4$, so $srs = r^5$.
        \end{proof}
    \end{enumerate}
\end{problem}

\pagebreak

\begin{problem}
    Write down a full proof of Lagrange's theorem: 
    
    \begin{center}
        if $H\leq G$, then $|H|$ divides $|G|$, and further, $|G| = [G:H]\cdot |H|$.
    \end{center}
    
    (This just entails stringing together the arguments we made on the slides before the Lagrange's theorem slide, but I think it's moderately nice to see it all written out.)

    \begin{proof}
        It's nice to split this proof up into three little lemmas:
        \begin{itemize}
            \item First, note that for any $g\in G$, $|gH| = |H|$. To prove this, consider the function $\phi:gH \to H$ defined by $\phi(gh) = h$. This function is injective (aka 1-1): if $gh_1 = gh_2$, then $h_1 = h_2$, so $\phi(gh_1) = \phi(gh_2)$. This function is also surjective (aka onto): if $h\in H$, then $\phi(gh) = h$. Therefore, this function is a bijection, so $|gH| = |H|$.
            \item Next, note that distinct cosets are disjoint. For suppose $g\in g_1H$ and $g\in g_2H$. Then there exist $h_1, h_2\in H$ such that $g_1h_1 = g = g_2h_2$. Therefore, $g_1 = g_2(h_2 h_1\inv)$, so $g_1H = g_2H$.
            \item Finally, note that the cosets cover all of $G$, because if $g\in G$, then $g\in gH$.
        \end{itemize}
        So: if $|H| = m$ and $[G:H]=n$, then $G$ is made up of $n$ sets of $m$ elements, so $|G| = mn$.
    \end{proof}
\end{problem}
\begin{problem}
    Prove that $|\operatorname{cl}_G(H)| = [G:N_G(H)]$.
    \begin{proof}
        The human-words translation of this sentence is that the number of subgroups conjugate to $H$ is the same as the number of cosets of $N_G(H)$. Because I am already annoyed at typing $N_G(H)$ repeatedly, let's just call it $N$. 
        
        Let's establish a bijection between cosets of $N$ and conjugate subgroups of $H$; specifically, let's map $xN$ to $xHx\inv$. 
        \begin{itemize}
            \item This map is clearly surjective.
            \item This map is injective: if $xHx\inv = yHy\inv$, then $(y\inv x) H (y\inv x)\inv = H$, so $y\inv x \in N$. This means that the coset $y\inv x N$ is the identity coset $N$. Therefore, $yN = y\left(y\inv xN\right) = xN$.
            \item (A secret third thing should be here. Do you know what it is?)
        \end{itemize}
    \end{proof}
\end{problem}

\end{document}

