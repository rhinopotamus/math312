\documentclass[12pt]{article}
% controlling the geometry of the page:
\usepackage[margin=1in, paperwidth=8.5in, paperheight=11in]{geometry} 
\usepackage{amsmath, amssymb} % useful math symbols and environments

\usepackage{multicol} % multiple columns side-by-side

\usepackage{amsthm} % Theorem-like environments
\theoremstyle{definition} % Without this line, theorem statements (and therefore problem statements etc.) show up in italic text.
\newtheorem{conjecture}{Conjecture}
\newtheorem{problem}{Problem}
\newtheorem*{definition}{Definition}

% pretty colors!
\usepackage[dvipsnames]{xcolor}
\colorlet{darkgrey}{black!70}
\colorlet{darkgreen}{green!50!black}

\definecolor{xRed}{rgb}{.9,0,0}
\newcommand{\Alert}[1]{\textcolor{xRed}{#1}}

\usepackage{tikz} % for drawing diagrams
\usetikzlibrary{arrows,automata,positioning} 
\usetikzlibrary{decorations.markings}
\usetikzlibrary{decorations.pathreplacing}
\usetikzlibrary{patterns}
\usetikzlibrary{shapes.geometric}

\usepackage{graphicx} % for inserting figures with \includegraphics
\usepackage{setspace} % for controlling space between lines, paragraphs, etc.

\usepackage{fancyhdr} % for controlling headers and footers
\usepackage{newtx} % changes the default font family
\usepackage[shortlabels]{enumitem} % controllable labels for ordered and unordered lists

\usepackage{hyperref} % controls hyperlinks, both internal and external
\hypersetup{
    colorlinks=true,
    urlcolor=blue,
}

\setlength{\headheight}{14.5pt}
\newcommand{\Q}{\mathbb{Q}}
\newcommand{\R}{\mathbb{R}}
\newcommand{\Z}{\mathbb{Z}}
\newcommand{\Zstar}{\mathbb{Z} - \{0\}}
\newcommand\inv{^{-1}} % I am very tired of typing ^{-1}
\def\<{\langle}
\def\>{\rangle}
\DeclareMathOperator\Rect{\mathbf{Rect}}
\DeclareMathOperator\Tri{\mathbf{Tri}}
\DeclareMathOperator\Sq{\mathbf{Sq}}
\DeclareMathOperator\Light{\mathbf{Light}}
\DeclareMathOperator{\lcm}{lcm}

\newenvironment{red}{\color{red}}{\ignorespacesafterend}

% I don't like how LaTeX renders section headings by default
\renewcommand{\section}[1]{\begin{center} \textbf{#1} \\\end{center}}
%
\setlength{\parindent}{0in}
%\oddsidemargin=-.25in
\allowdisplaybreaks
\pagestyle{fancy}
\renewcommand{\headrulewidth}{0pt}
\lhead{MATH 312}
\rhead{Spring 2025}
%\lfoot{\copyright\ CLEAR Calculus 2010}
\cfoot{}
\renewcommand{\thefootnote}{*} 
\hyphenpenalty=10000 % LaTeX by default really likes hyphenating things

% all the stuff above this line is called the preamble...
%##################################################################
\begin{document} % this is always the first line of what's actually produced
\section{Divides, gcd, and lcm}

Here is some stuff about the integers $\Z$ that we learn at some point and then eventually forget until it crops up again in surprising ways in abstract algebra.

\begin{definition}[divides]
    If $a, b\in \Z$ with $a \neq 0$, we say that $a$ \Alert{divides} $b$ if there is some other integer $c\in Z$ such that $b = a\cdot c$. If so, we write $a \mid b$; if not, we write $a \not\mid b$.
\end{definition}

Observations:

\begin{itemize}
    \item Example: $12 \mid 120$ because $120 = 12 \cdot 10$.
    \item $3 \not\mid 5$ because if I tried to write $5 = 3\cdot c$, then $c$ would have to be $\frac53$, which is not an integer.
    \item ``$a$ divides $b$'' is, annoyingly, synonymous with ``$b$ is a multiple of $a$''.
    \item Every integer divides 0.
    \item Every integer divides itself.
    \item 1 divides every integer.
\end{itemize}

\hrulefill

\begin{definition}[greatest common divisor, or gcd]
    If $a, b \in \Zstar$, then there is a unique integer $d\in\Zstar$, called the \Alert{greatest common divisor} of $a$ and $b$, such that:
    \begin{enumerate}[(a)]
        \item $d\mid a$ and $d\mid b$ (so $d$ is a ``common divisor'' of $a$ and $b$)
        \item if there's some integer $e\in \Z$ such that $e\mid a$ and $e \mid b$, then $e\mid d$ (so $d$ is the ``greatest'' aka ``biggest'' of all the common divisors of $a$ and $b$).
    \end{enumerate}
    We write $d = \gcd(a, b)$, or, whenever it's clear from context that this isn't an ordered pair or something, $d = (a, b)$.
\end{definition}

\begin{definition}[relatively prime]
    If $\gcd(a, b) = 1$, then we say that $a$ and $b$ are \Alert{relatively prime}.
\end{definition}

Observations:
\begin{itemize}
    \item 1 always divides both $a$ and $b$. If $a$ and $b$ are relatively prime, the point is that \textit{nothing else} divides both $a$ and $b$.
    \item If the numbers are smallish, you can decide about their gcd by looking at their prime factorization (but that's hard if the numbers are big). Otherwise, you can repeatedly do long division with remainders -- this is called the \Alert{Euclidean algorithm}.
    \item Example: $\gcd(180, 24) = \gcd(2^2\cdot 3^2 \cdot 5,\, 2^3 \cdot 3) = 2^2 \cdot 3 = 12$.
    \item Example: $10 = 2\cdot 5$ and $21 = 3 \cdot 7$ are relatively prime.
    \item Any two distinct prime numbers are relatively prime.
\end{itemize}

\pagebreak

\begin{definition}[least common multiple, or lcm]
    If $a, b \in \Zstar$, then there is a unique integer $\ell\in\Zstar$, called the \Alert{least common multiple} of $a$ and $b$ (or $\Alert{\lcm}(a, b)$) such that:
    \begin{enumerate}[(a)]
        \item $a\mid \ell$ and $b \mid \ell$ (so $\ell$ is a ``common multiple'' of $a$ and $b$)
        \item if there's some integer $m\in \Z$ such that $a\mid m$ and $b \mid m$, then $\ell \mid b$ (so $\ell$ is the ``least'' aka ``smallest'' of all the common multiples of $a$ and $b$).
    \end{enumerate}
\end{definition}

Observations:
\begin{itemize}
    \item I really wish it was called the ``smallest common multiple,'' because ``least-common'' sounds like it would mean something like ``rarest.'' 
    \item The problem is exactly that it's ``least common-multiple,'' not ``least-common multiple,'' if you see what I mean.
    \item Example: $\lcm(10, 15) = \lcm(2\cdot 5, 3\cdot 5) = 2\cdot 3 \cdot 5 = 30$. Check that $30 \mid 10 \cdot 15 = 150$.
    \item $ab$ is certainly a common multiple of $a$ and $b$, so $\ell \mid ab$.
    \item $\lcm(a, b) \cdot \gcd(a, b) = ab$. (!!)
    \item If $a$ and $b$ are relatively prime, then $\lcm(a, b) = ab$.
\end{itemize}

\end{document}

