\documentclass[12pt]{article}
% controlling the geometry of the page:
\usepackage[margin=1in, paperwidth=8.5in, paperheight=11in]{geometry} 
\usepackage{amsmath, amssymb} % useful math symbols and environments

\usepackage{multicol} % multiple columns side-by-side

\usepackage{amsthm} % Theorem-like environments
\theoremstyle{definition} % Without this line, theorem statements (and therefore problem statements etc.) show up in italic text.
\newtheorem{conjecture}{Conjecture}
\newtheorem{problem}{Problem}
\newtheorem*{definition}{Definition}
\newtheorem*{exercise}{Exercise}

% pretty colors!
\usepackage[dvipsnames]{xcolor}
\colorlet{darkgrey}{black!70}
\colorlet{darkgreen}{green!50!black}

\definecolor{xRed}{rgb}{.9,0,0}
\newcommand{\Alert}[1]{\textcolor{xRed}{#1}}
\definecolor{xGreen}{HTML}{009000}   %% "Islamic green"
\newcommand{\Galert}[1]{\textcolor{xGreen}{#1}}

\usepackage{tikz} % for drawing diagrams
\usetikzlibrary{arrows,automata,positioning} 
\usetikzlibrary{decorations.markings}
\usetikzlibrary{decorations.pathreplacing}
\usetikzlibrary{patterns}
\usetikzlibrary{shapes.geometric}

\usepackage{graphicx} % for inserting figures with \includegraphics
\usepackage{setspace} % for controlling space between lines, paragraphs, etc.

\usepackage{fancyhdr} % for controlling headers and footers
\usepackage{newtx} % changes the default font family
\usepackage[shortlabels]{enumitem} % controllable labels for ordered and unordered lists

\usepackage{hyperref} % controls hyperlinks, both internal and external
\hypersetup{
    colorlinks=true,
    urlcolor=blue,
}

\setlength{\headheight}{14.5pt}
\newcommand{\Q}{\mathbb{Q}}
\newcommand{\R}{\mathbb{R}}
\newcommand{\Z}{\mathbb{Z}}
\newcommand{\Zstar}{\mathbb{Z} - \{0\}}
\newcommand\inv{^{-1}} % I am very tired of typing ^{-1}
\def\<{\langle}
\def\>{\rangle}
\DeclareMathOperator\Rect{\mathbf{Rect}}
\DeclareMathOperator\Tri{\mathbf{Tri}}
\DeclareMathOperator\Sq{\mathbf{Sq}}
\DeclareMathOperator\Light{\mathbf{Light}}
\DeclareMathOperator{\lcm}{lcm}
\DeclareMathOperator{\im}{im}
\DeclareMathOperator{\id}{id}

\newenvironment{red}{\color{red}}{\ignorespacesafterend}

% I don't like how LaTeX renders section headings by default
\renewcommand{\section}[1]{\begin{center} \large\textbf{#1} \\\end{center}}
%
\setlength{\parindent}{0in}
%\oddsidemargin=-.25in
\allowdisplaybreaks
\pagestyle{fancy}
\renewcommand{\headrulewidth}{0pt}
\lhead{MATH 312}
\rhead{Spring 2025}
%\lfoot{\copyright\ CLEAR Calculus 2010}
\cfoot{}
\renewcommand{\thefootnote}{*} 
\hyphenpenalty=10000 % LaTeX by default really likes hyphenating things

% all the stuff above this line is called the preamble...
%##################################################################
\begin{document} % this is always the first line of what's actually produced
\section{What do isomorphisms do?}

I keep saying that \Alert{isomorphisms respect algebraic structure}. This is a hugely-encompassing idea and I want to unpack what I mean and what some of the consequences are.

\subsection*{What is an isomorphism?}

An isomorphism is a homomorphism that is also a bijection.

\subsection*{Okay, smartass, what is a homomorphism?}

Suppose that $(G, \star)$ and $(H, \odot)$ are two groups. Then $\phi:G \to H$ is a homomorphism if \[\phi(g_1 \star g_2) = \phi(g_1) \odot \phi(g_2).\] 

\begin{exercise}
    Circle three different things in that equation that are elements of $H$.
\end{exercise}

Morally what this means is that \Alert{a homomorphism is a function that respects the groups' operations.} Another good maxim here is that \Galert{a homomorphism sends products to products.} 

As a consequence of respecting the groups' \Alert{operations}, a homomorphism respects the groups' \Alert{algebraic structures}. Specifically:

\begin{exercise} Prove each of the following statements:
    \newcommand\hitem{\item A homomorphism sends }
    \begin{itemize}
        \hitem the identity to the identity. 
        \begin{proof} 
            Say that $e_G$ is the identity in $G$ and $e_H$ is the identity in $H$. Consider $\phi(e_G \star g)$. On the one hand, since $e_G\star g = g$, $\phi(e_G \star g) = \phi(g)$. On the other hand, using the homomorphism property, $\phi(e_G \star g) = \phi(e_G) \odot \phi(g)$. Therefore, \[\phi(g) = \phi(e_G)\odot \phi(g).\]
            Well, $\phi(g)$ is some element of $H$, so it has an inverse. Let's $H$-multiply both sides of this equation by the inverse on the right:
            \begin{align*}
                \phi(g)\odot [\phi(g)]\inv &= \phi(e_G) \odot \phi(g) \odot [\phi(g)]\inv \\
                e_H &= \phi(e_G) \odot \left(\phi(g) \odot [\phi(g)]\inv\right) \\
                e_H &= \phi(e_G) \odot e_H \\
                e_H &= \phi(e_G).
            \end{align*}So: $\phi$ sends $e_G$ to $e_H$.
        \end{proof}
        \hitem inverses to inverses.
        \hitem $G$ to a subgroup of $H$. (Vocabulary: the \Alert{image} of $G$ under $\phi$ is the set $\im(\phi) = \left\{\phi(g)\mid g\in G\right\}.$ Certainly this is a sub\textit{set} of $H$, but is it a sub\textit{group} of $H$?)
        \hitem powers to powers.
        \hitem orbits to orbits.
        \hitem conjugates to conjugates.
        \hitem conjugacy classes to conjugacy classes.
    \end{itemize}
\end{exercise}

Here are some examples of homomorphisms.

\begin{exercise} Prove that each of these ``sends products to products'':
    \begin{itemize}
        \item Squish everything in $G$ down to the identity in $H$. (This is a rude homomorphism.)
        \begin{itemize}
            \item Ponder: How does this map send orbits to orbits?
        \end{itemize}
        \item Do nothing. (Define the ``identity map'' $\id:G\to G$ as $\id(g) = g$.)
        \item If $G \leq H$, define the ``inclusion map'' $\iota:G\to H$ as $\iota(g) = g$.
        \item Define the ``exponential map'' $\exp:(\R, +) \to (\R^+, *)$ by $\exp(x) = e^x$.
        \item $\ln:(\R^+, *) \to (\R, +)$.
        \begin{itemize}
            \item This is, like, the best explanation for why the properties of logs are like that.
        \end{itemize}
        \item Here is an interesting \textbf{non}-example: Let $s:D_n \to D_n$ be the ``squaring map'' $s(x) = x^2$. (Hint: Remember that $D_n$ isn't abelian and compare $s(fr)$ to $s(f) s(r)$.)
        \item If $G$ is an \textbf{abelian} group, then the squaring map $s:G\to G$ is indeed a homomorphism.
        \item Define $\phi:Q_8 \to V_4$ as follows: $\phi(\pm 1) = 1, \quad \phi(\pm i) = a, \quad \phi(\pm j) = b, \quad \phi(\pm k) = ab.$
        \item Define the ``projection map'' $\pi_A:A\times B \to A$ as $\pi_A(a,b) = a$. (Similar for $\pi_B$.)
    \end{itemize}
\end{exercise}

\subsection*{What about isomorphisms?}
A general theme in math is that if you make something more special, you get stronger results. By adding ``bijection'' to ``homomorphism,'' you can thus expect to preserve even more structure.

\begin{exercise} Let $\phi:G\to H$ be an isomorphism. Prove that:
    \begin{itemize}
        \item $|\phi(g)| = |g|$. (``$\phi$ preserves orders.'')
        \begin{proof}
            Say that $|g| = n$ -- that is, $g^n = e$, but for any $k < n$, $g^k \neq e$. We need to show those two things are also true for $\phi(g)$. The first part is easy: since $\phi$ is a homomorphism, it sends powers to powers and the identity to the identity, so $\phi(g)^n = \phi(g^n) = \phi(e) = e$.

            For the second part, consider the orbit of $g$, $\<g\> = \{g, g^2, \ldots, g^{n-1}, g^n = e\}$. All these powers of $g$ are distinct. (Why?) So, since $\phi$ is a bijection (and in particular is 1-1), all their images $\{\phi(g), \phi(g^2), \ldots, \phi(g^{n-1}), \phi(g^n) = e\}$ are distinct. But since $\phi$ sends powers to powers, that list of distinct elements is $\{\phi(g), \phi(g)^2, \ldots \phi(g)^{n-1}, e\}$. Therefore, $\phi(g)^k\neq e$ for any $k < n$ -- $e$ is in that list of distinct elements at the end, so nobody else gets to be $e$.
        \end{proof}
        \item Corollary: $\phi$ sends orbits to orbits \textit{of the same size.}
        \item $\phi$ sends conjugacy classes to conjugacy classes \textit{of the same size.}
        \item $\phi$ sends subgroups to subgroups \textit{of the same size.}
    \end{itemize}
\end{exercise}

If there is an isomorphism $\phi:G \to H$, we say that $G$ and $H$ are \Alert{isomorphic} and write $G\cong H$. Since an isomorphism $\phi$ preserves \textit{so much} algebraic structure, this is why it's our formal version of the idea that $G$ and $H$ are ``basically the same'' but maybe just got relabeled or re-presented.

\begin{exercise} Suppose that $G\cong H$. Prove that:
    \begin{itemize}
        \item $G$ is abelian if and only if $H$ is abelian.
        \item $|G| = |H|$.
    \end{itemize}
\end{exercise}


\end{document}

