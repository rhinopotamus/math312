\documentclass[8pt, handout]{beamer} 


%% Math packages
%%
\usepackage{amsmath,amsthm,amssymb}
% Removes the "Too many math alphabets used in version normal" error.
\newcommand\hmmax{0}
\newcommand\bmmax{0}
\usepackage[new]{old-arrows}
\usepackage{cancel}
\usepackage{mathdots}
\usepackage{venndiagram}
\usepackage{mathrsfs}          % Math script font

% Graphics
%%
\graphicspath{{./}{figs/}}
\usepackage{graphicx}
\usepackage{tikz}
\usetikzlibrary{arrows}
\usetikzlibrary{decorations.markings}
\usetikzlibrary{decorations.pathreplacing}
\usetikzlibrary{patterns}
\usetikzlibrary{shapes.geometric}
\usetikzlibrary{positioning}
\usetikzlibrary{matrix}
\usepackage{tikz-3dplot}
\usepackage{tkz-graph}
\usepackage{tikz-cd}

%% Colors 
%%
\usepackage{xcolor}
\usepackage{color}
\usepackage{visualalgebra}  %% Put this *after* the TikZ packages
\usepackage{visualalgebraslides}  %% Put this *after* "visualalgebra"

%% Page layout packages
%%
\usepackage{url}
\usepackage{multicol}
\usepackage{multirow}
\usepackage[numbers,square,sort&compress]{natbib}

%% Font and formatting packages
%%
\usepackage[english]{babel}    % Removing this causes compiler error
\usepackage{alltt}             % Like verbatim, but excludes \ and { }
\usepackage{enumerate}         % [shortlabels] option??
\usepackage{comment}
\usepackage{soul}              % strikeout text
\usepackage{bm}                % Bold math
\usepackage[T1]{fontenc}
\usepackage{relsize}

%% Fixes the \mathbf{} not working for fonts under 10pt
\usepackage{cmbright}
\fontencoding{OT1}\fontfamily{cmbr}\selectfont %to load ot1cmbr.fd
\DeclareFontShape{OT1}{cmbr}{bx}{n}{% change bx definition
<->cmbrbx10%
}{}
\normalfont

\makeatletter
\renewcommand*\env@matrix[1][\arraystretch]{%
  \edef\arraystretch{#1}%
  \hskip -\arraycolsep
  \let\@ifnextchar\new@ifnextchar
  \array{*\c@MaxMatrixCols c}}
\makeatother


%%=======================================================================

%% Beamer packages
%%
\mode<presentation>
{
  \usetheme{boadilla} 
  \useinnertheme{rectangles}
  \usecolortheme{dolphin}
}

\setbeamersize{text margin left=6mm}
\setbeamersize{text margin right=6mm}
\setbeamersize{sidebar width right=0mm}
\setbeamersize{sidebar width left=0mm}
\setbeamertemplate{navigation symbols}{}

\def\newblock{\hskip .11em plus .33em minus .07em}

% Other options: ball, circle, square 
\setbeamertemplate{enumerate items}[default]
%\setbeamercolor{enumerate subitem}{fg=red!80!black}
\def\opacity{0.5}
%\setbeamercovered{transparent}
\setbeamercovered{invisible}

\newcommand{\Pause}{\pause}      %% Comment this out => lots more page breaks


\AtBeginSection[]{
  \begin{frame}
  \vfill
  \centering
  \begin{beamercolorbox}[sep=8pt,center,shadow=true,rounded=true]{title}
    \usebeamerfont{title}\insertsectionhead\par%
  \end{beamercolorbox}
  \vfill
  \end{frame}
}

%%====================================================================

\title[Rings and fields!]{Rings and fields!}

\author[\href{mailto:sbagley@westminsteru.edu}{S. Bagley}]
       {\href{mailto:sbagley@westminsteru.edu}{Spencer Bagley}}

\institute[Westminster] { 
  \normalsize With many thanks to Matthew Macauley, \\
  \url{http://www.math.clemson.edu/~macaule/}}

\date[28 Apr 2025]{28 Apr 2025}

\begin{document}

\frame{\titlepage}

%%====================================================================

\section{Groups!}

%%====================================================================

\begin{frame}{What is a group? Why is a group?} 

  \begin{block}{Definition} 
    A \Palert{group} is ...
    \onslide<2->{
      \Alert{a set of elements $G$} together with a \Balert{binary operation $\ast$} such that:
    }
    \onslide<3->{
      \begin{itemize}
        \item $\ast$ has an \Alert{identity} element $e$ such that $g\ast e = g$, and $e\ast g = g$, for all $g\in G$
        \onslide<4->{\item every element $g$ has an \Alert{inverse} element $g\inv$ such that $g\ast g\inv = g\inv \ast g = e$}
        \onslide<5->{\item the operation $\ast$ is \Balert{associative}, ie., $(g\ast h)\ast k = g\ast(h\ast k)$}
        \onslide<6->{\item (the set $G$ is \Balert{closed} under $\ast$, but that's implied by the precise definition of a binary operation)}
      \end{itemize}
    }
  \end{block}

  \medskip

  Why are we making this definition?

\end{frame}

%%====================================================================

\begin{frame}{Why is a group?}
  One reason to make this definition is that there are lots ways to \Balert{combine} \Alert{stuff}
  \\ that remind us a bit of \Balert{multiplying} \Alert{numbers}. \pause

  \medskip
  
  If we forget some specific things we know about \Alert{numbers}, what is still true about \Balert{multiplying}? \pause
  
  \medskip

  How much can we forget and still have something that ``works like'' \Balert{multiplying} \Alert{numbers}?
\end{frame}

%%====================================================================

\begin{frame}{Let's play rock-paper-scissors} 

  Let $M := \{r, p, s\}$ and define the binary operation $\ast$ as the winner between the two throws. For instance, $r\ast p = p$ because paper beats rock.

  \medskip \pause

  \[
    \begin{array}{c|ccc}
      \ast & r & p & s \\\hline
         r & r & p & r\\
         p & p & p & s\\
         s & r & s & s     
    \end{array}
  \]

  \medskip \pause

  This is not a group. Why not? \pause
  \begin{itemize}
    \item Is there an identity element? 
    \item Do elements have inverses?
    \item Is the operation associative? (Check $r\ast(p \ast s)$ vs. $(r\ast p)\ast s$.)
  \end{itemize} \pause

  \medskip

  We had to forget even more stuff about \Balert{multiplying} \Alert{numbers}! This is called a \Galert{magma}.
  
\end{frame}

%%====================================================================

\begin{frame}{Forgetting more stuff}

  \[
  \begin{tikzpicture}[xscale=3, yscale=2]
    \node at ( 0, 0) {magma};
    \node at (-1, 1) {unital magma};
    \node at ( 0, 1) {semigroup};
    \node at ( 1, 1) {quasigroup};
    \node at (-1, 2) {monoid};
    \node at ( 0, 2) {loop};
    \node at ( 1, 2) {assoc. quasigroup};
    \node at ( 0, 3) {group};
    %%
    \draw[r, shorten >= 12pt, shorten <= 12pt] (0, 0) -- (-1, 1) node [midway, left] {id.};
    \draw[r, shorten >= 12pt, shorten <= 12pt] (0, 1) -- (-1, 2) node [pos = 0.6, left] {id.};
    \draw[r, shorten >= 12pt, shorten <= 12pt] (1, 1) -- ( 0, 2) node [pos = 0.6, left] {id.};
    \draw[r, shorten >= 12pt, shorten <= 12pt] (1, 2) -- ( 0, 3) node [midway, left] {id.};
    %%
    \draw[b, shorten >= 5pt, shorten <= 5pt] ( 0, 0) -- ( 0, 1) node [midway, left] {assoc.};
    \draw[b, shorten >= 5pt, shorten <= 5pt] (-1, 1) -- (-1, 2) node [midway, left] {assoc.};
    \draw[b, shorten >= 5pt, shorten <= 5pt] ( 1, 1) -- ( 1, 2) node [midway, right] {assoc.};
    \draw[b, shorten >= 5pt, shorten <= 5pt] ( 0, 2) -- ( 0, 3) node [midway, left] {assoc.};
    %%
    \draw[g, shorten >= 12pt, shorten <= 12pt] ( 0, 0) -- ( 1, 1) node [midway, right] {inv.};
    \draw[g, shorten >= 12pt, shorten <= 12pt] ( 0, 1) -- ( 1, 2) node [pos = 0.6, right] {inv.};
    \draw[g, shorten >= 12pt, shorten <= 12pt] (-1, 1) -- ( 0, 2) node [pos = 0.6, right] {inv.};
    \draw[g, shorten >= 12pt, shorten <= 12pt] (-1, 2) -- ( 0, 3) node [midway, left] {inv.};
  \end{tikzpicture}
  \]
  
\end{frame}

%%====================================================================

\begin{frame}{Or, adding more stuff on}
  Even groups don't remind me that much of how numbers work: 
  
  numbers have \Alert{two} operations. \pause
  
  \medskip

  \begin{block}{Definition}
    A \Alert{ring} $(R, +, \cdot)$ is a set of elements together with \Alert{two} binary operations $+$ and $\cdot$, such that: \pause
    \begin{itemize}
      \item $R$ is an abelian group under $+$ (with identity called $0$) \pause
      \item $R$ has an element called $1$ such that $1\cdot a = a \cdot 1 = a$ for all $a\in R$ \pause
      \item The operation $\cdot$ is associative \pause
      \item $(a+b)\cdot c = a\cdot c + b\cdot c$
      \item $c\cdot (a+b) = c\cdot a + c\cdot b$
    \end{itemize}
  \end{block}\pause 

  \begin{exampleblock}{Morally:}
    $+$ is addition and $\cdot$ is multiplication and that's the distributive law.
  \end{exampleblock} \pause

  \medskip

  Examples?

  \medskip

  What are we still forgetting about how numbers work?
\end{frame}

%%====================================================================

\begin{frame}{I would like division please}
  \begin{block}{Definition}
    A \Alert{field} $(F, +, \cdot)$ is a set of elements together with \Alert{two} binary operations $+$ and $\cdot$, such that:*
    \begin{itemize}
      \item $F$ is an abelian group under $+$ (with identity called $0$) \pause
      \item $F - \{0\}$ is an abelian group under $\cdot$ (with identity called $1$) \pause
      \item $(a+b)\cdot c = a\cdot c + b\cdot c$
    \end{itemize}
  \end{block}
\end{frame}

%%====================================================================
\section*{The end!}
%%====================================================================

\end{document}
