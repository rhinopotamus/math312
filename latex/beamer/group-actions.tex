\documentclass[8pt, handout]{beamer} 

%% Math packages
%%
\usepackage{amsmath,amsthm,amssymb}
% Removes the "Too many math alphabets used in version normal" error.
\newcommand\hmmax{0}
\newcommand\bmmax{0}
\usepackage[new]{old-arrows}
\usepackage{cancel}
\usepackage{mathdots}
\usepackage{venndiagram}
\usepackage{mathrsfs}          % Math script font

% Graphics
%%
\graphicspath{{./}{figs/}}
\usepackage{graphicx}
\usepackage{tikz}
\usetikzlibrary{arrows}
\usetikzlibrary{decorations.markings}
\usetikzlibrary{decorations.pathreplacing}
\usetikzlibrary{patterns}
\usetikzlibrary{shapes.geometric}
\usetikzlibrary{positioning}
\usetikzlibrary{matrix}
\usepackage{tikz-3dplot}
\usepackage{tkz-graph}
\usepackage{tikz-cd}

%% Colors 
%%
\usepackage{xcolor}
\usepackage{color}
\usepackage{visualalgebra}  %% Put this *after* the TikZ packages
\usepackage{visualalgebraslides}  %% Put this *after* "visualalgebra"

%% Page layout packages
%%
\usepackage{url}
\usepackage{multicol}
\usepackage{multirow}
\usepackage[numbers,square,sort&compress]{natbib}

%% Font and formatting packages
%%
\usepackage[english]{babel}    % Removing this causes compiler error
\usepackage{alltt}             % Like verbatim, but excludes \ and { }
\usepackage{enumerate}         % [shortlabels] option??
\usepackage{comment}
\usepackage{soul}              % strikeout text
\usepackage{bm}                % Bold math
\usepackage[T1]{fontenc}
\usepackage{relsize}

%% Fixes the \mathbf{} not working for fonts under 10pt
\usepackage{cmbright}
\fontencoding{OT1}\fontfamily{cmbr}\selectfont %to load ot1cmbr.fd
\DeclareFontShape{OT1}{cmbr}{bx}{n}{% change bx definition
<->cmbrbx10%
}{}
\normalfont

\makeatletter
\renewcommand*\env@matrix[1][\arraystretch]{%
  \edef\arraystretch{#1}%
  \hskip -\arraycolsep
  \let\@ifnextchar\new@ifnextchar
  \array{*\c@MaxMatrixCols c}}
\makeatother


%%=======================================================================

%% Beamer packages
%%
\mode<presentation>
{
  \usetheme{boadilla} 
  \useinnertheme{rectangles}
  \usecolortheme{dolphin}
}

\setbeamersize{text margin left=6mm}
\setbeamersize{text margin right=6mm}
\setbeamersize{sidebar width right=0mm}
\setbeamersize{sidebar width left=0mm}
\setbeamertemplate{navigation symbols}{}

\def\newblock{\hskip .11em plus .33em minus .07em}

% Other options: ball, circle, square 
\setbeamertemplate{enumerate items}[default]
%\setbeamercolor{enumerate subitem}{fg=red!80!black}
\def\opacity{0.5}
%\setbeamercovered{transparent}
\setbeamercovered{invisible}

\newcommand{\Pause}{}      %% Comment this out => lots more page breaks


\AtBeginSection[]{
  \begin{frame}
  \vfill
  \centering
  \begin{beamercolorbox}[sep=8pt,center,shadow=true,rounded=true]{title}
    \usebeamerfont{title}\insertsectionhead\par%
  \end{beamercolorbox}
  \vfill
  \end{frame}
}

%%====================================================================

\title[Group actions!]{Group actions!}

\author[\href{mailto:sbagley@westminsteru.edu}{S. Bagley}]
       {\href{mailto:sbagley@westminsteru.edu}{Spencer Bagley}}

\institute[Westminster] { 
  \normalsize With many thanks to Matthew Macauley, \\
  \url{http://www.math.clemson.edu/~macaule/}}

\date[7 Apr 2025]{7 Apr 2025}

\begin{document}

\frame{\titlepage}


%%====================================================================

%\end{document}

%%====================================================================

\begin{frame}{Overview} %\Pause

  Intuitively, a \Alert{group action} occurs when a group $G$
  ``naturally permutes'' a set $S$ \emph{of states}.

  \medskip\Pause
  
  For example: \smallskip
  \begin{itemize}
  \item The ``Rubik's cube group'' consists of the $4.3\times 10^{19}$
    \Alert{actions} that \emph{permute} the
    $4.3\times 10^{19}$ \Balert{configurations} of the cube. \Pause
  \item The group $D_4$ consists of the 8 \Alert{symmetries} of the
    square. These symmetries are \emph{actions} that
    \emph{permute} the $8$ \Balert{configurations} of the
    square.
  \end{itemize}
  
  \medskip\Pause 
  
  Group actions formalize the interplay between the actual
  \Alert{group of actions} and the \Balert{sets of objects} that they
  ``rearrange.''
  
  \medskip\Pause
  
  There are many other examples of groups that ``act on'' sets of
  objects. We will see examples when the group and the set have
  different sizes.
  
  \medskip\pause

  The rich theory of group actions can be used to prove many deep
  results in group theory.

  \medskip\Pause

  We have actually already seen many group actions, without knowing
  it, such as: \smallskip
  \begin{itemize}
  \item groups acting on themselves by multiplication
  \item groups acting on themselves by conjugation
  \item groups acting on their subgroups by conjugation
  \item groups acting on cosets by multiplication
  \item automorphism groups acting on groups.
  \end{itemize}
  
\end{frame}

%%====================================================================

\begin{frame}{Actions vs.\ configurations}

  The group $D_4$ can be thought of as the 8 \Alert{symmetries} of the
  square:\quad     
  \begin{tikzpicture}[scale=.75,baseline=2ex]
    \path[fill=actRed] (0,.5) rectangle ++(.5,.5); 
    \path[fill=actYellow] (.5,.5) rectangle ++(.5,.5);
    \path[fill=actGreen] (0,0) rectangle ++(.5,.5);
    \path[fill=actBlue] (.5,0) rectangle ++(.5,.5);
    \draw (0,0) rectangle (1,1);
    \draw (.25,.75) node{$1$}; \draw (.75,.75) node{$2$};
    \draw (.25,.25) node{$4$}; \draw (.75,.25) node{$3$};
  \end{tikzpicture}

  \bigskip\Pause

  There is a subtle but \emph{important} distinction to make, between
  the actual 8 \Alert{symmetries} of the square, and the 8
  \Balert{configurations}. 
  
  \medskip\Pause

  For example, the 8 \Alert{symmetries} (alternatively, ``actions'')
  can be thought of as
  \[
  1,\qquad r,\qquad r^2,\qquad r^3,\qquad f,\qquad rf,\qquad r^2f,\qquad r^3f\,.
  \]
  \Pause The 8 \Balert{configurations} (or \emph{states}) of the
  square are the following:
  %%
  %% The 8 configurations of colored square w/ 1,2,3,4 labels
  \[
    \begin{tikzpicture}[scale=.75]
      \path[fill=actRed] (0,.5) rectangle ++(.5,.5); 
      \path[fill=actYellow] (.5,.5) rectangle ++(.5,.5);
      \path[fill=actGreen] (0,0) rectangle ++(.5,.5);
      \path[fill=actBlue] (.5,0) rectangle ++(.5,.5);
      \draw (.25,.75) node{$1$}; \draw (.75,.75) node{$2$};
      \draw (.25,.25) node{$4$}; \draw (.75,.25) node{$3$};
      \draw (0,0) rectangle (1,1);
    \end{tikzpicture}
    \qquad
    \begin{tikzpicture}[scale=.75]
      \path[fill=actGreen] (0,.5) rectangle ++(.5,.5); 
      \path[fill=actRed] (.5,.5) rectangle ++(.5,.5);
      \path[fill=actBlue] (0,0) rectangle ++(.5,.5);
      \path[fill=actYellow] (.5,0) rectangle ++(.5,.5);
      \draw (0,0) rectangle (1,1);
      \draw (.25,.75) node{$4$}; \draw (.75,.75) node{$1$};
        \draw (.25,.25) node{$3$}; \draw (.75,.25) node{$2$};
    \end{tikzpicture}
    \qquad
    \begin{tikzpicture}[scale=.75]
      \path[fill=actBlue] (0,.5) rectangle ++(.5,.5); 
      \path[fill=actGreen] (.5,.5) rectangle ++(.5,.5);
      \path[fill=actYellow] (0,0) rectangle ++(.5,.5);
      \path[fill=actRed] (.5,0) rectangle ++(.5,.5);
      \draw (0,0) rectangle (1,1);
      \draw (.25,.75) node{$3$}; \draw (.75,.75) node{$4$};
        \draw (.25,.25) node{$2$}; \draw (.75,.25) node{$1$};
    \end{tikzpicture}
    \qquad
    \begin{tikzpicture}[scale=.75]
      \path[fill=actYellow] (0,.5) rectangle ++(.5,.5); 
      \path[fill=actBlue] (.5,.5) rectangle ++(.5,.5);
      \path[fill=actRed] (0,0) rectangle ++(.5,.5);
      \path[fill=actGreen] (.5,0) rectangle ++(.5,.5);
      \draw (0,0) rectangle (1,1);
      \draw (.25,.75) node{$2$}; \draw (.75,.75) node{$3$};
      \draw (.25,.25) node{$1$}; \draw (.75,.25) node{$4$};
    \end{tikzpicture}
    \qquad
    \begin{tikzpicture}[scale=.75]
      \path[fill=actYellow] (0,.5) rectangle ++(.5,.5); 
      \path[fill=actRed] (.5,.5) rectangle ++(.5,.5);
      \path[fill=actBlue] (0,0) rectangle ++(.5,.5);
      \path[fill=actGreen] (.5,0) rectangle ++(.5,.5);
      \draw (0,0) rectangle (1,1);
      \draw (.25,.75) node{$2$}; \draw (.75,.75) node{$1$};
      \draw (.25,.25) node{$3$}; \draw (.75,.25) node{$4$};
    \end{tikzpicture}
    \qquad
    \begin{tikzpicture}[scale=.75]
      \path[fill=actBlue] (0,.5) rectangle ++(.5,.5); 
      \path[fill=actYellow] (.5,.5) rectangle ++(.5,.5);
      \path[fill=actGreen] (0,0) rectangle ++(.5,.5);
      \path[fill=actRed] (.5,0) rectangle ++(.5,.5);
      \draw (0,0) rectangle (1,1);
      \draw (.25,.75) node{$3$}; \draw (.75,.75) node{$2$};
        \draw (.25,.25) node{$4$}; \draw (.75,.25) node{$1$};
    \end{tikzpicture}
    \qquad
    \begin{tikzpicture}[scale=.75]
      \path[fill=actGreen] (0,.5) rectangle ++(.5,.5); 
      \path[fill=actBlue] (.5,.5) rectangle ++(.5,.5);
      \path[fill=actRed] (0,0) rectangle ++(.5,.5);
      \path[fill=actYellow] (.5,0) rectangle ++(.5,.5);
      \draw (0,0) rectangle (1,1);
      \draw (.25,.75) node{$4$}; \draw (.75,.75) node{$3$};
      \draw (.25,.25) node{$1$}; \draw (.75,.25) node{$2$};
    \end{tikzpicture}
    \qquad
    \begin{tikzpicture}[scale=.75]
      \path[fill=actRed] (0,.5) rectangle ++(.5,.5); 
      \path[fill=actGreen] (.5,.5) rectangle ++(.5,.5);
      \path[fill=actYellow] (0,0) rectangle ++(.5,.5);
      \path[fill=actBlue] (.5,0) rectangle ++(.5,.5);
      \draw (0,0) rectangle (1,1);
      \draw (.25,.75) node{$1$}; \draw (.75,.75) node{$4$};
      \draw (.25,.25) node{$2$}; \draw (.75,.25) node{$3$};
    \end{tikzpicture}
    \]

  \medskip\Pause

  When we were just learning about groups, we made an \Alert{action
    graph}. \smallskip

  %\Pause

  \begin{itemize}
    \item The \Balert{vertices} corresponded to the
      \Balert{states}. \smallskip\Pause
    \item The \Alert{edges} corresponded to \Alert{generators}. \smallskip\Pause
    \item The \Palert{paths} corresponded to \Palert{actions} (group
      elements).
  \end{itemize}

\end{frame}


%%====================================================================

\section{Action graphs!}

%%====================================================================

\begin{frame}{Action graph of $D_4$}
  
  Here is the \Alert{action graph} of the group
  $D_4=\<\Alert{r},\Balert{f}\>$:

  %% Cayley graph (action graph) of D_4 with configurations of the square
    \[  
    \begin{tikzpicture}
      \begin{scope}[shift={(-5,-1.25)},scale=.8]
        \tikzstyle{v} = [circle, draw, fill=lightgray,inner sep=0pt, 
          minimum size=5.75mm]
        \node at (.5,3) {\emph{``Group switchboard''}};
        \draw [midgray,fill=vYellow!50,rounded corners] (-.5,-1.5)
        rectangle ++(2,4); 
        \node at (0,2) [v] {$1$}; \node at (1,2) [v,fill=vBlue] {$f$};
        \node at (0,1) [v,fill=vRed] {$r$}; \node at (1,1) [v] {$rf$};
        \node at (0,0) [v] {$r^2$}; \node at (1,0) [v] {$r^2\!f$};
        \node at (0,-1) [v] {$r^3$}; \node at (1,-1) [v] {$r^3\!f$};
      \end{scope}
      %%
      \begin{scope}[scale=.65]
      \path[fill=actRed] (0,.5) rectangle ++(.5,.5); 
      \path[fill=actYellow] (.5,.5) rectangle ++(.5,.5);
      \path[fill=actGreen] (0,0) rectangle ++(.5,.5);
      \path[fill=actBlue] (.5,0) rectangle ++(.5,.5);
      \draw (.25,.75) node{$1$}; \draw (.75,.75) node{$2$};
      \draw (.25,.25) node{$4$}; \draw (.75,.25) node{$3$};
      \draw (0,0) rectangle (1,1);
      \begin{scope}[shift={(3,0)}]
        \path[fill=actGreen] (0,.5) rectangle ++(.5,.5); 
        \path[fill=actRed] (.5,.5) rectangle ++(.5,.5);
        \path[fill=actBlue] (0,0) rectangle ++(.5,.5);
        \path[fill=actYellow] (.5,0) rectangle ++(.5,.5);
        \draw (0,0) rectangle (1,1);
        \draw (.25,.75) node{$4$}; \draw (.75,.75) node{$1$};
        \draw (.25,.25) node{$3$}; \draw (.75,.25) node{$2$};
      \end{scope}
      \begin{scope}[shift={(3,-3)}]
        \path[fill=actBlue] (0,.5) rectangle ++(.5,.5); 
        \path[fill=actGreen] (.5,.5) rectangle ++(.5,.5);
        \path[fill=actYellow] (0,0) rectangle ++(.5,.5);
        \path[fill=actRed] (.5,0) rectangle ++(.5,.5);
        \draw (0,0) rectangle (1,1);
        \draw (.25,.75) node{$3$}; \draw (.75,.75) node{$4$};
        \draw (.25,.25) node{$2$}; \draw (.75,.25) node{$1$};
      \end{scope}
      \begin{scope}[shift={(0,-3)}]
        \path[fill=actYellow] (0,.5) rectangle ++(.5,.5); 
        \path[fill=actBlue] (.5,.5) rectangle ++(.5,.5);
        \path[fill=actRed] (0,0) rectangle ++(.5,.5);
        \path[fill=actGreen] (.5,0) rectangle ++(.5,.5);
        \draw (0,0) rectangle (1,1);
        \draw (.25,.75) node{$2$}; \draw (.75,.75) node{$3$};
        \draw (.25,.25) node{$1$}; \draw (.75,.25) node{$4$};
      \end{scope}
      \begin{scope}[shift={(-2,2)}]
        \path[fill=actYellow] (0,.5) rectangle ++(.5,.5); 
        \path[fill=actRed] (.5,.5) rectangle ++(.5,.5);
        \path[fill=actBlue] (0,0) rectangle ++(.5,.5);
        \path[fill=actGreen] (.5,0) rectangle ++(.5,.5);
        \draw (0,0) rectangle (1,1);
        \draw (.25,.75) node{$2$}; \draw (.75,.75) node{$1$};
        \draw (.25,.25) node{$3$}; \draw (.75,.25) node{$4$};
      \end{scope}
      \begin{scope}[shift={(-2,-5)}]
        \path[fill=actBlue] (0,.5) rectangle ++(.5,.5); 
        \path[fill=actYellow] (.5,.5) rectangle ++(.5,.5);
        \path[fill=actGreen] (0,0) rectangle ++(.5,.5);
        \path[fill=actRed] (.5,0) rectangle ++(.5,.5);
        \draw (0,0) rectangle (1,1);
        \draw (.25,.75) node{$3$}; \draw (.75,.75) node{$2$};
        \draw (.25,.25) node{$4$}; \draw (.75,.25) node{$1$};
      \end{scope}
      \begin{scope}[shift={(5,-5)}]
        \path[fill=actGreen] (0,.5) rectangle ++(.5,.5); 
        \path[fill=actBlue] (.5,.5) rectangle ++(.5,.5);
        \path[fill=actRed] (0,0) rectangle ++(.5,.5);
        \path[fill=actYellow] (.5,0) rectangle ++(.5,.5);
        \draw (0,0) rectangle (1,1);
        \draw (.25,.75) node{$4$}; \draw (.75,.75) node{$3$};
        \draw (.25,.25) node{$1$}; \draw (.75,.25) node{$2$};
      \end{scope}
      \begin{scope}[shift={(5,2)}]
        \path[fill=actRed] (0,.5) rectangle ++(.5,.5); 
        \path[fill=actGreen] (.5,.5) rectangle ++(.5,.5);
        \path[fill=actYellow] (0,0) rectangle ++(.5,.5);
        \path[fill=actBlue] (.5,0) rectangle ++(.5,.5);
        \draw (0,0) rectangle (1,1);
        \draw (.25,.75) node{$1$}; \draw (.75,.75) node{$4$};
        \draw (.25,.25) node{$2$}; \draw (.75,.25) node{$3$};
      \end{scope}
      \draw [r] (1,.5) to (3,.5);
      \draw [r] (3.5,0) to (3.5,-2);
      \draw [r] (3,-2.5) to (1,-2.5);
      \draw [r] (.5,-2) to (.5,0);
      \draw [r] (-1.5,2) to (-1.5,-4);
      \draw [r] (-1,-4.5) to (5,-4.5);
      \draw [r] (5.5,-4) to (5.5,2);
      \draw [r] (5,2.5) to (-1,2.5);
      \draw [bb] (0,1) to (-1,2);
      \draw [bb] (4,1) to (5,2);
      \draw [bb] (4,-3) to (5,-4);
      \draw [bb] (0,-3) to (-1,-4);
      \end{scope}
    \end{tikzpicture}
    \]
  
  %\vspace{-1mm}\Pause
  
  In the beginning of this course, we picked a configuration to be the
  ``solved state,'' and this gave us a \emph{bijection} between \Balert{
    configurations} and \Alert{actions} (group elements). 

  \medskip\Pause 

  The resulting graph was a Cayley graph.
  

\end{frame}


%%====================================================================
\begin{frame}{Action graphs}

  In all of the examples we saw in the beginning of the course, we had
  a bijective correspondence between actions and configurations. \emph{This
    need not always happen!}

  \medskip\Pause

  Suppose we have a size-$7$ set consisting of the following ``binary
  squares.''

  %% Set of 7 binary squares
  \[
  \begin{tikzpicture}[scale=.65]
    \node at (-1,.5) {$S=\Bigg\{$};
    \node at (13.5,.5) {$\Bigg\}$};
    \node at (1.5,.25) {,};
    \node at (3.5,.25) {,};
    \node at (5.5,.25) {,};
    \node at (7.5,.25) {,};
    \node at (9.5,.25) {,};
    \node at (11.5,.25) {,};
    %%
    \begin{scope}[shift={(0,0)}]
      \path[fill=actOrange] (0,.5) rectangle ++(.5,.5); 
      \path[fill=actOrange] (.5,.5) rectangle ++(.5,.5);
      \path[fill=actOrange] (0,0) rectangle ++(.5,.5);
      \path[fill=actOrange] (.5,0) rectangle ++(.5,.5);
      \draw (.25,.75) node{$0$}; \draw (.75,.75) node{$0$};
      \draw (.25,.25) node{$0$}; \draw (.75,.25) node{$0$};
      \draw (0,0) rectangle (1,1);
    \end{scope}
    %%
    \begin{scope}[shift={(2,0)}]
      \path[fill=actOrange] (0,.5) rectangle ++(.5,.5); 
      \path[fill=actPurple] (.5,.5) rectangle ++(.5,.5);
      \path[fill=actPurple] (0,0) rectangle ++(.5,.5);
      \path[fill=actOrange] (.5,0) rectangle ++(.5,.5);
      \draw (0,0) rectangle (1,1);
      \draw (.25,.75) node{$0$}; \draw (.75,.75) node{$1$};
      \draw (.25,.25) node{$1$}; \draw (.75,.25) node{$0$};
    \end{scope}
    \begin{scope}[shift={(4,0)}]
      \path[fill=actPurple] (0,.5) rectangle ++(.5,.5); 
      \path[fill=actOrange] (.5,.5) rectangle ++(.5,.5);
      \path[fill=actOrange] (0,0) rectangle ++(.5,.5);
      \path[fill=actPurple] (.5,0) rectangle ++(.5,.5);
      \draw (0,0) rectangle (1,1);
      \draw (.25,.75) node{$1$}; \draw (.75,.75) node{$0$};
      \draw (.25,.25) node{$0$}; \draw (.75,.25) node{$1$};
    \end{scope}
    \begin{scope}[shift={(6,0)}]
      \path[fill=actPurple] (0,.5) rectangle ++(.5,.5); 
      \path[fill=actPurple] (.5,.5) rectangle ++(.5,.5);
      \path[fill=actOrange] (0,0) rectangle ++(.5,.5);
      \path[fill=actOrange] (.5,0) rectangle ++(.5,.5);
      \draw (0,0) rectangle (1,1);
      \draw (.25,.75) node{$1$}; \draw (.75,.75) node{$1$};
      \draw (.25,.25) node{$0$}; \draw (.75,.25) node{$0$};
    \end{scope}
    \begin{scope}[shift={(8,0)}]
      \path[fill=actOrange] (0,.5) rectangle ++(.5,.5); 
      \path[fill=actPurple] (.5,.5) rectangle ++(.5,.5);
      \path[fill=actOrange] (0,0) rectangle ++(.5,.5);
      \path[fill=actPurple] (.5,0) rectangle ++(.5,.5);
      \draw (0,0) rectangle (1,1);
      \draw (.25,.75) node{$0$}; \draw (.75,.75) node{$1$};
      \draw (.25,.25) node{$0$}; \draw (.75,.25) node{$1$};
    \end{scope}
    \begin{scope}[shift={(10,0)}]
      \path[fill=actOrange] (0,.5) rectangle ++(.5,.5); 
      \path[fill=actOrange] (.5,.5) rectangle ++(.5,.5);
      \path[fill=actPurple] (0,0) rectangle ++(.5,.5);
      \path[fill=actPurple] (.5,0) rectangle ++(.5,.5);
      \draw (0,0) rectangle (1,1);
      \draw (.25,.75) node{$0$}; \draw (.75,.75) node{$0$};
        \draw (.25,.25) node{$1$}; \draw (.75,.25) node{$1$};
    \end{scope}
    \begin{scope}[shift={(12,0)}]
      \path[fill=actPurple] (0,.5) rectangle ++(.5,.5); 
      \path[fill=actOrange] (.5,.5) rectangle ++(.5,.5);
      \path[fill=actPurple] (0,0) rectangle ++(.5,.5);
      \path[fill=actOrange] (.5,0) rectangle ++(.5,.5);
      \draw (0,0) rectangle (1,1);
      \draw (.25,.75) node{$1$}; \draw (.75,.75) node{$0$};
      \draw (.25,.25) node{$1$}; \draw (.75,.25) node{$0$};
    \end{scope}
  \end{tikzpicture}
  \]

  \Pause%\vspace{-2mm}

  Let's see what happens to these binary squares when we push different buttons on the $D_4$ ``group switchboard.''

  \[
  \begin{tikzpicture}[scale=.65]
      \tikzstyle{v} = [circle, draw, fill=lightgray,inner sep=0pt, 
        minimum size=5.5mm]
      \node at (.5,4) {\small \emph{``Group switchboard''}};
      \draw [midgray,fill=vYellow!50,rounded corners] (-.5,-.5)
      rectangle ++(2,4); 
      \node at (0,3) [v] {$1$}; \node at (1,3) [v,fill=vBlue] {$f$};
      \node at (0,2) [v,fill=vRed] {$r$}; \node at (1,2) [v] {$rf$};
      \node at (0,1) [v] {$r^2$}; \node at (1,1) [v] {$r^2\!f$};
      \node at (0,0) [v] {$r^3$}; \node at (1,0) [v] {$r^3\!f$};
  \end{tikzpicture}
  \]

\end{frame}

%%====================================================================

\begin{frame}{Action graphs}

  In all of the examples we saw in the beginning of the course, we had
  a bijective correspondence between actions and states. \emph{This
    need not always happen!}

  \medskip

  Suppose we have a size-$7$ set consisting of the following ``binary
  squares.''

  %% Set of 7 binary squares
  \[
  \begin{tikzpicture}[scale=.65]
    \node at (-1,.5) {$S=\Bigg\{$};
    \node at (13.5,.5) {$\Bigg\}$};
    \node at (1.5,.25) {,};
    \node at (3.5,.25) {,};
    \node at (5.5,.25) {,};
    \node at (7.5,.25) {,};
    \node at (9.5,.25) {,};
    \node at (11.5,.25) {,};
    %%
    \begin{scope}[shift={(0,0)}]
      \path[fill=actOrange] (0,.5) rectangle ++(.5,.5); 
      \path[fill=actOrange] (.5,.5) rectangle ++(.5,.5);
      \path[fill=actOrange] (0,0) rectangle ++(.5,.5);
      \path[fill=actOrange] (.5,0) rectangle ++(.5,.5);
      \draw (.25,.75) node{$0$}; \draw (.75,.75) node{$0$};
      \draw (.25,.25) node{$0$}; \draw (.75,.25) node{$0$};
      \draw (0,0) rectangle (1,1);
    \end{scope}
    %%
    \begin{scope}[shift={(2,0)}]
      \path[fill=actOrange] (0,.5) rectangle ++(.5,.5); 
      \path[fill=actPurple] (.5,.5) rectangle ++(.5,.5);
      \path[fill=actPurple] (0,0) rectangle ++(.5,.5);
      \path[fill=actOrange] (.5,0) rectangle ++(.5,.5);
      \draw (0,0) rectangle (1,1);
      \draw (.25,.75) node{$0$}; \draw (.75,.75) node{$1$};
      \draw (.25,.25) node{$1$}; \draw (.75,.25) node{$0$};
    \end{scope}
    \begin{scope}[shift={(4,0)}]
      \path[fill=actPurple] (0,.5) rectangle ++(.5,.5); 
      \path[fill=actOrange] (.5,.5) rectangle ++(.5,.5);
      \path[fill=actOrange] (0,0) rectangle ++(.5,.5);
      \path[fill=actPurple] (.5,0) rectangle ++(.5,.5);
      \draw (0,0) rectangle (1,1);
      \draw (.25,.75) node{$1$}; \draw (.75,.75) node{$0$};
      \draw (.25,.25) node{$0$}; \draw (.75,.25) node{$1$};
    \end{scope}
    \begin{scope}[shift={(6,0)}]
      \path[fill=actPurple] (0,.5) rectangle ++(.5,.5); 
      \path[fill=actPurple] (.5,.5) rectangle ++(.5,.5);
      \path[fill=actOrange] (0,0) rectangle ++(.5,.5);
      \path[fill=actOrange] (.5,0) rectangle ++(.5,.5);
      \draw (0,0) rectangle (1,1);
      \draw (.25,.75) node{$1$}; \draw (.75,.75) node{$1$};
      \draw (.25,.25) node{$0$}; \draw (.75,.25) node{$0$};
    \end{scope}
    \begin{scope}[shift={(8,0)}]
      \path[fill=actOrange] (0,.5) rectangle ++(.5,.5); 
      \path[fill=actPurple] (.5,.5) rectangle ++(.5,.5);
      \path[fill=actOrange] (0,0) rectangle ++(.5,.5);
      \path[fill=actPurple] (.5,0) rectangle ++(.5,.5);
      \draw (0,0) rectangle (1,1);
      \draw (.25,.75) node{$0$}; \draw (.75,.75) node{$1$};
      \draw (.25,.25) node{$0$}; \draw (.75,.25) node{$1$};
    \end{scope}
    \begin{scope}[shift={(10,0)}]
      \path[fill=actOrange] (0,.5) rectangle ++(.5,.5); 
      \path[fill=actOrange] (.5,.5) rectangle ++(.5,.5);
      \path[fill=actPurple] (0,0) rectangle ++(.5,.5);
      \path[fill=actPurple] (.5,0) rectangle ++(.5,.5);
      \draw (0,0) rectangle (1,1);
      \draw (.25,.75) node{$0$}; \draw (.75,.75) node{$0$};
        \draw (.25,.25) node{$1$}; \draw (.75,.25) node{$1$};
    \end{scope}
    \begin{scope}[shift={(12,0)}]
      \path[fill=actPurple] (0,.5) rectangle ++(.5,.5); 
      \path[fill=actOrange] (.5,.5) rectangle ++(.5,.5);
      \path[fill=actPurple] (0,0) rectangle ++(.5,.5);
      \path[fill=actOrange] (.5,0) rectangle ++(.5,.5);
      \draw (0,0) rectangle (1,1);
      \draw (.25,.75) node{$1$}; \draw (.75,.75) node{$0$};
      \draw (.25,.25) node{$1$}; \draw (.75,.25) node{$0$};
    \end{scope}
  \end{tikzpicture}
  \]

  The group $D_4=\<\Alert{r},\Balert{f}\>$ ``acts on $S$'' as follows:
  %%
  %%
  %% Action graph of D_4 acting on our 7 binary squares
  \[
  \begin{tikzpicture}[scale=.65]
    %%
    %% Group switchboard
    %%
    \begin{scope}[shift={(-4,.5)},scale=1]
      \tikzstyle{v} = [circle, draw, fill=lightgray,inner sep=0pt, 
        minimum size=5.5mm]
      \node at (.5,4) {\small \emph{``Group switchboard''}};
      \draw [midgray,fill=vYellow!50,rounded corners] (-.5,-.5)
      rectangle ++(2,4); 
      \node at (0,3) [v] {$1$}; \node at (1,3) [v,fill=vBlue] {$f$};
      \node at (0,2) [v,fill=vRed] {$r$}; \node at (1,2) [v] {$rf$};
      \node at (0,1) [v] {$r^2$}; \node at (1,1) [v] {$r^2\!f$};
      \node at (0,0) [v] {$r^3$}; \node at (1,0) [v] {$r^3\!f$};
    \end{scope}
    %%
    %% Size-1 component
    %%
    \begin{scope}[shift={(0,0)}]
      \path[fill=actOrange] (0,.5) rectangle ++(.5,.5); 
      \path[fill=actOrange] (.5,.5) rectangle ++(.5,.5);
      \path[fill=actOrange] (0,0) rectangle ++(.5,.5);
      \path[fill=actOrange] (.5,0) rectangle ++(.5,.5);
      \draw (.25,.75) node{$0$}; \draw (.75,.75) node{$0$};
      \draw (.25,.25) node{$0$}; \draw (.75,.25) node{$0$};
      \draw (0,0) rectangle (1,1);
      \Loop[dist=1.5cm,dir=NO,color=eRed](0,1);
      \Loop[dist=1.5cm,dir=NO,color=eBlue](1,1);
    \end{scope}
    %%
    %% Size-2 component
    %%
    \begin{scope}[shift={(3.25,0)}]
      \begin{scope}[shift={(0,3)}]
        \path[fill=actOrange] (0,.5) rectangle ++(.5,.5); 
        \path[fill=actPurple] (.5,.5) rectangle ++(.5,.5);
        \path[fill=actPurple] (0,0) rectangle ++(.5,.5);
        \path[fill=actOrange] (.5,0) rectangle ++(.5,.5);
        \draw (0,0) rectangle (1,1);
        \draw (.25,.75) node{$0$}; \draw (.75,.75) node{$1$};
        \draw (.25,.25) node{$1$}; \draw (.75,.25) node{$0$};
        \draw [rr] (.25,0) to (.25,-2);
        \draw [bb] (.75,0) to (.75,-2);
      \end{scope}
      \begin{scope}[shift={(0,0)}]
        \path[fill=actPurple] (0,.5) rectangle ++(.5,.5); 
        \path[fill=actOrange] (.5,.5) rectangle ++(.5,.5);
        \path[fill=actOrange] (0,0) rectangle ++(.5,.5);
        \path[fill=actPurple] (.5,0) rectangle ++(.5,.5);
        \draw (0,0) rectangle (1,1);
        \draw (.25,.75) node{$1$}; \draw (.75,.75) node{$0$};
        \draw (.25,.25) node{$0$}; \draw (.75,.25) node{$1$};
      \end{scope}
    \end{scope}
    %%
    %% Size-4 component
    %%
    \begin{scope}[shift={(7,0)}]
    \begin{scope}[shift={(0,3)}]
      \path[fill=actOrange] (0,.5) rectangle ++(.5,.5); 
      \path[fill=actOrange] (.5,.5) rectangle ++(.5,.5);
      \path[fill=actPurple] (0,0) rectangle ++(.5,.5);
      \path[fill=actPurple] (.5,0) rectangle ++(.5,.5);
      \draw (0,0) rectangle (1,1);
      \draw (.25,.75) node{$0$}; \draw (.75,.75) node{$0$};
      \draw (.25,.25) node{$1$}; \draw (.75,.25) node{$1$};
      \draw [r] (3,.5) to (1,.5);
      \Loop[dist=1.5cm,dir=WE,color=eBlue](0,.5);
    \end{scope}
    %%
    \begin{scope}[shift={(0,0)}]
      \path[fill=actOrange] (0,.5) rectangle ++(.5,.5); 
      \path[fill=actPurple] (.5,.5) rectangle ++(.5,.5);
      \path[fill=actOrange] (0,0) rectangle ++(.5,.5);
      \path[fill=actPurple] (.5,0) rectangle ++(.5,.5);
      \draw (0,0) rectangle (1,1);
      \draw (.25,.75) node{$0$}; \draw (.75,.75) node{$1$};
      \draw (.25,.25) node{$0$}; \draw (.75,.25) node{$1$};
      \draw [r] (.5,3) to (.5,1);
      \draw [bb] (1,1) to (3,3);
    \end{scope}
    %%
    \begin{scope}[shift={(3,3)}]
      \path[fill=actPurple] (0,.5) rectangle ++(.5,.5); 
      \path[fill=actOrange] (.5,.5) rectangle ++(.5,.5);
      \path[fill=actPurple] (0,0) rectangle ++(.5,.5);
      \path[fill=actOrange] (.5,0) rectangle ++(.5,.5);
      \draw (0,0) rectangle (1,1);
      \draw (.25,.75) node{$1$}; \draw (.75,.75) node{$0$};
      \draw (.25,.25) node{$1$}; \draw (.75,.25) node{$0$};
      \draw [r] (.5,-2) to (.5,0);
    \end{scope}
    \begin{scope}[shift={(3,0)}]
      \path[fill=actPurple] (0,.5) rectangle ++(.5,.5); 
      \path[fill=actPurple] (.5,.5) rectangle ++(.5,.5);
      \path[fill=actOrange] (0,0) rectangle ++(.5,.5);
      \path[fill=actOrange] (.5,0) rectangle ++(.5,.5);
      \draw (0,0) rectangle (1,1);
      \draw (.25,.75) node{$1$}; \draw (.75,.75) node{$1$};
      \draw (.25,.25) node{$0$}; \draw (.75,.25) node{$0$};
      \draw [r] (-2,.5) to (0,.5);
      \Loop[dist=1.5cm,dir=EA,color=eBlue](1,.5);
    \end{scope}
    \end{scope}
  \end{tikzpicture}
  \]

  \vspace{-2mm}\Pause

  The \Alert{action graph} above has some properties of Cayley
  graphs, but there are some fundamental differences as well.

\end{frame}

%%====================================================================

\begin{frame}{The ``group switchboard'' analogy} %\Pause

  Suppose we have a ``switchboard'' for $G$, with every element $g\in
  G$ having a ``button.''

  \bigskip\Pause 

  If $a\in G$, then pressing the $a$-button rearranges the
  objects in $S$---it is a \Alert{permutation}
  of $S$; call it $\phi(a)$.

  \bigskip\Pause 

  If $b\in G$, then pressing the $b$-button also rearranges the objects in
  $S$. Call this permutation $\phi(b)$.

  \bigskip\Pause 

  The element $ab\in G$ also has a button. We require that
  \Balert{pressing the $ab$-button does the same as
    pressing the $a$-button, followed by the $b$-button}. \Pause That
  is,
  \[
  \phi(ab)=\phi(a)\phi(b)\,,\qquad\text{for all } a,b\in G\,.
  \]
  \Pause Let $\Perm(S)$ be the group of permutations of $S$. Thus, if
  $|S|=n$, then $\Perm(S)\cong S_n$. (We typically think of
  $S_n$ as the permutations of $\{1,2,\dots,n\}$.)
  
  \medskip\Pause 
  
  \begin{block}{Definition}
    A group $G$ \Alert{acts on} a set $S$ if there is a homomorphism
    $\phi\colon G\to\Perm(S)$.
  \end{block}

\end{frame}

%%====================================================================

\begin{frame}{Action graphs and $G$-sets}
  %\Pause
  
  \begin{block}{Definition}
    A set $S$ with an action by $G$ is called a (right)
    \Alert{$G$-set}.
  \end{block}
  
  \begin{alertblock}{Big ideas}
   \begin{itemize}
    \item An action $\phi\colon G\to\Perm(S)$ endows $S$ with an
      \textbf{algebraic structure}.
    \item \emph{\Alert{Action graphs are to $G$-sets}, like how
      \Balert{Cayley graphs are to groups}.}
    \end{itemize}
  \end{alertblock}

  \vspace{-3mm}

  %% Action graph of D_4 = <r,f> acting on our 7 binary squares,
  %% labeled w/ stabilizers
  \[
  \scalebox{.85}{
  \begin{tikzpicture}[scale=.65,shorten >= -2pt, shorten <= -2pt]
    %%
    %% Group switchboard
    %%
    \begin{scope}[shift={(-9.25,-1.5)},scale=1]
      \tikzstyle{v} = [circle, draw, fill=lightgray,inner sep=0pt, 
        minimum size=5.4mm]
      \node at (.5,4) {\small \emph{``Group switchboard''}};
      \draw [midgray,fill=vYellow!50,rounded corners] (-.5,-.5)
      rectangle ++(2,4); 
      \node at (0,3) [v] {$1$}; \node at (1,3) [v,fill=vBlue] {$f$};
      \node at (0,2) [v,fill=vRed] {$r$}; \node at (1,2) [v,fill=vGreen] {$rf$};
      \node at (0,1) [v] {$r^2$}; \node at (1,1) [v] {$r^2\!f$};
      \node at (0,0) [v] {$r^3$}; \node at (1,0) [v] {$r^3\!f$};
    \end{scope}
    %%
    %% the size-1 orbit
    %%
    \begin{scope}[shift={(-5,0)},shorten >= 0pt, shorten <= 0pt]  
        \path[fill=actOrange] (-.5,0) rectangle ++(.5,.5); 
        \path[fill=actOrange] (0,0) rectangle ++(.5,.5);
        \path[fill=actOrange] (-.5,-.5) rectangle ++(.5,.5);
        \path[fill=actOrange] (0,-.5) rectangle ++(.5,.5);
        \draw (-.5,-.5) rectangle (.5,.5);
        \draw (-.25,.25) node{$0$}; \draw (.25,.25) node{$0$};
        \draw (-.25,-.25) node{$0$}; \draw (.25,-.25) node{$0$};
      \draw (-.5,-.5) rectangle (.5,.5);
      \Loop[dist=1.5cm,dir=NO,color=eRed](-.5,.5);
      \Loop[dist=1.5cm,dir=NO,color=eBlue](.5,.5);
      \node at (0,-1.2) {\normalsize $\bm{D_4\!=\!\<\Alert{r},\Balert{f}\>}$};
    \end{scope}
      %%
      %% the size-2 orbit
      %%
    \begin{scope}[shift={(0,0)},shorten >= -2pt, shorten <= -2pt] 
      \begin{scope}[shift={(1.75,0)}]  %% H
        \path[fill=actPurple] (-.5,0) rectangle ++(.5,.5); 
        \path[fill=actOrange] (0,0) rectangle ++(.5,.5);
        \path[fill=actOrange] (-.5,-.5) rectangle ++(.5,.5);
        \path[fill=actPurple] (0,-.5) rectangle ++(.5,.5);
        \draw (-.5,-.5) rectangle (.5,.5);
        \draw (-.25,.25) node{$1$}; \draw (.25,.25) node{$0$};
        \draw (-.25,-.25) node{$0$}; \draw (.25,-.25) node{$1$};
        \node (0-out) at (-.5,.1) {};
        \node (0-in) at (-.5,-.1) {};
        \node at (0,-1.2) {\normalsize $\bm{\<r^2,rf\>}$};
      \end{scope}
      %%
      \begin{scope}[shift={(-1.75,0)}] %% Hr^3
        \path[fill=actOrange] (-.5,0) rectangle ++(.5,.5); 
        \path[fill=actPurple] (0,0) rectangle ++(.5,.5);
        \path[fill=actPurple] (-.5,-.5) rectangle ++(.5,.5);
        \path[fill=actOrange] (0,-.5) rectangle ++(.5,.5);
        \draw (-.5,-.5) rectangle (.5,.5);
        \draw (-.25,.25) node{$0$}; \draw (.25,.25) node{$1$};
        \draw (-.25,-.25) node{$1$}; \draw (.25,-.25) node{$0$};
        \node (180-out) at (.5,-.1) {};
        \node (180-in) at (.5,.1) {};
        \node at (0,-1.2) {\normalsize $\bm{\<r^2,rf\>}$};
      \end{scope}
     \draw [rr] (0-out) to [bend right=15] (180-in);
     \draw [bb] (180-out) to [bend right=15] (0-in);
    \end{scope} %% end of the size-2 orbit
  %%
  %% the size-4 orbit
  %%
    \begin{scope}[shift={(7,0)},shorten >= 0pt, shorten <= 0pt]  
      \begin{scope}[shift={(1.75,0)}]  %% H
        \path[fill=actOrange] (-.5,0) rectangle ++(.5,.5); 
        \path[fill=actOrange] (0,0) rectangle ++(.5,.5);
        \path[fill=actPurple] (-.5,-.5) rectangle ++(.5,.5);
        \path[fill=actPurple] (0,-.5) rectangle ++(.5,.5);
        \draw (-.5,-.5) rectangle (.5,.5);
        \draw (-.25,.25) node{$0$}; \draw (.25,.25) node{$0$};
        \draw (-.25,-.25) node{$1$}; \draw (.25,-.25) node{$1$};
        \node (0-out) at (0,.5) {};
        \node (0-in) at (0,-.5) {};
        \Loop[dist=1.5cm,dir=EA,color=eBlue](.5,0);
        \node at (.9,.9) {\normalsize $\bm{\<f\>}$};
      \end{scope}
      %%
      \begin{scope}[shift={(0,1.75)}] %% Hr
        \path[fill=actOrange] (-.5,0) rectangle ++(.5,.5); 
        \path[fill=actPurple] (0,0) rectangle ++(.5,.5);
        \path[fill=actOrange] (-.5,-.5) rectangle ++(.5,.5);
        \path[fill=actPurple] (0,-.5) rectangle ++(.5,.5);
        \draw (-.5,-.5) rectangle (.5,.5);
        \draw (-.25,.25) node{$0$}; \draw (.25,.25) node{$1$};
        \draw (-.25,-.25) node{$0$}; \draw (.25,-.25) node{$1$};
        \node (90-out) at (-.5,0) {};
        \node (90-in) at (.5,0) {};
        \node (90-bb) at (0,-.5) {};
        \node at (-1.3,.5) {\normalsize $\bm{\<r^2f\>}$};
      \end{scope}
      %%
      \begin{scope}[shift={(-1.75,0)}] %% Hr^2
        \path[fill=actPurple] (-.5,0) rectangle ++(.5,.5); 
        \path[fill=actPurple] (0,0) rectangle ++(.5,.5);
        \path[fill=actOrange] (-.5,-.5) rectangle ++(.5,.5);
        \path[fill=actOrange] (0,-.5) rectangle ++(.5,.5);
        \draw (-.5,-.5) rectangle (.5,.5);
        \draw (-.25,.25) node{$1$}; \draw (.25,.25) node{$1$};
        \draw (-.25,-.25) node{$0$}; \draw (.25,-.25) node{$0$};
        \Loop[dist=1.5cm,dir=WE,color=eBlue](-.5,0);
        \node (180-out) at (0,-.5) {};
        \node (180-in) at (0,.5) {};
        \node at (-1.1,-1) {\normalsize $\bm{\<f\>}$};
      \end{scope}
      %%
      \begin{scope}[shift={(0,-1.75)}] %% Hr^3
        \path[fill=actPurple] (-.5,0) rectangle ++(.5,.5); 
        \path[fill=actOrange] (0,0) rectangle ++(.5,.5);
        \path[fill=actPurple] (-.5,-.5) rectangle ++(.5,.5);
        \path[fill=actOrange] (0,-.5) rectangle ++(.5,.5);
        \draw (-.5,-.5) rectangle (.5,.5);
        \draw (-.25,.25) node{$1$}; \draw (.25,.25) node{$0$};
        \draw (-.25,-.25) node{$1$}; \draw (.25,-.25) node{$0$};        
        \node (270-out) at (.5,0) {};
        \node (270-in) at (-.5,0) {};
        \node (270-bb) at (0,.5) {};
        \node at (1.2,-.5) {\normalsize $\bm{\<r^2f\>}$};
      \end{scope}
      \draw [r,shorten >= -2pt, shorten <= -2pt] (0-out)
      to [bend right=25] (90-in);
      \draw [r,shorten >= -2pt, shorten <= -2pt] (90-out)
      to [bend right=25] (180-in);
      \draw [r,shorten >= -2pt, shorten <= -2pt] (180-out)
      to [bend right=25] (270-in);
      \draw [r,shorten >= -2pt, shorten <= -2pt] (270-out)
      to [bend right=25] (0-in);
     \draw [bb] (90-bb) to (270-bb);
      \end{scope} %% end of the size-4 orbit
  \end{tikzpicture}}
  \]

  \vspace{-6mm}

  %% Action graph of D_4 = <s,t> acting on our 7 binary squares, labeled w/ stabilizers
  \[
  \hspace*{-4mm}
  \scalebox{.85}{
  \begin{tikzpicture}[scale=.65,shorten >= -2pt, shorten <= -2pt]
    %%
    %% the size-1 orbit
    %%
    \begin{scope}[shift={(-4.5,0)},shorten >= 0pt, shorten <= 0pt]  
      \path[fill=actOrange] (-.5,0) rectangle ++(.5,.5); 
      \path[fill=actOrange] (0,0) rectangle ++(.5,.5);
      \path[fill=actOrange] (-.5,-.5) rectangle ++(.5,.5);
      \path[fill=actOrange] (0,-.5) rectangle ++(.5,.5);
      \draw (-.5,-.5) rectangle (.5,.5);
      \draw (-.25,.25) node{$0$}; \draw (.25,.25) node{$0$};
      \draw (-.25,-.25) node{$0$}; \draw (.25,-.25) node{$0$};
      \draw (-.5,-.5) rectangle (.5,.5);
      \Loop[dist=1.5cm,dir=NO,color=eGreen](-.5,.5);
      \Loop[dist=1.5cm,dir=NO,color=eBlue](.5,.5);
      \node at (0,-1.2) {\normalsize $\bm{D_4\!=\!\<\Balert{s},\Galert{t}\>}$};
    \end{scope}
    %%
      %% the size-2 orbit
      %%
    \begin{scope}[shift={(0,0)}] 
      \begin{scope}[shift={(1.5,0)}]  %% H
        \path[fill=actPurple] (-.5,0) rectangle ++(.5,.5); 
        \path[fill=actOrange] (0,0) rectangle ++(.5,.5);
        \path[fill=actOrange] (-.5,-.5) rectangle ++(.5,.5);
        \path[fill=actPurple] (0,-.5) rectangle ++(.5,.5);
        \draw (-.5,-.5) rectangle (.5,.5);
        \draw (-.25,.25) node{$1$}; \draw (.25,.25) node{$0$};
        \draw (-.25,-.25) node{$0$}; \draw (.25,-.25) node{$1$};
        \node (0-out) at (-.5,.1) {};
        \node (0-in) at (-.5,-.1) {};
        \Loop[dist=1.5cm,dir=NO,color=eGreen](0,.5);
        \node at (0,-1.2) {\normalsize $\bm{\<t,sts\>}$};
      \end{scope}
      %%
      \begin{scope}[shift={(-1.5,0)}] %% Hr^3
        \path[fill=actOrange] (-.5,0) rectangle ++(.5,.5); 
        \path[fill=actPurple] (0,0) rectangle ++(.5,.5);
        \path[fill=actPurple] (-.5,-.5) rectangle ++(.5,.5);
        \path[fill=actOrange] (0,-.5) rectangle ++(.5,.5);
        \draw (-.5,-.5) rectangle (.5,.5);
        \draw (-.25,.25) node{$0$}; \draw (.25,.25) node{$1$};
        \draw (-.25,-.25) node{$1$}; \draw (.25,-.25) node{$0$};
        \node (180-out) at (.5,-.1) {};
        \node (180-in) at (.5,.1) {};
        \Loop[dist=1.5cm,dir=NO,color=eGreen](0,.5);
        \node at (0,-1.2) {\normalsize $\bm{\<t,sts\>}$};
      \end{scope}
     \draw [bb,shorten >= -2pt, shorten <= -2pt] (180-out) to (0-in);
    \end{scope} %% end of the size-2 orbit
  %%
  %% the size-4 orbit
  %%
    \begin{scope}[shift={(4.5,0)},shorten >= 0pt, shorten <= 0pt]  
      \begin{scope}[shift={(9,0)}]  %% H
        \path[fill=actOrange] (-.5,0) rectangle ++(.5,.5); 
        \path[fill=actOrange] (0,0) rectangle ++(.5,.5);
        \path[fill=actPurple] (-.5,-.5) rectangle ++(.5,.5);
        \path[fill=actPurple] (0,-.5) rectangle ++(.5,.5);
        \draw (-.5,-.5) rectangle (.5,.5);
        \draw (-.25,.25) node{$0$}; \draw (.25,.25) node{$0$};
        \draw (-.25,-.25) node{$1$}; \draw (.25,-.25) node{$1$};
        \node (4L) at (-.5,0) {};
        \Loop[dist=1.5cm,dir=NO,color=eBlue](0,.5);
        \node at (0,-1.2) {\normalsize $\bm{\<s\>}$};
      \end{scope}
      %%
      \begin{scope}[shift={(3,0)}] %% Hr
        \path[fill=actOrange] (-.5,0) rectangle ++(.5,.5); 
        \path[fill=actPurple] (0,0) rectangle ++(.5,.5);
        \path[fill=actOrange] (-.5,-.5) rectangle ++(.5,.5);
        \path[fill=actPurple] (0,-.5) rectangle ++(.5,.5);
        \draw (-.5,-.5) rectangle (.5,.5);
        \draw (-.25,.25) node{$0$}; \draw (.25,.25) node{$1$};
        \draw (-.25,-.25) node{$0$}; \draw (.25,-.25) node{$1$};
        \node (2L) at (-.5,0) {};
        \node (2R) at (.5,0) {};
        \node at (0,-1.2) {\normalsize $\bm{\<tst\>}$};
      \end{scope}
      %%
      \begin{scope}[shift={(0,0)}] %% Hr^2
        \path[fill=actPurple] (-.5,0) rectangle ++(.5,.5); 
        \path[fill=actPurple] (0,0) rectangle ++(.5,.5);
        \path[fill=actOrange] (-.5,-.5) rectangle ++(.5,.5);
        \path[fill=actOrange] (0,-.5) rectangle ++(.5,.5);
        \draw (-.5,-.5) rectangle (.5,.5);
        \draw (-.25,.25) node{$1$}; \draw (.25,.25) node{$1$};
        \draw (-.25,-.25) node{$0$}; \draw (.25,-.25) node{$0$};
        \node (1R) at (.5,0) {};
        \Loop[dist=1.5cm,dir=NO,color=eBlue](0,.5);
        \node at (0,-1.2) {\normalsize $\bm{\<s\>}$};
      \end{scope}
      %%
      \begin{scope}[shift={(6,0)}] %% Hr^3
        \path[fill=actPurple] (-.5,0) rectangle ++(.5,.5); 
        \path[fill=actOrange] (0,0) rectangle ++(.5,.5);
        \path[fill=actPurple] (-.5,-.5) rectangle ++(.5,.5);
        \path[fill=actOrange] (0,-.5) rectangle ++(.5,.5);
        \draw (-.5,-.5) rectangle (.5,.5);
        \draw (-.25,.25) node{$1$}; \draw (.25,.25) node{$0$};
        \draw (-.25,-.25) node{$1$}; \draw (.25,-.25) node{$0$};        
        \node (3L) at (-.5,0) {};
        \node (3R) at (.5,0) {};
        \node at (0,-1.2) {\normalsize $\bm{\<tst\>}$};
      \end{scope}
      \draw [gg,shorten >= -2pt, shorten <= -2pt] (1R) to (2L);
      \draw [bb,shorten >= -2pt, shorten <= -2pt] (2R) to (3L);
      \draw [gg,shorten >= -2pt, shorten <= -2pt] (3R) to (4L);
      \end{scope} %% end of the size-4 orbit
  \end{tikzpicture}}
  \]
  
\end{frame}

%%====================================================================

\begin{frame}{The ``group switchboard'' analogy} 
  
  In our binary square example, pressing the \Alert{$r$-button} and
  \Balert{$f$-button} permutes $S$ as follows:
  
  %% Permutation diagrams of \phi(r) for our binary square example
  \[
  \begin{tikzpicture}[scale=.5]
    \tikzstyle{p} = [draw,bend left=55,-stealth']
    \tikzstyle{every node}=[font=\scriptsize]
    %%
    \begin{scope}[shift={(0,0)}]
      \node at (-2.5,.5) {\normalsize {\color{xRed}$\phi(r)\;$}:};
      \path[fill=actOrange] (0,.5) rectangle ++(.5,.5); 
      \path[fill=actOrange] (.5,.5) rectangle ++(.5,.5);
      \path[fill=actOrange] (0,0) rectangle ++(.5,.5);
      \path[fill=actOrange] (.5,0) rectangle ++(.5,.5);
      \draw (.25,.75) node{$0$};\draw (.75,.75) node{$0$};
      \draw (.25,.25) node{$0$};\draw (.75,.25) node{$0$};
    \draw (0,0) rectangle (1,1);
    \end{scope}
    %%
    \begin{scope}[shift={(2.5,0)}]
      \path[fill=actOrange] (0,.5) rectangle ++(.5,.5); 
      \path[fill=actPurple] (.5,.5) rectangle ++(.5,.5);
      \path[fill=actPurple] (0,0) rectangle ++(.5,.5);
      \path[fill=actOrange] (.5,0) rectangle ++(.5,.5);
      \draw (0,0) rectangle (1,1);
      \draw(.25,.75) node{$0$};\draw(.75,.75)node{$1$};
      \draw(.25,.25) node{$1$};\draw (.75,.25)node{$0$};
      \draw [p,eRed] (.5,1) to (3,1);
    \end{scope}
    %%
    \begin{scope}[shift={(5,0)}]
      \path[fill=actPurple] (0,.5) rectangle ++(.5,.5); 
      \path[fill=actOrange] (.5,.5) rectangle ++(.5,.5);
      \path[fill=actOrange] (0,0) rectangle ++(.5,.5);
      \path[fill=actPurple] (.5,0) rectangle ++(.5,.5);
      \draw (0,0) rectangle (1,1);
      \draw(.25,.75)node{$1$};\draw (.75,.75)node{$0$};
      \draw(.25,.25)node{$0$};\draw (.75,.25)node{$1$};
      \draw [p,eRed] (.5,0) to (-2,0);
    \end{scope}
    %%
    \begin{scope}[shift={(7.5,0)}]
      \path[fill=actPurple] (0,.5) rectangle ++(.5,.5); 
      \path[fill=actPurple] (.5,.5) rectangle ++(.5,.5);
      \path[fill=actOrange] (0,0) rectangle ++(.5,.5);
      \path[fill=actOrange] (.5,0) rectangle ++(.5,.5);
      \draw (0,0) rectangle (1,1);
      \draw(.25,.75) node{$1$};\draw(.75,.75)node{$1$};
      \draw(.25,.25) node{$0$}; \draw(.75,.25)node{$0$};
      \draw [p,eRed] (.5,1) to (3,1);
    \end{scope}
    %%
    \begin{scope}[shift={(10,0)}]
      \path[fill=actPurple] (0,.5) rectangle ++(.5,.5); 
      \path[fill=actOrange] (.5,.5) rectangle ++(.5,.5);
      \path[fill=actPurple] (0,0) rectangle ++(.5,.5);
      \path[fill=actOrange] (.5,0) rectangle ++(.5,.5);
      \draw (0,0) rectangle (1,1);
      \draw(.25,.75)node{$1$};\draw(.75,.75)node{$0$};
      \draw(.25,.25)node{$1$};\draw(.75,.25) node{$0$};
      \draw [p,eRed] (.5,1) to (3,1);
    \end{scope}
    %%
    \begin{scope}[shift={(12.5,0)}]
      \path[fill=actOrange] (0,.5) rectangle ++(.5,.5); 
      \path[fill=actOrange] (.5,.5) rectangle ++(.5,.5);
      \path[fill=actPurple] (0,0) rectangle ++(.5,.5);
      \path[fill=actPurple] (.5,0) rectangle ++(.5,.5);
      \draw (0,0) rectangle (1,1);
      \draw(.25,.75)node{$0$};\draw(.75,.75) node{$0$};
      \draw(.25,.25)node{$1$};\draw(.75,.25) node{$1$};
      \draw [p,eRed] (.5,1) to (3,1);
    \end{scope}
    %%
    \begin{scope}[shift={(15,0)}]
      \path[fill=actOrange] (0,.5) rectangle ++(.5,.5); 
      \path[fill=actPurple] (.5,.5) rectangle ++(.5,.5);
      \path[fill=actOrange] (0,0) rectangle ++(.5,.5);
      \path[fill=actPurple] (.5,0) rectangle ++(.5,.5);
      \draw (0,0) rectangle (1,1);
      \draw(.25,.75)node{$0$};\draw(.75,.75) node{$1$};
      \draw(.25,.25)node{$0$};\draw(.75,.25) node{$1$};
      \draw [p,eRed,bend left=25] (.5,0) to (-7,0);
    \end{scope}
  \end{tikzpicture}
  \]

  \vspace{-6mm}\Pause

  %% Permutation diagrams of \phi(f) for our binary square example
  \[
  \begin{tikzpicture}[scale=.5]
    \tikzstyle{p} = [draw,bend left=55,-stealth]
    \tikzstyle{every node}=[font=\scriptsize]
    %%
    \begin{scope}[shift={(0,0)}]
      \node at (-2.5,.5) {\normalsize \Balert{$\phi(f)\;$}:};
      \path[fill=actOrange] (0,.5) rectangle ++(.5,.5); 
      \path[fill=actOrange] (.5,.5) rectangle ++(.5,.5);
      \path[fill=actOrange] (0,0) rectangle ++(.5,.5);
      \path[fill=actOrange] (.5,0) rectangle ++(.5,.5);
      \draw (.25,.75) node{$0$};\draw (.75,.75) node{$0$};
      \draw (.25,.25) node{$0$};\draw (.75,.25) node{$0$};
      \draw (0,0) rectangle (1,1);
    \end{scope}
    %%
    \begin{scope}[shift={(2.5,0)}]
      \path[fill=actOrange] (0,.5) rectangle ++(.5,.5); 
      \path[fill=actPurple] (.5,.5) rectangle ++(.5,.5);
      \path[fill=actPurple] (0,0) rectangle ++(.5,.5);
      \path[fill=actOrange] (.5,0) rectangle ++(.5,.5);
      \draw (0,0) rectangle (1,1);
      \draw(.25,.75) node{$0$};\draw(.75,.75)node{$1$};
      \draw(.25,.25) node{$1$};\draw (.75,.25)node{$0$};
      \draw [p,eBlue] (.5,1) to (3,1);
    \end{scope}
    %%
    \begin{scope}[shift={(5,0)}]
      \path[fill=actPurple] (0,.5) rectangle ++(.5,.5); 
      \path[fill=actOrange] (.5,.5) rectangle ++(.5,.5);
      \path[fill=actOrange] (0,0) rectangle ++(.5,.5);
      \path[fill=actPurple] (.5,0) rectangle ++(.5,.5);
      \draw (0,0) rectangle (1,1);
      \draw(.25,.75)node{$1$};\draw (.75,.75)node{$0$};
      \draw(.25,.25)node{$0$};\draw (.75,.25)node{$1$};
      \draw [p,eBlue] (.5,0) to (-2,0);
    \end{scope}
    %%
    \begin{scope}[shift={(7.5,0)}]
      \path[fill=actPurple] (0,.5) rectangle ++(.5,.5); 
      \path[fill=actPurple] (.5,.5) rectangle ++(.5,.5);
      \path[fill=actOrange] (0,0) rectangle ++(.5,.5);
      \path[fill=actOrange] (.5,0) rectangle ++(.5,.5);
      \draw (0,0) rectangle (1,1);
      \draw(.25,.75) node{$1$};\draw(.75,.75)node{$1$};
      \draw(.25,.25) node{$0$}; \draw(.75,.25)node{$0$};
    \end{scope}
    %%
    \begin{scope}[shift={(10,0)}]
     \path[fill=actPurple] (0,.5) rectangle ++(.5,.5); 
      \path[fill=actOrange] (.5,.5) rectangle ++(.5,.5);
      \path[fill=actPurple] (0,0) rectangle ++(.5,.5);
      \path[fill=actOrange] (.5,0) rectangle ++(.5,.5);
      \draw (0,0) rectangle (1,1);
      \draw(.25,.75)node{$1$};\draw(.75,.75)node{$0$};
      \draw(.25,.25)node{$1$};\draw(.75,.25) node{$0$};
      \draw [p,eBlue,bend left=40] (1,1) to (5,1);
    \end{scope}
    %%
    \begin{scope}[shift={(12.5,0)}]
      \path[fill=actOrange] (0,.5) rectangle ++(.5,.5); 
      \path[fill=actOrange] (.5,.5) rectangle ++(.5,.5);
      \path[fill=actPurple] (0,0) rectangle ++(.5,.5);
      \path[fill=actPurple] (.5,0) rectangle ++(.5,.5);
      \draw (0,0) rectangle (1,1);
      \draw(.25,.75)node{$0$};\draw(.75,.75) node{$0$};
      \draw(.25,.25)node{$1$};\draw(.75,.25) node{$1$};
    \end{scope}
    %%
    \begin{scope}[shift={(15,0)}]
      \path[fill=actOrange] (0,.5) rectangle ++(.5,.5); 
      \path[fill=actPurple] (.5,.5) rectangle ++(.5,.5);
      \path[fill=actOrange] (0,0) rectangle ++(.5,.5);
      \path[fill=actPurple] (.5,0) rectangle ++(.5,.5);
      \draw (0,0) rectangle (1,1);
      \draw(.25,.75)node{$0$};\draw(.75,.75) node{$1$};
      \draw(.25,.25)node{$0$};\draw(.75,.25) node{$1$}; 
      \draw [p,eBlue,bend left=40] (0,0) to (-4,0);
    \end{scope}
  \end{tikzpicture}
  \]

  \vspace{0mm}\Pause

  Observe how these permutations are encoded in the action
  graph. (Next to each $s\in S$ is the subgroup that fixes it.)

  \vspace{-4mm}
  
  %% Action graph of D_4 = <r,f> acting on our 7 binary squares,
  %% labeled w/ stabilizers
  \[
  \scalebox{.85}{
    \begin{tikzpicture}[scale=.65,shorten >= -2pt, shorten <= -2pt]
      %%
      %% Group switchboard
      %%
      \begin{scope}[shift={(-9.25,-1.5)},scale=1]
        \tikzstyle{v} = [circle, draw, fill=lightgray,inner sep=0pt, 
          minimum size=5.4mm]
        \node at (.5,4) {\small \emph{``Group switchboard''}};
        \draw [midgray,fill=vYellow!50,rounded corners] (-.5,-.5)
        rectangle ++(2,4); 
        \node at (0,3) [v] {$1$}; \node at (1,3) [v,fill=vBlue] {$f$};
        \node at (0,2) [v,fill=vRed] {$r$}; \node at (1,2) [v] {$rf$};
        \node at (0,1) [v] {$r^2$}; \node at (1,1) [v] {$r^2\!f$};
        \node at (0,0) [v] {$r^3$}; \node at (1,0) [v] {$r^3\!f$};
      \end{scope}
      %%
      %% the size-1 orbit
      %%
      \begin{scope}[shift={(-5,0)},shorten >= 0pt, shorten <= 0pt]  
        \path[fill=actOrange] (-.5,0) rectangle ++(.5,.5); 
        \path[fill=actOrange] (0,0) rectangle ++(.5,.5);
        \path[fill=actOrange] (-.5,-.5) rectangle ++(.5,.5);
        \path[fill=actOrange] (0,-.5) rectangle ++(.5,.5);
        \draw (-.5,-.5) rectangle (.5,.5);
        \draw (-.25,.25) node{$0$}; \draw (.25,.25) node{$0$};
        \draw (-.25,-.25) node{$0$}; \draw (.25,-.25) node{$0$};
        \draw (-.5,-.5) rectangle (.5,.5);
        \Loop[dist=1.5cm,dir=NO,color=eRed](-.5,.5);
        \Loop[dist=1.5cm,dir=NO,color=eBlue](.5,.5);
        \node at (0,-1.2) {\normalsize $\bm{D_4\!=\!\<\Alert{r},\Balert{f}\>}$};
      \end{scope}
      %%
      %% the size-2 orbit
      %%
      \begin{scope}[shift={(0,0)},shorten >= -2pt, shorten <= -2pt] 
        \begin{scope}[shift={(1.75,0)}]  %% H
          \path[fill=actPurple] (-.5,0) rectangle ++(.5,.5); 
          \path[fill=actOrange] (0,0) rectangle ++(.5,.5);
          \path[fill=actOrange] (-.5,-.5) rectangle ++(.5,.5);
          \path[fill=actPurple] (0,-.5) rectangle ++(.5,.5);
          \draw (-.5,-.5) rectangle (.5,.5);
          \draw (-.25,.25) node{$1$}; \draw (.25,.25) node{$0$};
          \draw (-.25,-.25) node{$0$}; \draw (.25,-.25) node{$1$};
          \node (0-out) at (-.5,.1) {};
          \node (0-in) at (-.5,-.1) {};
          \node at (0,-1.2) {\normalsize $\bm{\<r^2,rf\>}$};
        \end{scope}
        %%
        \begin{scope}[shift={(-1.75,0)}] %% Hr^3
          \path[fill=actOrange] (-.5,0) rectangle ++(.5,.5); 
          \path[fill=actPurple] (0,0) rectangle ++(.5,.5);
          \path[fill=actPurple] (-.5,-.5) rectangle ++(.5,.5);
          \path[fill=actOrange] (0,-.5) rectangle ++(.5,.5);
          \draw (-.5,-.5) rectangle (.5,.5);
          \draw (-.25,.25) node{$0$}; \draw (.25,.25) node{$1$};
          \draw (-.25,-.25) node{$1$}; \draw (.25,-.25) node{$0$};
          \node (180-out) at (.5,-.1) {};
          \node (180-in) at (.5,.1) {};
          \node at (0,-1.2) {\normalsize $\bm{\<r^2,rf\>}$};
        \end{scope}
        \draw [rr] (0-out) to [bend right=15] (180-in);
        \draw [bb] (180-out) to [bend right=15] (0-in);
      \end{scope} %% end of the size-2 orbit
      %%
      %% the size-4 orbit
      %%
      \begin{scope}[shift={(7,0)},shorten >= 0pt, shorten <= 0pt]  
        \begin{scope}[shift={(1.75,0)}]  %% H
          \path[fill=actOrange] (-.5,0) rectangle ++(.5,.5); 
          \path[fill=actOrange] (0,0) rectangle ++(.5,.5);
          \path[fill=actPurple] (-.5,-.5) rectangle ++(.5,.5);
          \path[fill=actPurple] (0,-.5) rectangle ++(.5,.5);
          \draw (-.5,-.5) rectangle (.5,.5);
          \draw (-.25,.25) node{$0$}; \draw (.25,.25) node{$0$};
          \draw (-.25,-.25) node{$1$}; \draw (.25,-.25) node{$1$};
          \node (0-out) at (0,.5) {};
          \node (0-in) at (0,-.5) {};
          \Loop[dist=1.5cm,dir=EA,color=eBlue](.5,0);
          \node at (.9,.9) {\normalsize $\bm{\<f\>}$};
        \end{scope}
        %%
        \begin{scope}[shift={(0,1.75)}] %% Hr
          \path[fill=actOrange] (-.5,0) rectangle ++(.5,.5); 
          \path[fill=actPurple] (0,0) rectangle ++(.5,.5);
          \path[fill=actOrange] (-.5,-.5) rectangle ++(.5,.5);
          \path[fill=actPurple] (0,-.5) rectangle ++(.5,.5);
          \draw (-.5,-.5) rectangle (.5,.5);
          \draw (-.25,.25) node{$0$}; \draw (.25,.25) node{$1$};
          \draw (-.25,-.25) node{$0$}; \draw (.25,-.25) node{$1$};
          \node (90-out) at (-.5,0) {};
          \node (90-in) at (.5,0) {};
          \node (90-bb) at (0,-.5) {};
          \node at (-1.3,.5) {\normalsize $\bm{\<r^2f\>}$};
        \end{scope}
        %%
        \begin{scope}[shift={(-1.75,0)}] %% Hr^2
          \path[fill=actPurple] (-.5,0) rectangle ++(.5,.5); 
          \path[fill=actPurple] (0,0) rectangle ++(.5,.5);
          \path[fill=actOrange] (-.5,-.5) rectangle ++(.5,.5);
          \path[fill=actOrange] (0,-.5) rectangle ++(.5,.5);
          \draw (-.5,-.5) rectangle (.5,.5);
          \draw (-.25,.25) node{$1$}; \draw (.25,.25) node{$1$};
          \draw (-.25,-.25) node{$0$}; \draw (.25,-.25) node{$0$};
          \Loop[dist=1.5cm,dir=WE,color=eBlue](-.5,0);
          \node (180-out) at (0,-.5) {};
          \node (180-in) at (0,.5) {};
          \node at (-1.1,-1) {\normalsize $\bm{\<f\>}$};
        \end{scope}
        %%
        \begin{scope}[shift={(0,-1.75)}] %% Hr^3
          \path[fill=actPurple] (-.5,0) rectangle ++(.5,.5); 
          \path[fill=actOrange] (0,0) rectangle ++(.5,.5);
          \path[fill=actPurple] (-.5,-.5) rectangle ++(.5,.5);
          \path[fill=actOrange] (0,-.5) rectangle ++(.5,.5);
          \draw (-.5,-.5) rectangle (.5,.5);
          \draw (-.25,.25) node{$1$}; \draw (.25,.25) node{$0$};
          \draw (-.25,-.25) node{$1$}; \draw (.25,-.25) node{$0$};        
          \node (270-out) at (.5,0) {};
          \node (270-in) at (-.5,0) {};
          \node (270-bb) at (0,.5) {};
          \node at (1.2,-.5) {\normalsize $\bm{\<r^2f\>}$};
        \end{scope}
        \draw [r,shorten >= -2pt, shorten <= -2pt] (0-out) to [bend right=25] (90-in);
        \draw [r,shorten >= -2pt, shorten <= -2pt] (90-out) to [bend right=25] (180-in);
        \draw [r,shorten >= -2pt, shorten <= -2pt] (180-out) to [bend right=25] (270-in);
        \draw [r,shorten >= -2pt, shorten <= -2pt] (270-out) to [bend right=25] (0-in);
        \draw [bb,shorten >= -2pt, shorten <= -2pt] (90-bb) to (270-bb);
      \end{scope} %% end of the size-4 orbit
  \end{tikzpicture}}
  \]
  
\end{frame}

%%====================================================================

\begin{frame}{The ``group switchboard'' analogy} %\Pause

  %In our binary square example, pressing the \Alert{$r$-button} and
  %\Balert{$f$-button} permutes $S$ as follows:

  This action is an embedding $\phi\colon D_4\into\Perm(S)\cong S_7$. \pause How? \pause

  On your handout, label your binary squares 1 through 7. \pause 
  
  Then write your arrows in permutation notation. \pause

  %% Permutation diagrams of for binary square example, faded and w/ 1,...,7 labels
  \[
  \begin{tikzpicture}[scale=.5]
  %%
  %% Permutation diagram of \phi(r)
  %%
  \begin{scope}[shift={(0,0)}]
    \tikzstyle{p} = [draw,bend left=55,-stealth]
    \node at (-2.5,.5) {{\color{xRed}$\phi(r)\;$}:};
%    \node at (13.5,.5) {$\Bigg\}$};
    \path[fill=actOrange] (0,.5) rectangle ++(.5,.5); 
    \path[fill=actOrange] (.5,.5) rectangle ++(.5,.5);
    \path[fill=actOrange] (0,0) rectangle ++(.5,.5);
    \path[fill=actOrange] (.5,0) rectangle ++(.5,.5);
    \draw (.25,.75) node{\scriptsize $0$};\draw (.75,.75) node{\scriptsize $0$};
    \draw (.25,.25) node{\scriptsize $0$};\draw (.75,.25) node{\scriptsize $0$};
    \draw (0,0) rectangle (1,1);
    \filldraw[fill=white,opacity=0.7] 
    (0,0)--(1,0)--(1,1)--(0,1)--cycle;
    \node at (.5,.5) {\large\bf $\mathbf{1}$};
    %%
    \begin{scope}[shift={(2.5,0)}]
      \path[fill=actOrange] (0,.5) rectangle ++(.5,.5); 
      \path[fill=actPurple] (.5,.5) rectangle ++(.5,.5);
      \path[fill=actPurple] (0,0) rectangle ++(.5,.5);
      \path[fill=actOrange] (.5,0) rectangle ++(.5,.5);
      \draw (0,0) rectangle (1,1);
      \draw(.25,.75) node{\scriptsize $0$};\draw(.75,.75)node{\scriptsize $1$};
      \draw(.25,.25) node{\scriptsize $1$};\draw (.75,.25)node{\scriptsize $0$};
      \draw [p,eRed] (.5,1) to (3,1);
      \filldraw[fill=white,opacity=0.7] 
      (0,0)--(1,0)--(1,1)--(0,1)--cycle;
      \node at (.5,.5) {\large\bf $\mathbf{2}$};
    \end{scope}
    %%
    \begin{scope}[shift={(5,0)}]
      \path[fill=actPurple] (0,.5) rectangle ++(.5,.5); 
      \path[fill=actOrange] (.5,.5) rectangle ++(.5,.5);
      \path[fill=actOrange] (0,0) rectangle ++(.5,.5);
      \path[fill=actPurple] (.5,0) rectangle ++(.5,.5);
      \draw (0,0) rectangle (1,1);
      \draw(.25,.75)node{\scriptsize $1$};\draw (.75,.75)node{\scriptsize $0$};
      \draw(.25,.25)node{\scriptsize $0$};\draw (.75,.25)node{\scriptsize $1$};
      \draw [p,eRed] (.5,0) to (-2,0);
      \filldraw[fill=white,opacity=0.7] 
      (0,0)--(1,0)--(1,1)--(0,1)--cycle;
      \node at (.5,.5) {\large\bf $\mathbf{3}$};
    \end{scope}
    %%
    \begin{scope}[shift={(7.5,0)}]
      \path[fill=actPurple] (0,.5) rectangle ++(.5,.5); 
      \path[fill=actPurple] (.5,.5) rectangle ++(.5,.5);
      \path[fill=actOrange] (0,0) rectangle ++(.5,.5);
      \path[fill=actOrange] (.5,0) rectangle ++(.5,.5);
      \draw (0,0) rectangle (1,1);
      \draw(.25,.75) node{\scriptsize $1$};\draw(.75,.75)node{\scriptsize $1$};
      \draw(.25,.25) node{\scriptsize $0$}; \draw(.75,.25)node{\scriptsize $0$};
      \draw [p,eRed] (.5,1) to (3,1);
      \filldraw[fill=white,opacity=0.7] 
      (0,0)--(1,0)--(1,1)--(0,1)--cycle;
      \node at (.5,.5) {\large\bf $\mathbf{4}$};
    \end{scope}
    %%
    \begin{scope}[shift={(10,0)}]
      \path[fill=actPurple] (0,.5) rectangle ++(.5,.5); 
      \path[fill=actOrange] (.5,.5) rectangle ++(.5,.5);
      \path[fill=actPurple] (0,0) rectangle ++(.5,.5);
      \path[fill=actOrange] (.5,0) rectangle ++(.5,.5);
      \draw (0,0) rectangle (1,1);
      \draw(.25,.75)node{\scriptsize $1$};\draw(.75,.75)node{\scriptsize $0$};
      \draw(.25,.25)node{\scriptsize $1$};\draw(.75,.25) node{\scriptsize $0$};
      \draw [p,eRed] (.5,1) to (3,1);
      \filldraw[fill=white,opacity=0.7] 
      (0,0)--(1,0)--(1,1)--(0,1)--cycle;
      \node at (.5,.5) {\large\bf $\mathbf{5}$};
    \end{scope}
    \begin{scope}[shift={(12.5,0)}]
      \path[fill=actOrange] (0,.5) rectangle ++(.5,.5); 
      \path[fill=actOrange] (.5,.5) rectangle ++(.5,.5);
      \path[fill=actPurple] (0,0) rectangle ++(.5,.5);
      \path[fill=actPurple] (.5,0) rectangle ++(.5,.5);
      \draw (0,0) rectangle (1,1);
      \draw(.25,.75)node{\scriptsize $0$};\draw(.75,.75) node{\scriptsize $0$};
      \draw(.25,.25)node{\scriptsize $1$};\draw(.75,.25) node{\scriptsize $1$};
      \draw [p,eRed] (.5,1) to (3,1);
      \filldraw[fill=white,opacity=0.7] 
      (0,0)--(1,0)--(1,1)--(0,1)--cycle;
      \node at (.5,.5) {\large\bf $\mathbf{6}$};
    \end{scope}
    \begin{scope}[shift={(15,0)}]
      \path[fill=actOrange] (0,.5) rectangle ++(.5,.5); 
      \path[fill=actPurple] (.5,.5) rectangle ++(.5,.5);
      \path[fill=actOrange] (0,0) rectangle ++(.5,.5);
      \path[fill=actPurple] (.5,0) rectangle ++(.5,.5);
      \draw (0,0) rectangle (1,1);
      \draw(.25,.75)node{\scriptsize $0$};\draw(.75,.75) node{\scriptsize $1$};
      \draw(.25,.25)node{\scriptsize $0$};\draw(.75,.25) node{\scriptsize $1$};
      \draw [p,eRed,bend left=25] (.5,0) to (-7,0);
      \filldraw[fill=white,opacity=0.7] 
      (0,0)--(1,0)--(1,1)--(0,1)--cycle;
      \node at (.5,.5) {\large\bf $\mathbf{7}$};
    \end{scope}
  \end{scope}
  %%
  %% Permutation diagrams of \phi(f) for our binary square example, faded and w/ 1,...,7 labels
  %%
  \begin{scope}[shift={(0,-3)}]
    \tikzstyle{p} = [draw,bend left=55,-stealth]
    \node at (-2.5,.5) {\Balert{$\phi(f)\;$}:};
%    \node at (13.5,.5) {$\Bigg\}$};
    \path[fill=actOrange] (0,.5) rectangle ++(.5,.5); 
    \path[fill=actOrange] (.5,.5) rectangle ++(.5,.5);
    \path[fill=actOrange] (0,0) rectangle ++(.5,.5);
    \path[fill=actOrange] (.5,0) rectangle ++(.5,.5);
    \draw (.25,.75) node{\scriptsize $0$};\draw (.75,.75) node{\scriptsize $0$};
    \draw (.25,.25) node{\scriptsize $0$};\draw (.75,.25) node{\scriptsize $0$};
    \draw (0,0) rectangle (1,1);
    \filldraw[fill=white,opacity=0.7] 
    (0,0)--(1,0)--(1,1)--(0,1)--cycle;
    \node at (.5,.5) {\large\bf $\mathbf{1}$};
    %%
    \begin{scope}[shift={(2.5,0)}]
      \path[fill=actOrange] (0,.5) rectangle ++(.5,.5); 
      \path[fill=actPurple] (.5,.5) rectangle ++(.5,.5);
      \path[fill=actPurple] (0,0) rectangle ++(.5,.5);
      \path[fill=actOrange] (.5,0) rectangle ++(.5,.5);
      \draw (0,0) rectangle (1,1);
      \draw(.25,.75) node{\scriptsize $0$};\draw(.75,.75)node{\scriptsize $1$};
      \draw(.25,.25) node{\scriptsize $1$};\draw (.75,.25)node{\scriptsize $0$};
      \draw [p,eBlue] (.5,1) to (3,1);
      \filldraw[fill=white,opacity=0.7] 
      (0,0)--(1,0)--(1,1)--(0,1)--cycle;
      \node at (.5,.5) {\large\bf $\mathbf{2}$};
    \end{scope}
    %%
    \begin{scope}[shift={(5,0)}]
      \path[fill=actPurple] (0,.5) rectangle ++(.5,.5); 
      \path[fill=actOrange] (.5,.5) rectangle ++(.5,.5);
      \path[fill=actOrange] (0,0) rectangle ++(.5,.5);
      \path[fill=actPurple] (.5,0) rectangle ++(.5,.5);
      \draw (0,0) rectangle (1,1);
      \draw(.25,.75)node{\scriptsize $1$};\draw (.75,.75)node{\scriptsize $0$};
      \draw(.25,.25)node{\scriptsize $0$};\draw (.75,.25)node{\scriptsize $1$};
      \draw [p,eBlue] (.5,0) to (-2,0);
      \filldraw[fill=white,opacity=0.7] 
      (0,0)--(1,0)--(1,1)--(0,1)--cycle;
      \node at (.5,.5) {\large\bf $\mathbf{3}$};
    \end{scope}
    %%
    \begin{scope}[shift={(7.5,0)}]
      \path[fill=actPurple] (0,.5) rectangle ++(.5,.5); 
      \path[fill=actPurple] (.5,.5) rectangle ++(.5,.5);
      \path[fill=actOrange] (0,0) rectangle ++(.5,.5);
      \path[fill=actOrange] (.5,0) rectangle ++(.5,.5);
      \draw (0,0) rectangle (1,1);
      \draw(.25,.75) node{\scriptsize $1$};\draw(.75,.75)node{\scriptsize $1$};
      \draw(.25,.25) node{\scriptsize $0$}; \draw(.75,.25)node{\scriptsize $0$};
      \filldraw[fill=white,opacity=0.7] 
      (0,0)--(1,0)--(1,1)--(0,1)--cycle;
      \node at (.5,.5) {\large\bf $\mathbf{4}$};
    \end{scope}
    %%
    \begin{scope}[shift={(10,0)}]
     \path[fill=actPurple] (0,.5) rectangle ++(.5,.5); 
      \path[fill=actOrange] (.5,.5) rectangle ++(.5,.5);
      \path[fill=actPurple] (0,0) rectangle ++(.5,.5);
      \path[fill=actOrange] (.5,0) rectangle ++(.5,.5);
      \draw (0,0) rectangle (1,1);
      \draw(.25,.75)node{\scriptsize $1$};\draw(.75,.75)node{\scriptsize $0$};
      \draw(.25,.25)node{\scriptsize $1$};\draw(.75,.25) node{\scriptsize $0$};
      \draw [p,eBlue,bend left=40] (1,1) to (5,1);
      \filldraw[fill=white,opacity=0.7] 
      (0,0)--(1,0)--(1,1)--(0,1)--cycle;
      \node at (.5,.5) {\large\bf $\mathbf{5}$};
    \end{scope}
    %%
    \begin{scope}[shift={(12.5,0)}]
      \path[fill=actOrange] (0,.5) rectangle ++(.5,.5); 
      \path[fill=actOrange] (.5,.5) rectangle ++(.5,.5);
      \path[fill=actPurple] (0,0) rectangle ++(.5,.5);
      \path[fill=actPurple] (.5,0) rectangle ++(.5,.5);
      \draw (0,0) rectangle (1,1);
      \draw(.25,.75)node{\scriptsize $0$};\draw(.75,.75) node{\scriptsize $0$};
      \draw(.25,.25)node{\scriptsize $1$};\draw(.75,.25) node{\scriptsize $1$};
      \filldraw[fill=white,opacity=0.7] 
      (0,0)--(1,0)--(1,1)--(0,1)--cycle;
      \node at (.5,.5) {\large\bf $\mathbf{6}$};
    \end{scope}
    %%
    \begin{scope}[shift={(15,0)}]
      \path[fill=actOrange] (0,.5) rectangle ++(.5,.5); 
      \path[fill=actPurple] (.5,.5) rectangle ++(.5,.5);
      \path[fill=actOrange] (0,0) rectangle ++(.5,.5);
      \path[fill=actPurple] (.5,0) rectangle ++(.5,.5);
      \draw (0,0) rectangle (1,1);
      \draw(.25,.75)node{\scriptsize $0$};\draw(.75,.75) node{\scriptsize $1$};
      \draw(.25,.25)node{\scriptsize $0$};\draw(.75,.25) node{\scriptsize $1$}; 
      \draw [p,eBlue,bend left=40] (0,0) to (-4,0);
      \filldraw[fill=white,opacity=0.7] 
      (0,0)--(1,0)--(1,1)--(0,1)--cycle;
      \node at (.5,.5) {\large\bf $\mathbf{7}$};
    \end{scope}
    \end{scope}
  \end{tikzpicture}
  \]

  \vspace{0mm}\Pause
  
  %Only $8$ of the $|S_7|=5040$ permutations arise:
  %this action:

  Notice that
  $\Image(\phi)=\big\<\Alert{(23)(4567)},\Balert{(23)(57)}\big\>\cong
  D_4\leq S_7$.
  
  
  \vspace{-2mm}

  %% Action graph of D_4 = <r,f> acting on our 7 binary squares, labeled w/ stabilizers
  \[
  \hspace*{-2mm}
  \scalebox{.9}{
  \begin{tikzpicture}[scale=.65,shorten >= -2pt, shorten <= -2pt]
    %%
    %% Group switchboard
    %%
    \begin{scope}[shift={(-8.75,-1.5)},scale=1]
      \tikzstyle{v} = [circle, draw, fill=lightgray,inner sep=0pt, 
        minimum size=5.4mm]
      \node at (.5,4) {\small \emph{``Group switchboard''}};
      \draw [midgray,fill=vYellow!50,rounded corners] (-.5,-.5)
      rectangle ++(2,4); 
      \node at (0,3) [v] {$1$}; \node at (1,3) [v,fill=vBlue] {$f$};
      \node at (0,2) [v,fill=vRed] {$r$}; \node at (1,2) [v] {$rf$};
      \node at (0,1) [v] {$r^2$}; \node at (1,1) [v] {$r^2\!f$};
      \node at (0,0) [v] {$r^3$}; \node at (1,0) [v] {$r^3\!f$};
    \end{scope}
    %%
    %% the size-1 orbit
    %%
    \begin{scope}[shift={(-4.5,0)},shorten >= 0pt, shorten <= 0pt]  
      \path[fill=actOrange] (-.5,0) rectangle ++(.5,.5); 
      \path[fill=actOrange] (0,0) rectangle ++(.5,.5);
      \path[fill=actOrange] (-.5,-.5) rectangle ++(.5,.5);
      \path[fill=actOrange] (0,-.5) rectangle ++(.5,.5);
      \draw (-.5,-.5) rectangle (.5,.5);
      \draw (-.25,.25) node{$0$}; \draw (.25,.25) node{$0$};
      \draw (-.25,-.25) node{$0$}; \draw (.25,-.25) node{$0$};
      \draw (-.5,-.5) rectangle (.5,.5);
      \Loop[dist=1.5cm,dir=NO,color=eRed](-.5,.5);
      \Loop[dist=1.5cm,dir=NO,color=eBlue](.5,.5);
      \filldraw[fill=white,opacity=0.7] 
      (-.5,-.5)--(.5,-.5)--(.5,.5)--(-.5,.5)--cycle;
      \node at (0,0) {\large\bf $\mathbf{1}$};
      \node at (0,-1.2) {\normalsize $\bm{D_4\!=\!\<\Alert{r},\Balert{f}\>}$};
    \end{scope}
    %%
    %% the size-2 orbit
    %%
    \begin{scope}[shift={(0,0)},shorten >= -2pt, shorten <= -2pt] 
      \begin{scope}[shift={(1.75,0)}]  %% H
        \path[fill=actPurple] (-.5,0) rectangle ++(.5,.5); 
        \path[fill=actOrange] (0,0) rectangle ++(.5,.5);
        \path[fill=actOrange] (-.5,-.5) rectangle ++(.5,.5);
        \path[fill=actPurple] (0,-.5) rectangle ++(.5,.5);
        \draw (-.5,-.5) rectangle (.5,.5);
        \draw (-.25,.25) node{$1$}; \draw (.25,.25) node{$0$};
        \draw (-.25,-.25) node{$0$}; \draw (.25,-.25) node{$1$};
        \node (0-out) at (-.5,.1) {};
        \node (0-in) at (-.5,-.1) {};
        \filldraw[fill=white,opacity=0.7] 
        (-.5,-.5)--(.5,-.5)--(.5,.5)--(-.5,.5)--cycle;
        \node at (0,0) {\large\bf $\mathbf{3}$};
        \node at (0,-1.2) {\normalsize $\bm{\<r^2,rf\>}$};
      \end{scope}
      %%
      \begin{scope}[shift={(-1.75,0)}] %% Hr^3
        \path[fill=actOrange] (-.5,0) rectangle ++(.5,.5); 
        \path[fill=actPurple] (0,0) rectangle ++(.5,.5);
        \path[fill=actPurple] (-.5,-.5) rectangle ++(.5,.5);
        \path[fill=actOrange] (0,-.5) rectangle ++(.5,.5);
        \draw (-.5,-.5) rectangle (.5,.5);
        \draw (-.25,.25) node{$0$}; \draw (.25,.25) node{$1$};
        \draw (-.25,-.25) node{$1$}; \draw (.25,-.25) node{$0$};
        \node (180-out) at (.5,-.1) {};
        \node (180-in) at (.5,.1) {};
        \filldraw[fill=white,opacity=0.7] 
        (-.5,-.5)--(.5,-.5)--(.5,.5)--(-.5,.5)--cycle;
        \node at (0,0) {\large\bf $\mathbf{2}$};
        \node at (0,-1.2) {\normalsize $\bm{\<r^2,rf\>}$};
      \end{scope}
      \draw [rr] (0-out) to [bend right=15] (180-in);
      \draw [bb] (180-out) to [bend right=15] (0-in);
    \end{scope} %% end of the size-2 orbit
    %%
    %% the size-4 orbit
    %%
    \begin{scope}[shift={(6.75,0)},shorten >= 0pt, shorten <= 0pt]  
      \begin{scope}[shift={(1.75,0)}]  %% H
        \path[fill=actOrange] (-.5,0) rectangle ++(.5,.5); 
        \path[fill=actOrange] (0,0) rectangle ++(.5,.5);
        \path[fill=actPurple] (-.5,-.5) rectangle ++(.5,.5);
        \path[fill=actPurple] (0,-.5) rectangle ++(.5,.5);
        \draw (-.5,-.5) rectangle (.5,.5);
        \draw (-.25,.25) node{$0$}; \draw (.25,.25) node{$0$};
        \draw (-.25,-.25) node{$1$}; \draw (.25,-.25) node{$1$};
        \node (0-out) at (0,.5) {};
        \node (0-in) at (0,-.5) {};
        \filldraw[fill=white,opacity=0.7] 
        (-.5,-.5)--(.5,-.5)--(.5,.5)--(-.5,.5)--cycle;
        \node at (0,0) {\large\bf $\mathbf{6}$};
        \Loop[dist=1.5cm,dir=EA,color=eBlue](.5,0);
        \node at (.9,.9) {\normalsize $\bm{\<f\>}$};
      \end{scope}
      %%
      \begin{scope}[shift={(0,1.75)}] %% Hr
        \path[fill=actOrange] (-.5,0) rectangle ++(.5,.5); 
        \path[fill=actPurple] (0,0) rectangle ++(.5,.5);
        \path[fill=actOrange] (-.5,-.5) rectangle ++(.5,.5);
        \path[fill=actPurple] (0,-.5) rectangle ++(.5,.5);
        \draw (-.5,-.5) rectangle (.5,.5);
        \draw (-.25,.25) node{$0$}; \draw (.25,.25) node{$1$};
        \draw (-.25,-.25) node{$0$}; \draw (.25,-.25) node{$1$};
        \node (90-out) at (-.5,0) {};
        \node (90-in) at (.5,0) {};
        \node (90-bb) at (0,-.5) {};
        \filldraw[fill=white,opacity=0.7] 
        (-.5,-.5)--(.5,-.5)--(.5,.5)--(-.5,.5)--cycle;
        \node at (0,0) {\large\bf $\mathbf{7}$};
        \node at (-1.3,.5) {\normalsize $\bm{\<r^2f\>}$};
      \end{scope}
      %%
      \begin{scope}[shift={(-1.75,0)}] %% Hr^2
        \path[fill=actPurple] (-.5,0) rectangle ++(.5,.5); 
        \path[fill=actPurple] (0,0) rectangle ++(.5,.5);
        \path[fill=actOrange] (-.5,-.5) rectangle ++(.5,.5);
        \path[fill=actOrange] (0,-.5) rectangle ++(.5,.5);
        \draw (-.5,-.5) rectangle (.5,.5);
        \draw (-.25,.25) node{$1$}; \draw (.25,.25) node{$1$};
        \draw (-.25,-.25) node{$0$}; \draw (.25,-.25) node{$0$};
        \Loop[dist=1.5cm,dir=WE,color=eBlue](-.5,0);
        \node (180-out) at (0,-.5) {};
        \node (180-in) at (0,.5) {};
        \filldraw[fill=white,opacity=0.7] 
        (-.5,-.5)--(.5,-.5)--(.5,.5)--(-.5,.5)--cycle;
        \node at (0,0) {\large\bf $\mathbf{4}$};
        \node at (-1.1,-1) {\normalsize $\bm{\<f\>}$};
      \end{scope}
      %%
      \begin{scope}[shift={(0,-1.75)}] %% Hr^3
        \path[fill=actPurple] (-.5,0) rectangle ++(.5,.5); 
        \path[fill=actOrange] (0,0) rectangle ++(.5,.5);
        \path[fill=actPurple] (-.5,-.5) rectangle ++(.5,.5);
        \path[fill=actOrange] (0,-.5) rectangle ++(.5,.5);
        \draw (-.5,-.5) rectangle (.5,.5);
        \draw (-.25,.25) node{$1$}; \draw (.25,.25) node{$0$};
        \draw (-.25,-.25) node{$1$}; \draw (.25,-.25) node{$0$};        
        \node (270-out) at (.5,0) {};
        \node (270-in) at (-.5,0) {};
        \node (270-bb) at (0,.5) {};
        \filldraw[fill=white,opacity=0.7] 
        (-.5,-.5)--(.5,-.5)--(.5,.5)--(-.5,.5)--cycle;
        \node at (0,0) {\large\bf $\mathbf{5}$};
        \node at (1.2,-.5) {\normalsize $\bm{\<r^2f\>}$};
      \end{scope}
      \draw [r,shorten >= -2pt, shorten <= -2pt] (0-out)
      to [bend right=25] (90-in);
      \draw [r,shorten >= -2pt, shorten <= -2pt] (90-out)
      to [bend right=25] (180-in);
      \draw [r,shorten >= -2pt, shorten <= -2pt] (180-out)
      to [bend right=25] (270-in);
      \draw [r,shorten >= -2pt, shorten <= -2pt] (270-out)
      to [bend right=25] (0-in);
      \draw [bb,shorten >= -2pt, shorten <= -2pt] (90-bb) to (270-bb);
    \end{scope} %% end of the size-4 orbit
  \end{tikzpicture}}
  \]
  
\end{frame}

%%====================================================================

\begin{frame}{$G$-sets generalize groups. Action graphs generalize Cayley graphs} 
  
  The group $G=D_4=\<{\color{xRed}r},{\color{xBlue}f}\>$ can act on
  itself ($S=D_4$), or on its subgroups,
  \[
  S=\big\{D_4,\<r\>,\<r^2,f\>,\<r^2,rf\>,
  \<f\>,\<rf\>,\<r^2f\>,\<r^3f\>,\<r^2\>,\<1\>\big\}.
  \]

  \Pause
  
  There are several ways to define the result of
  \emph{``pressing the $g$-button on our switchboard''}. \medskip\Pause
  
  We say that: ``\emph{$G$ acts on\dots}''
  
  \vspace{-2mm}

  %% 3 action graphs of D_4, acting on itself by mult, conj, and
  %% subgroups by conjugation
  \[
  \begin{tikzpicture}[scale=.4,auto]
    \tikzstyle{v} = [circle, draw, fill=lightgray,inner sep=0pt,
      minimum size=3.65mm] 
    \tikzstyle{every node}=[font=\footnotesize]
    %%
    \begin{scope}[shift={(-.6,1.85)},scale=.75]
      \tikzstyle{r-out} = [draw, very thick, eRed,-stealth,bend right=33]
      \tikzstyle{r-in} = [draw, very thick, eRed,-stealth,bend left=25]
      %%
      \node (e) at (0:4) [v] {$1$};
      \node (r) at (90:4) [v] {$r$};
      \node (r2) at (180:4) [v] {$r^2$};
      \node (r3) at (270:4) [v] {$r^3$};
      \node (f) at (0:2) [v] {$f$};
      \node (r3f) at (270:2) [v] {$r^3\!f$};
      \node (r2f) at (180:2) [v] {$r^2\!f$};
      \node (rf) at (90:2) [v] {$rf$};
      \draw [r-out] (e) to (r);
      \draw [r-out] (r) to (r2);
      \draw [r-out] (r2) to (r3);
      \draw [r-out] (r3) to (e);
      \draw [r-in] (f) to (r3f);
      \draw [r-in] (r3f) to (r2f);
      \draw [r-in] (r2f) to (rf);
      \draw [r-in] (rf) to (f);
      \draw [bb] (e) to (f);
      \draw [bb] (r) to (rf);
      \draw [bb] (r2) to (r2f);
      \draw [bb] (r3) to (r3f);
      \node at (0,-7.5)
            {\footnotesize``\emph{\dots itself by right-multiplication}''};
    \end{scope}
    %%
    \begin{scope}[shift={(6,-2.05)},scale=2,yscale=1.25]
      \tikzstyle{every node}=[font=\small]
      %%
      \node (1) at (0,0) {$1$};
      \node (r2) at (1.65,0) {$r^2$};
      \node (r) at (0,1) {$r$};
      \node (r3) at (1.65,1) {$r^3$};
      \node (f) at (0,2) {$f$};
      \node (r2f) at (1.65,2) {$r^2f$};
      \node (rf) at (0,3) {$rf$};
      \node (r3f) at (1.65,3) {$r^3f$};
      \draw [bb] (r) to (r3);
      \draw [rr] (f) to (r2f);
      \path (f) edge [b,loop left,>=stealth] (f);
      \path (r2f) edge [b,loop right,>=stealth] (r2f);
      \draw [bb] (rf) to[bend right=15] (r3f);
      \draw [rr] (rf) to[bend left=15] (r3f);
      \path (1) edge [r,loop left,>=stealth] (1);
      \path (1) edge [b,loop right,>=stealth] (1);
      \path (r2) edge [r,loop left,>=stealth] (r2);
      \path (r2) edge [b,loop right,>=stealth] (r2);
      \path (r) edge [r,loop left,>=stealth] (r);
      \path (r3) edge [r,loop right,>=stealth] (r3);
      \node at (.875,-.7) {\footnotesize``\emph{\dots itself by conjugation}''};
    \end{scope}
    %%
    \begin{scope}[shift={(18.5,-2.5)},scale=2.2,
        shorten >= 0pt, shorten <= 0pt,yscale=1.2]
      \tikzstyle{every node}=[font=\footnotesize]
      %%
      \node (D4) at (0,3) {$D_4$};
      \node (r2-f) at (-1.25,2) {$\<r^2,f\>$};
      \node (r) at (0,2) {$\<r\>$};
      \node (r2-rf) at (1.25,2) {$\<r^2,rf\>$};
      %%
      \node (f) at (-2.5,1) {$\<f\>$};
      \node (r2f) at (-1.25,1) {$\<r^2f\>$};
      \node (r2) at (0,1) {$\<r^2\>$};
      \node (r3f) at (1.25,1) {$\<r^3f\>$};
      \node (rf) at (2.5,1) {$\<rf\>$};
      %%
      \node (1) at (0,0) {$\<1\>$};
      %%
      \draw[faded] (D4) to (r2-f);
      \draw[faded] (D4) to (r);
      \draw[faded] (D4) to (r2-rf);
      \draw[faded] (r2-f) to (f);
      \draw[faded] (r2-f) to (r2f);
      \draw[faded] (r2-f) to (r2);
      \draw[faded] (r2-rf) to (rf);
      \draw[faded] (r2-rf) to (r3f);
      \draw[faded] (r2-rf) to (r2);
      \draw[faded] (r2) to (r);
      \draw[faded] (r2) to (1);
      \draw[faded] (f) to (1);
      \draw[faded] (r2f) to (1);
      \draw[faded] (r3f) to (1);
      \draw[faded] (rf) to (1);
      \tikzset{every loop/.style={min distance=5mm}}
      \path (D4) edge [r,loop left,>=stealth] (D4);
      \path (D4) edge [b,loop right,>=stealth] (D4);
      \path (r2-f) edge [r,loop left,>=stealth,min distance=4mm] (r2-f);
      \path (r2-f) edge [b,loop right,>=stealth,min distance=4mm] (r2-f);
      \path (r) edge [r,loop above,>=stealth] (r);
      \path (r) edge [b,loop below,>=stealth] (r);
      \path (r2-rf) edge [r,loop left,>=stealth,min distance=4mm] (r2-rf);
      \path (r2-rf) edge [b,loop right,>=stealth,min distance=4mm] (r2-rf);
      \path (f) edge [b,loop above,>=stealth,min distance=4mm] (f);
      \path (r2f) edge [b,loop above,>=stealth,min distance=4mm] (r2f);
      \draw [rr] (f) to (r2f);
      \draw [rr] (rf) to[bend right=15] (r3f);
      \draw [bb] (rf) to[bend left=15] (r3f);
      \path (r2) edge [r,loop left,>=stealth] (r2);
      \path (r2) edge [b,loop right,>=stealth] (r2);
      \path (1) edge [r,loop left,>=stealth] (1);
      \path (1) edge [b,loop right,>=stealth] (1);
      \node at (0,-.5)
            {\footnotesize``\emph{\dots its subgroups by conjugation}''};
    \end{scope}
  \end{tikzpicture}
  \]
  
  \vspace{-2mm}\Pause
  
  \begin{alertblock}{Big idea}
    Every Cayley graph is the action graph of a particular
    group action.
  \end{alertblock}
  
\end{frame}

%%====================================================================

\begin{frame}{Left actions vs. right actions (an annoyance we can deal with)} 
  %\Pause
  
  As we've defined group actions, ``\emph{pressing the $a$-button
    followed by the $b$-button should be the same as \Alert{pressing
      the $ab$-button}}.''
  
  \medskip\Pause
  
  However, sometimes it appears like it's the same as
  ``\emph{\Balert{pressing the $ba$-button}}.''
  
  \medskip\Pause
  
  This is best seen by an example. Suppose our action is
  conjugation: \vspace{-3mm}
  
  %% Left vs. right conjugation of a subgroup H
  \[
  \begin{tikzpicture}[scale=.5] 
    \tikzstyle{to} = [draw,-stealth]
    %%
    \begin{scope}[shift={(0,0)}]
      \node (H) at (0,0) {\small \color{xBlue}$H$};
      \node (aH) at (4,0) {\small \color{xBlue}$aHa^{-1}$};
      \node (baH) at (9,0) {\small \color{xBlue}$baHa^{-1}b^{-1}$};
      \draw[to] (H) to (aH);
      \draw[to] (aH) to (baH);
      \draw[to,bend right=23] (H) to (baH);
      \node at (1.75,.7) {\tiny conjugate};
      \node at (1.75,.3) {\tiny by $a$};
      \node at (6.25,.7) {\tiny conjugate};
      \node at (6.25,.3) {\tiny by $b$};
      \node at (4.25,-.88) {\tiny conjugate by $ba$};
      \node at (4.5,-2) {\Balert{$\phi(a)\phi(b)=\phi(ba)$}};
      \node at (4.5,1.75) {\Balert{``Left group action''}};
    \end{scope}
    %%
    \begin{scope}[shift={(12,0)}]
      \node (H) at (0,0) {\small \Alert{$H$}};
      \node (aH) at (4,0) {\small \Alert{$a^{-1}Ha$}};
      \node (baH) at (9,0) {\small \Alert{$b^{-1}a^{-1}Hab$}};
      \draw[to] (H) to (aH);
      \draw[to] (aH) to (baH);
      \draw[to,bend right=23] (H) to (baH);
      \node at (1.75,.7) {\tiny conjugate};
      \node at (1.75,.3) {\tiny by $a$};
      \node at (6.25,.7) {\tiny conjugate};
      \node at (6.25,.3) {\tiny by $b$};
      \node at (4.25,-.88) {\tiny conjugate by $ab$};
      \node at (4.5,-2) {\Alert{$\phi(a)\phi(b)=\phi(ab)$}};
      \node at (4.5,1.75) {\Alert{``Right group action''}};
    \end{scope}
  \end{tikzpicture}
  \]
  
  \Pause %%

  We'll call $aHa^{-1}$ the \Balert{left conjugate} of $H$ by $a$, and
  $a^{-1}Ha$ the \Alert{right conjugate}. \medskip\Pause
  
  Some books forgo our ``$\phi$-notation'' and use the following
  notation to distinguish left vs. right group actions:
  \[
    {\color{xBlue}g\Dot(h\Dot s)=(gh)\Dot s},
    \qquad\qquad\Alert{(s\Dot g)\Dot h=s\Dot(gh)}.
  \]
  \Pause We'll usually keep the $\phi$, and write
  {\color{xBlue}$\phi(g)\phi(h)s=\phi(gh)s$} and
  \Alert{$s\Dot\phi(g)\phi(h)=s\Dot\phi(gh)$}. As with groups, the ``dot''
  will be optional.
  
\end{frame}

%%====================================================================

\begin{frame}{Left actions vs. right actions (an annoyance we can deal with)} 

  \begin{block}{Alternative definition (other textbooks)}
    A \Alert{right group action} is a mapping
    \[
    G\times S\longto S\,,\qquad (a,s)\longmapsto s\Dot a
    \]
    \Pause such that 
    \begin{itemize}
    \item $s\Dot(ab)=(s\Dot a)\Dot b$,\; for all $a,b\in G$ and $s\in S$ \Pause
    \item $s\Dot e=s$,\; for all $s\in S$.
    \end{itemize}
  \end{block}

  \medskip\Pause 
  
  A {\color{xBlue}left group action} is defined similarly. Theorems
  for left actions have analogues for right actions.

  \Pause\medskip

  Each left action has a related right action, and vice-versa. \Pause
  \Alert{We'll use right actions}, and write
  \[
  \Alert{s\Dot\phi(g)}
  \]
  for ``\emph{the element of $S$ that the permutation $\phi(g)$ sends $s$
  to},'' i.e., where pressing the $g$-button sends $s$.
 
  \Pause\medskip

  If we have a left action, we'll write {\color{xBlue}$\phi(g)\Dot s$}. 

  \Pause\medskip

  If needed, we can distinguish \Alert{left $G$-sets} with
  \Balert{right $G$-sets}.


\end{frame}%%====================================================================

\section*{The end!}
%%====================================================================

\end{document}